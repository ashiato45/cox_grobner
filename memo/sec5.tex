\label{sec:多様体上の多項式関数と有理関数}

\subsection{多項式写像}
\label{sub:多項式写像}
\begin{enumerate}[label=(\arabic*)]
  \item
  $V=\set{(t,t^2,t^3)}$。
  \begin{align}
    \phi(t,t^2,t^3)=
    (t(t^2),t^3+t^2(t^2)^2)
    =
    (t^3, t^3+t^6).
  \end{align}
  \begin{align}
    v-u-u^2 = (t^3+t^6)-t^3-(t^3)^2 = 0.
  \end{align}
  よって、$\phi(V) \subset W$となる。
  \item
  $\set{(t^2-t,t^2,t-3t^2)}$の陰関数表示を求める。
  \begin{align}
    x=t^2-t,\quad y = t^2,\quad z=t-3t^2
  \end{align}
  から、$\gen{x-t^2+t,\, y-t^2,\, z-t+3t^2}$の
  グレブナ基底を求める。
  %\insertcalc{ex5112.tex}
  よって、このグレブナ基底は
  \begin{itemize}
    \item $y^4 + \frac{2}{3}y^3 z - \frac{10}{9}y^3 + \frac{1}{9}y^2 z^2 -\frac{2}{3}y^2 z + \frac{1}{9}y^2 - \frac{1}{9}yz^2$
    \item $xy + 2y^2 + yz$
    \item $x^2 + 5y^2 + 2yz - y$
    \item $t + x - y$
  \end{itemize}
  である。よって、先のイデアルの1次消去イデアルを考えれば、$\phi(\R^2)$
  を包む最小の多様体になる。さらに、$\phi(\R^2)$が$\var(I_1)$と一致することを示す。
  任意の$(x,y,z)\in \var(I_1)$をとる。
  これに対し、条件をみたす$t$があること、解が拡張できることを示せばよい。
  まず、$t+x-y$の$t$の係数が$1$で係数なので、常にこれは$\C$で拡張できる。
  さらに、$x,y,z\in \R$なので、$t+x-y$を見れば$t\in \R$となる。
  よって、常に拡張できて、一致する。
  \item

\end{enumerate}

\subsection{多項式環の商}
\label{sub:多項式環の商}

\subsection{$k[x_1,\dots,x_n]/I$のアルゴリズム的計算}
\label{sub:のアルゴリズム的計算}

\subsection{アフィン多様体の座標環}
\label{sub:アフィン多様体の座標環}

\begin{framed}
命題3:
$V\subset k^n$をアフィン多様体とする。
\begin{enumerate}[label=(\roman*)]
  \item イデアル$J\subset k[V]$に対して、$W=\var_V(J)$は
  $V$に含まれる$k^n$のアフィン多様体となる。
  \item
  部分集合$W\subset V$に対して、$\ideal_V(W)$は$k[V]$のイデアルとなる。
  \item
  イデアル$J\subset k[V]$に対して、$J\subset \sqrt{J} \subset \ideal_V(\var_V(J))$
  \item
  部分多様体$W\subset V$に対して、$W=\var_V(\ideal_V(W))$。
\end{enumerate}
\end{framed}
まず、(i)を示す。
\begin{enumerate}
  \item
  $J \subset k[V] \simeq k[x_1,\dots,x_n]/\ideal(V)$なので、
  対応する$k[x_1,\dots,x_n]$のイデアル
  $\tilde J$が存在する。$\ideal(V) \subset \tilde J$となる。
  \item
  \begin{align}
    \tilde J = \set{f \in k[x_1,\dots,x_n]; [f]\in J}.
  \end{align}
  \item
  $\gen{0}\subset J$に対応して、$\ideal(V) \subset \tilde J$となる。
  \item
  上で両辺$\var$をとって、$\var(J)\subset V$となる。
  \begin{align}
    \var(J)
    &=
    \set{p\in k^n; \Forall{[f]\in J}[f](p)=0}\\
    &=
    \set{p\in k^n; \Forall{[f]\in J}f(p)=0}\\
    &=
    \set{p\in k^n; \Forall{f\in \tilde J} f(p)=0}\\
    &=
    \var(\tilde J)\\
    &\subset
    \var(\ideal(V))\\
    &=
    V.
  \end{align}
  \item
  $\var(\tilde J) = \var_V(J)$
  \item 4,5より$\var(\tilde J) \subset V$
\end{enumerate}

\newpage
\begin{enumerate}
  \item $V=\var(I),\, V\neq \emptyset$
  \item
  ($V$が既約のとき) ($I=\ideal(V)$としてよい。)
  \begin{enumerate}
    \item ($l=1$のとき)
  \begin{enumerate}
    \item $W_0 \subsetneq V$
    \item $\exists$: $(a_1,\dots,a_n) \in V-W_0$
    \item $\exists$: $f\in \ideal(W_0)$、$f(a_1,\dots,a_n) \neq 0$
    \item (場合1)
    \begin{enumerate}
      \item $m,g_\bullet$:
      \begin{align}
        f=\sum_{i=0}^m g_i(x_2,\dots,x_n) x_i^i
      \end{align}
      \item $W_1 = \var(I_1)\cap \var(g_0,\dots,g_m)$
      \item $(c_2,\dots,c_n) \in \var(I_1)-W_1$
      \item $\exists$: $c_1 \in k$、$f(c_1,\dots,c_n) \neq 0$
    \end{enumerate}
    \item (場合2)
    \begin{enumerate}
      \item $\exists (b_2,\dots,b_n) \in \var(I_1) \exists b_1 \in k$:
      $(b_1,\dots,b_n) \notin V$
      \item $\exists h\in I$: $h(b_1,\dots,b_n) \neq 0$
      \item $r, u_\bullet$:
      \begin{align}
        h=\sum_{i=0}^r u_i(x_2,\dots,x_n) x_1^i.
      \end{align}
      \item 「(4)~を示そう」
      \begin{itemize}
        \item $0\le j\le r,\, N_j,\, v_{j0},\dots, v_{j,r-1}$:
        \begin{align}
          u_r^{N_j}f^j =
          q_j h + \ub{v_{j0}}_{\in k[x_2,\dots,x_n]} + v_{j1}x_1  + \dots + v_{j,r-1}x_1^{r-1}.
        \end{align}
        \item $K$: $k[\var(I_1)]$の分数体
        \item $\exists$ $\phi_0,\dots,\phi_r \in K$(すべては0でない):
        \begin{align}
          \Forall{0\le i \le r-1} \sum_{j=0}^r \phi_j [v_{ji}] = [0]
        \end{align}
        あるいは、
        \begin{align}
          \sum_{j=0}^r \phi_j ([v_{j0}],\dots,[v_{j,r-1}]) = ([0],\dots, [0]).
        \end{align}
        \item
        とりなおし:$\phi_\bullet \in k[x_2,\dots,x_n]/I_1$
        \item
        $\exists w_j \in k[x_2,\dots,x_n]$: $\phi_j = [w_j]$.
        うち少なくとも1つは$w_j \notin I_1$
        \item
        $ v_j = w_j u_r^{N_j}$
      \end{itemize}
      \item $g=u_r v_0$
      \item $W_1 = \var(g) \cap \var(I_1)$
    \end{enumerate}
  \end{enumerate}
  \item $l-1$について:
  \end{enumerate}
  \item ($V$は既約とはかぎらない)
  \item $\exists$: $V_\bullet: V = V_1\cup \dots \cup V_m.$既約
  \item
  $V_i'$: $\pi_l(V_i)$のザリスキ閉包
  \item
  $V_1' \not\subset V_i'$
  \item
  $W_1$: $V_1$と$\emptyset$に定理を適用したもの。
  $V_1' - W_1 \subset \pi_l(V_1)$、$W_1 \subsetneq V_1'$。
  \item
  $W = W_1 \cup V_2' \cup \dots \cup V_m'$
\end{enumerate}

\begin{enumerate}
  \item $W_1 = \var(I_l)$
  \item
  $Z_1$: 閉包定理。$W_1 - Z_1 \subset \pi_l(V)$、$Z_1 \subsetneq W_1$
  \item $V_1$:
  \begin{align}
    V_1 = V \cap \set{(a_1,\dots,a_n)\in k^n; (a_{l+1},\dots,a_n)\in Z_1}
  \end{align}
  $V_1 \subsetneq V$、$\pi_l(V) = (W_1-Z_1) \cup \pi_l(V_1)$
  \item
  $W_2$: $\pi_l(V_1)$のザリスキ閉包
  \item
  $Z_2$:閉包定理。$W_2 - Z_2 \subset \pi_l(V_1)$、$Z_2 \subsetneq W_2$
  \item
  $V_2$:
  \begin{align}
    V_2 = V_1 \cap \set{(a_1,\dots,a_n)\in k^n; (a_{l+1},\dots,a_n)\in Z_2}
  \end{align}
  \item
  3-4を、できた$V_\bullet$が$\emptyset$になるまでくりかえす。
\end{enumerate}

\subsubsection{演習}
\label{subs:演習}
\begin{enumerate}[label=(\arabic*)]
  \item
  \begin{enumerate}[label=(\alph*)]
    \item
    $a,b \notin I\cap k[x_{l+1},\dots,x_n]$とする。
    \begin{itemize}
      \item $a\notin I$かつ$b\notin I$のとき:
      $I$が素イデアルなので、$ab \notin I$となる。よって、
      $ab \notin I_l$となる。
      \item $a\notin I$かつ$b\notin k[x_{l+1},\dots,x_n]$のとき:
      $b$は$x_1,\dots,x_l$の1文字以上を含まなければならない。
      よって、$ab$も同様で、$ab\notin k[x_{l+1},\dots,x_n]$となる。
      よって、$ab\notin I_l$となる。
      \item $a\notin k[x_{l+1},\dots,x_n]$かつ$b\notin I$のとき:
      上と同様。
      \item $a,b \notin k[x_{l+1},\dots,x_n]$のとき:
      上と同様。
    \end{itemize}
    \item
    ?案。
    $\var(I_l)$が既約でないとし、
    \begin{align}
      \var(I_l) &= \var(fg,h_1,h_2,\dots,h_s)\\
      &=
      \var(f,h_1,\dots,h_s) \cup \var(g,h_1,\dots,h_s)
    \end{align}
    であるとする。ただし、最後2つで$\var(I_l)$を覆い、
    どちらかだけで全てを覆うということはない(既約でないから)。

    $\var(I_l)$は$k[x_{l+1},\dots,x_n]$のsubsetだが、
    これを$k[x_1,\dots,x_n]$で捉えると、
    $\var(I)$よりも広い(?)。
    したがて、
    \begin{align}
      \var(I)
      &=
      \var(F_1,\dots,F_s)\\
      &=
      \var(F_1,\dots,F_S)\cap \var(I_l)\\
      &=
      \ub{(\var(F_1,\dots,F_S)\cap \var(f,h_1,\dots,h_s))}_{A}
      \cup
      \ub{(\var(F_1,\dots,F_S)\cap V(g,h_1,\dots,h_s))}_{B}
    \end{align}
    $A$と$B$で$\var(I)$が覆えていて、しかもどちらかだけで覆ってはいない。
    (後者は、先の「どちらかだけで全て$(\var(I_l))$を覆うことはない」による。)
    よって、$\var(I)$は既約でなく、対偶がしめされた。
  \end{enumerate}
  \item
  体でやって戻せばいいと思っていた。つまり、
  $h$を$u_r$でわって先頭の$x_1$に関する係数を$1$にしておく。
  このとき、わったあとの係数の分母はすべて$u_r$のべきになる。
  これで$k(x_2,\dots,x_n)$係数で割り算して、最後に
  $u_r$のべきを解消する。
  \item
  \begin{enumerate}[label=(\alph*)]
    \item
    $x=(z,y) \in V_y $とする。$\pi_1(x) = y$となる。
    $(z,y)\in \C \times \set{y}$はあきらか(実際はもっとせまい)。

    $V_y\neq \emptyset$とする$\pi_1(x)=y$となる$x=(z,y) \in V$が存在する。
    $\pi_1(x) = y$なので、$y$にうつる$V$の元として
    $x$が見つかったことになり、$y \in\pi_1(V)$である。
    逆に、$y\in \pi_1(V)$とする。$\pi_1(x) = y$となる$x\in V$が存在する。
    すると、$x\in V_y$である。よって、$V_y \neq \emptyset$である。
    \item
    場合1は
    \begin{quote}
      すべての$(b_2,\dots,b_n)\in \var(I_1)$とすべての
      $b_1 \in k$にたいして、$(b_1,\dots,b_n) \in V$となる
    \end{quote}
    ときだった。
    $\pi_1(V)\subset \var(I_1)$は一般に成立するが、
    さらに部分解$\var(I_1)$から筒状に伸ばしたものが$V$であるというのがこの場合であるから、
    $\pi_1(V) = \var(I_1)$である。

    $V_y \subset \C \times \set{y}$は一般に成立する。
    $(z,y)\in \C \times \set{y}$とする。$y\in \pi_1(V)=\var(I_1)$であり、
    $z\in \C$なので、$(z,y)\in V$である(場合1)。
    さらに、$\pi_1(z,y)  = y$なので、$(z,y) \in V_y$となる。
    \item
    場合2は
    \begin{quote}
      ある$(b_2,\dots,b_n) \in \var(I_1)$と
      $b_1 \in k$に対して$(b_1,\dots,b_n) \notin V$となる
    \end{quote}
    場合だった。$I=\ideal(V)$としてあるので、
    (与えられた$b_\bullet$を使って)$h(b_1,\dots,b_n)\neq 0$となる
    多項式$h\in I$が存在する。
    $h$を$x_1$の多項式として、
    \begin{align}
      h=\sum_{i=0}^r u_i(x_2,\dots,x_n)x_1^i
    \end{align}
    とあらわす。$h(b_1,\dots,b_n)\neq 0$より、ある$i$に対して
    $u_i(b_2,\dots,b_n)\neq 0$となり、$u_i \notin I_1$となる
    \footnote{これは最高次の$x_1^r$の係数ではないかもしれない}。
    もし、$u_r \in I_1$ならば$h-u_r x_1^r$も$(b_1,\dots,b_n)$
    で消えない\footnote{最高次をつぶして$h$を置き換える。
    このとき、消えないのは$(b_2,\dots,b_n)\in \var(I_1)$という仮定
    が条件2でかかっていたから。}から、$h-u_r x_1^r$で置き換えることができる。
    このおきかえを繰替えして(次数を下げて)$u_r \notin I_1$と仮定してもよい。

    $\tilde W = \var(u_r)$とする。
    まず、$\pi_1(V) \not\subset \tilde W$をしらべる。
    $u_r \notin I_1$となるようにしておいたので、
    $u_r(b_2,\dots,b_n) \neq 0$である。すると、
    \begin{align}
      h(x_1,b_2,\dots,b_n) = \sum_{i=0}^r u_i(b_2,\dots,b_n)x_1^i
    \end{align}
    という$x_1$についての方程式が得られる。
    代数学の基本定理より、これをみたす$\tilde b_1$が存在する。
    ?

  \end{enumerate}
  \item
  \begin{align}
    (\tilde \pi_{l-1} \circ \pi_1)(x_1,\dots,x_n)
    &=
    \tilde \pi_{l-1}(x_2,\dots,x_n)\\
    &=
    (x_{n-l+1},\dots,x_n)\\
    &=
    \pi_l (x_1,\dots,x_n).
  \end{align}
  \item
  \begin{enumerate}[label=(\alph*)]
    \item
  \begin{align}
    (V \subset V_1\cup V_2)
    &\iff
     (\ideal(V) \supset \ideal(V_1 \cup V_2))\\
     &\iff
     (\ideal (V) \supset \ideal(V_1) \ideal(V_2))\\
     &\iff
     ((\ideal (V) \supset \ideal(V_1)) \vee (\ideal(V)\supset \ideal(V_2)))\\
     &\iff
     ((V\subset V_1)\vee (V\subset V_2)).
  \end{align}
  \item
  $n=2$のときは示されている。
  $n$で成立するとする。$n+1$のとき示す。
  \begin{align}
    (V \subset V_1\cup \dots \cup V_{n+1})
    &\iff
    (V \subset (V_1 \cup \dots \cup V_n) \cup V_{n+1})\\
    &\iff
    ((V\subset (V_1\cup \dots \cup V_n))\vee (V\subset V_{n+1}))\\
    &\iff
    ((V\subset V_1)\vee \dots \vee (V\subset V_{n+1})).
  \end{align}
  \end{enumerate}
  \item
  \item
  \begin{enumerate}[label=(\alph*)]
    \item
    \begin{align}
      Z_1 &= \var_{W_1}([\phi_1],\dots,[\phi_s])\\
      &=
      \set{(x_1,\dots,x_{n-l})\in W_1; \Forall{[\phi] \in \gen{[\phi_1],\dots,[\phi_{s}]}} [\phi](x_1,\dots,x_{n-l}) = 0}\\
      &=
      \set{(x_1,\dots,x_{n-l})\in W_1; \Forall{\phi \in \gen{\phi_1,\dots,\phi_{n-l}}} \phi(x_1,\dots,x_{n-l}) = 0}.
    \end{align}
    よって、
    \begin{align}
      \set{(a_1,\dots,a_n)\in k^n; (a_{l+1},\dots,a_n)\in Z_1}
      &=
      \set{(a_1,\dots,a_n)\in k^n; \Forall{\phi\in \gen{\phi_1,\dots,\phi_{n-l}}} \phi(a_{l+1},\dots,a_n)  = 0}\\
      &=
      \gen{\phi_1,\dots,\phi_{n-l}}.
    \end{align}
    これはアフィン多様体になっている。$V_1$はこれと$V$との
    交わりなのでOK。
    \item
    \begin{align}
      (W_1-Z_1)\cup \pi_l(V_1)
      &=
      (W_1-Z_1)\cup \pi_l(\set{(a_1,\dots,a_n)\in k^n ; (a_{l+1},\dots,a_n)\in Z_1}\cap V)\\
      &=
      (W_1-Z_1)\cup (Z_1\cap \pi_l(V))\\
      &=
      ((W_1-Z_1)\cup Z_1) \cap ((W_1-Z_1)\cup \pi_l(V))\\
      &=
      W_1 \cap ((W_1-Z_1)\cup \pi_l(V))\\
      & \fbox{$Z_1 \subsetneq W_1$}\\
      &=
      W_1 \cap \pi_l(V)\\
      &\fbox{$W_1 - Z_1 \subset \pi_l(V)$(閉包定理)}\\
      &=
      \var(V_1) \cap \pi_l(V)\\
      &=
      \pi_l(V)\\
      &
      \fbox{$\pi_l(V)$のほうがせまい}.
    \end{align}
  \end{enumerate}
\end{enumerate}

\begin{framed}
  定理1(閉包定理の後半)
  $k$を代数的閉体とし、$V=\var(I)\subset k^n$とする。
  $V\neq \emptyset$ならば、
  \begin{align}
    \var(I_l) - W \subset \pi_l(V)
  \end{align}
  となるようなアフィン多様体$W\subsetneq \var(I_l)$が存在する。
\end{framed}

\begin{enumerate}
  \item ???:$l=1$のときは済んでいる。
  \item ???:$l>1$を考える。

  \item
  $V=\var(I)$としていたが、$V=\var(\ideal(V))$が成立しているので、
  $I=\ideal(V)$としてもよい。
  (「$V$を定義するどのイデアル$I$も同じ$\var(I_l)$を与える。)
  \item
  $V$が既約なとき:
  \begin{enumerate}
    \item
    $V$は既約なので、$I=\ideal(V)$は素イデアル
    \item
    Fact: $I$が素イデアル$\implies$ $I_l$は素イデアル。演習1。

    略証:$a,b \notin I\cap k[x_{l+1},\dots,x_n]$とする。あとは4通りにわける。
    $I$が素イデアルであることを使うパートと、$a,b$が$x_1,\dots,x_l$を含んでしまうパートに分かれる。
    \item
    Fact: $V$が既約$\implies$ $\var(I_l)$は既約。

    略証: (3)で$I=\ideal(V)$とした。$I$は素イデアルなので(a)、
    $I_l$も素イデアルであり(b)、代数的閉体上では
    素イデアルと既約多様体が対応するので
    \footnote{4-5-Prop3 一般の体で、$V$が既約$\iff$ $\ideal(V)$は素イデアル。}
    、$\var(I_l)$は既約。
    \item
    「$\var(I_l)-W \subset \pi_l(V)$となる$W\subsetneq \var(I_l)$が存在する」
    よりも強い、
    \begin{framed}
      任意の多様体$W_0 \subsetneq V$にたいして、
      \begin{align}
        \var(I_l) - W_l \subset \pi_l(V-W_0)
      \end{align}
      となる多様体$W_l \subsetneq \var(I_l)$が存在する
    \end{framed}
    を示す。
    \begin{enumerate}
      \item $l=1$のとき
      \item $\exists a_\bullet$: $W_0\neq V$なので、$(a_1,\dots,a_n) \in V-W_0$なる点が存在する。
      \item $\exists f$: $f\in \ideal(W_0)$($W_0$できえる)で、$f(a_1,\dots,a_n) \neq 0$となる多項式が存在する。(なぜ?)
      \item 場合I: すべての$(b_2,\dots,b_n) \in \var(I_l)$とすべての
      $b_1 \in k$に対して$(b_1,\dots,b_n)\in V$となる場合:
      \begin{enumerate}
        \item $m,g_\bullet$: $f$を$x_1$について$m$次であり、
        \begin{align}
          f=\sum_{i=0}^m g_i(x_2,\dots,x_n) x_1^i
        \end{align}
        とかく。
        \item $\exists W_1$:
        $W_1 = \var(I_1) \cap \var(g_0,\dots,g_m)$とする。(これが条件をみたす)
        \item Fact: $W_1 \subsetneq \var(I_1)$である。($\subset$はあきらか。)

        実際$(a_2,\dots,a_n) \in \var(I_1)\setminus W_1$である。なぜなら、
        $f(a_1,\dots,a_n) \neq 0$なので(iii)、$g_i$のどれかは$(a_2,\dots,a_n)$
        で非零である。よって、$(a_2,\dots,a_n)\notin W_1$にはなっている。
        また、$(a_1,\dots,a_n) \in V$なので$(a_2,\dots,a_n)$は
        その部分解$\var(I_1)$になっている。
        \item $\forall c_\bullet$:
        $(c_2,\dots,c_n) \in \var(I_1)-W_1$とする。
        \item
        $(c_1,\dots,c_n)\notin W_1 = \var(I_1)\cap \var(g_0,\dot,g_m)$
        なので、$g_0,\dots,g_m$のいずれかで消えない。
        \item
        よって、$f(x_1,\dots,c_2,\dots,c_n)\in k[x_1]$は非零な多項式。
        \item $\exists c_1$:
        $k$は無限体なので、$f(c_1,\dots,c_n) \neq 0$となる$c_1 \notin k$が存在する。
        \item
        $f$は$W_0$で消えるようにとっていたので(iii)、$f$で消えないやつ$(c_1,\dots,c_n) \notin W_0$。
        \item
        場合Iの仮定の、ファイバーがちゃんとのびているというやつより、
        $(c_1,\dots,c_n) \in V$。
        \item
        $(c_1,\dots,c_n) \in V-W_0$となる。(H,I)
        \item
        $(c_2,\dots,c_n) \in \pi_1(V-W_0)$となる。
      \end{enumerate}
      \item 場合 II: ある$(b_2,\dots,b_n) \in \var(I_1)$と
      ある$b_1 \in k$に対して$(b_1,\dots,b_n) \notin V$となるとき。
      \begin{enumerate}
        \item $\exists h$:
        $(b_1,\dots,b_n)\notin V$なので、
        $h(b_1,\dots,b_n)\neq 0$となる$h\in I$が存在する。(なぜ?)
        \item
        $r,u_i$: $h$を$x_1$について整理する。
        \begin{align}
          h = \sum_{i=0}^r u_i(x_2,\dots,x_n) x_1^i.
        \end{align}
        \item
        $h(b_1,\dots,b_n) \neq 0$なので、ある$i$について
        $u_i(b_2,\dots,b_n) \neq 0$となり($(b_2,\dots,b_n)\notin \var(I_1)$なので)、$u_i \notin I_1$となる。
        \item
        $u_r \in I_1$(最高次)ならば、$h-u_rx_1^r$も$(b_1,\dots,b_n)$
        で消えないのでこれを置き換えて、最高次$u_r\notin I_1$となるようにできる。
        \item
        \begin{framed}
          $v_i \in k[x_2,\dots,x_n]$で、
          \begin{align}
            \sum_{i=0}^r v_i f^i \in I かつ
            v_0 \notin I_1
          \end{align}
          なるものが存在することを示す。
        \end{framed}
        \item
        $q,v_\bullet$:
        $0\le j \le r$にたいして、
        \begin{align}
          u_r^{N_j} f^j = q_j h+ v_{j0}+ v_{j1}x_1 + \dots + v_{j,r-1}x_1^{r-1}.
        \end{align}
        とする。$f$を$k(x_2,\dots,x_n)$係数で割り算して最後に払えばよい。
        \item
        $I_1 = \ideal(\var(I_1))$だたので、$k[x_2,\dots,x_n]/I_1 \simeq k[\var(I_1)]$となる。
        \item $K$:
        $\var(I_1)$は既約なので(a,b)、この環は整域で分数体$K$が考えられる。
        \item
        $K$を元とする$(r+1)\times r$行列
        \begin{align}
          \begin{pmatrix}
            [v_{00}] & \ldots & [v_{0,r-1}]\\
            \vdots & & \vdots \\
            [v_{r0}] & \ldots & [v_{r,r-1}]
          \end{pmatrix}
        \end{align}
        を作る。横に先の割り算の結果が並んで、縦には$1,f,\dots,f^r$となっている。
        \item $\exists \phi_\bullet$:
        行は$r+1$個あり、その行たちは$K^r$に属しているので、線型従属であり、
        係数$\phi_0,\dots,\phi_r \in K$で、
        \begin{align}
          0\le i \le r-1 \implies \sum_{j=0}^r \phi_j [v_{ji}] = [0]
        \end{align}
        となるものがある。あるいは、
        \begin{align}
          \sum_{j=0}^r \phi_j ([v_{j0}] ,\dots, [v_{j,r-1}]) = ([0],\dots,[0]).
        \end{align}
        \item
        $\phi_\bullet$たちの分母を払って、$\phi_\bullet \in k[x_2,\dots,x_n] \I_1$と思ってよい。
        \item
        $w_\bullet$: $\phi_j = [w_j]$となる$w_j \in k[x_2,\dots,x_n]$が存在する。
        \item
        $\phi_\bullet$すべては0ではないのだから、$w_\bullet$の少なくとも1つは$I_1$に入らない。
        \item Jを書き直せば、
        \begin{align}
          \sum_{j=0}^r [w_j]([v_{j0}],\dots,[v_{j,r-1}]) = ([0],\dots,[0]).
        \end{align}
        \item
        上は、
        \begin{align}
          \sum_{j=0}^r w_j(v_{j0},\dots,v_{j,r-1}) \in (I_1)^r
        \end{align}
        であり、
        \begin{align}
          \Forall{i} \sum_{j=0}^r w_j v_{ji} \in I_1
        \end{align}
        となっている。
        \item
        擬除算の式に$w_j$をかけて$\sum_{j=0}^r$をとる。
        \begin{align}
          \sum_{j=0}^r w_j(u_r^{N_j} f^j) \mod I
          &=
          \sum_{j=0}^r w_j(q_j h + v_{j0} + v_{j1}x_1 + \dots + v_{j,r-1}x_1^{r-1}) \mod I \\
          &=
          \sum_{j=0}^r w_j (q_j h) \mod I\\
          & \fbox{O, $\sum w_j v_{ji} \in I_1$}\\
          &=
          0 \mod I\\
          & \fbox{Aより。$h\in I$}.
        \end{align}
        よって、
        \begin{align}
          \sum_{j=0}^r w_j(u_r^{N_j} f^j)\in I.
        \end{align}
        \item
        $v_\bullet$: $v_j = w_j u_r^{N_j}$
        \item
        $u_r \notin I_1$であり(擬除算「分母」にするためだった。Dより。)
        、
        ある$j$に対し$w_j \notin I_1$であるから(線型従属より。M。)
        、
        $I_1$が素イデアルであること(既約と仮定している。a,b)より、$v_j \notin I_1$となる。
        いまのところ、多項式として
        \begin{align}
          \sum_{v_j f^j}
        \end{align}
        まで作った。うちどれか$v_j \notin I_1$までわかっている。
        \item
        ($1=f^0$の係数$v_0 \notin I_1$となるようにとりなおす。)
        \item $\exists t$:
        $v_0,\dots,v_{t-1}\in I_1$かつ$v_t \notin I_1$とする。
        \item
        \begin{align}
          f^t \sum_{j=t}^r v_t f^{j-t} \in I
        \end{align}
        を考える。
        $f\notin I$なので(iii。$f$は$(a_1,\dots,a_n)$で消えず、この点は$V-W_0$の元だった。)
        、
        \begin{align}
          \sum_{j=t}^r v_t f^{j-t} \in I.
        \end{align}
        これは定数の係数$v_t$が(Tより)$v_t \notin I_1$となっている。
        \item
        Dおわり。$v_i \in k[x_2,\dots,x_n]$で、
        $\sum_{i=0}^r v_i f^i \in I$かつ$v_0 \notin I_1$となるものが存在する。
        \item
        次を示す:
        \begin{align}
          \pi_1(V) \cap (k^{n-1}-\var(v_0)) \subset \pi_1(V-W_0).
        \end{align}

        実際、
        $\sum_{i=0}^r v_i f^i \in I$なので、
        任意の$(c_1,\dots,c_n) \in V$に対して、
        \begin{align}
          v_0(c_2,\dots,c_n)
          +
          f(c_1,\dots,c_n)\sum_{i=1}^r v_i(c_2,\dots,c_n) f(c_1,\dots,c_n)^{i-1} = 0
        \end{align}
        となる。したがって、$v_0(c_2,\dots,c_n)\neq 0$ならば
        $f(c_1,\dots,c_n)\neq 0$となり($ab=0 \implies a=0 \vee b=0$)、
        (iiiより、$f$は$W_0$上消えるので)$(c_1,\dots,c_n)\notin W_0$である。
        (まず$V$をとって、そこから射影を考え、引き算の条件をならばにして示した。)
        \item $g$:
        $u_r\notin I_1$(Dより。最高次係数はこうしておいた。)であり、
        $v_0 \notin I_1$(Uより。定数の係数はこうしておいた)であり、
        $I_1$は素イデアルなので、$g=u_r v_0$とすると
        $g\notin I_1$である。
        \item
        $W_1$: $W_1 = \var(g)\cap \var(I_1)$とする。
        \item
        Xの$g\notin I_1$より、$W_1 \subsetneq \var(I_1)$である。
        \item[AA]
        示したいのは、(d)の
        \begin{align}
          \var(I_1) - W_1 \subset \pi_1(V-W_0)
        \end{align}
        だった。$(c_2,\dots,c_n)\in \var(I_1)-W_1$をとる。
        $(c_2,\dots,c_n)\notin W_1$なので、
        「$u_r$で消えるか$v_0$消える」の否定で、$u_r,v_0$のどちらでも消えない。
        \item[AB]
        $\exists f_\bullet$: $I=\gen{f_1,\dots,f_s}$とする。
        \item[AC]
        $h\in I$なので(A)、$I=\gen{h,f_1,\dots,f_s}$となる。
        \item[AD] $\exists c_1$:
        拡張定理と、$h$の先頭係数$u_r(c_2,\dots,c_n) \neq 0$であること(AA)、
        より、ある$c_1 \in k$で、$(c_1,\dots,c_n) \in V$となるものが存在する。
        \item[AE]
        Wの式(の左側から自由にとって),
        $v_0(c_2,\dots,c_n) \neq 0$より、$(c_2,\dots,c_n)\in \pi_1(V-W_0)$となり、
        AA、あるいは(d)の式が示される。($\pi(I_1)-W_1$から元をとると、それは自動的に$\var(I_1)-W_1$に入り、AAの式が使える。)
      \end{enumerate}
      \item $l=1$のときは示したので、$l-1$で成立を仮定する。
      \item $\forall W_0$:
      $W_0\subset \neq V$を自由にとる。
      \item $\exists W_1$:
      $l=1$のときは示したので適用する。
      \begin{align}
        W_1 \subsetneq \var(I_1) かつ \var(I_1) - W_1 \subset \pi_1(V-W_0)
      \end{align}
      をみたすものがある。
      \item
      $I=l = (I_1)_{l-1}$である。
      \item
      $\var(I_1)$は既約である。(a,b,c)
      \item $\exists W_l$:
    帰納法の仮定を使う。
    \begin{align}
      W_l \subsetneq \var(I_l) かつ \var(I\1) - W_1 \subset \tilde \pi_{l-1}(\var(I_1)-W_1)
    \end{align}
    なるものが存在する。
    \item
    ここで、$\tilde \pi_{l-1} \colon k^{n-1} \to k^{n-l}$は射影であるが、
    domainが違ったので区別している。$\pi_l = \tilde \pi_{l-1}\circ \pi_1$なので
    \begin{align}
      \var(I_l)- W_l
      \descsubset{xi}
      \tilde \pi_{l-1}(\var(I_1)-W_1)
      \descsubset{viii}
      \pi_l(V-W_0).
    \end{align}
    となる。
    \item $l$全体で示され、既約な多様体について、定理1(の強いやつ)が成立する。
    \end{enumerate}
  \end{enumerate}
  \item
  既約でない(!!)場合に示す。
  \item $V_\bullet$:
  \begin{align}
    V = V_1 \cup \dots \cup V_m
  \end{align}
  と分解する。$V_\bullet$は既約。
  \item
  $V_\bullet'$: $\pi_l(V_\bullet)$のザリスキ閉包。$V_\bullet' = \var(\ideal(\pi_l(V_\bullet)))$。
  \item
  \begin{framed}
    \begin{align}
    \var(I_l) = V_1' \cup \dots \cup V_m'
    \end{align}
  \end{framed}
  を示す。
  \begin{enumerate}
    \item
    $V_1'\cup \dots \cup V_m7$は$\pi_l(V_1)\cup \dots \cup \pi_l(V_m)= \pi_l(V)$を含む多様体である。
    \item
    $\var(I_l)$は$\pi_l(V)$のザリスキ閉包なので(代数的閉体、閉包定理)
    \begin{align}
      \var(I_l) \subset V_1' \cup \dots \cup V_m'.
    \end{align}
    \item 逆を示す。各$i$にたいし、
    \begin{align}
      \pi_l(V_i) \subset \pi_l(V) \subset \var(I_l).
    \end{align}
    \item
    $V_i'$は$\pi_l(V_i)$のザリスキ閉包なので、
    \begin{align}
      V_i' \subset \var(I_l).
    \end{align}
    \item
      \begin{align}
        V_1' \cup \dots V_m' \subset \var(I_l).
      \end{align}
  \end{enumerate}
  \item
  4-4-定理3によれば、$\var(I_l)$は$\pi_l(V)$のザリスキ閉包なので、(代数的閉体だし)
  \begin{align}
    V_i' = (\pi_l(V_i)のザリスキ閉包)
    =
    \var(\ideal(V_i)_l).
  \end{align}
  \item
  $V_i$は既約としておいたので$\ideal(V_i)$は既約で、
  $\ideal(V_i)_l$も既約で、$V_i'$も既約になる。よって、(7)の分解は既約分解である。
  \item
  他のものには含まれない$V_\bullet'$があるはずなので、
  それを番号をつけかえて$V_1'$が他のものに含まれないということにしておく。

  すべての$V_\bullet'$が等しいということはおこらない。なぜなら、
  すべてが等しいとしたら$\var(I_l)$が既約ということになる。
  すると、$I_l$は素イデアルになる。
  5により、$V$は既約でないとしたのだから、$I$は素イデアルでなく、
  $I_l$も素イデアルでない??
  \item $W_1$: $V_1$にいままでの「既約にたいする強い定理」を使って、
  定理の$W_0=\emptyset$とすることで、
  \begin{align}
    \var(\ideal(V_1)_l)- W_1 \subset \pi_l(V_1)
  \end{align}
  となる$W_1 \subsetneq V_1'$が存在する。
  \item
  9と上より、
  \begin{align}
    V_1' - W_1 \subset \pi_l(V_1).
  \end{align}
  \item
  $W$: $W=W_1 \cup V_2' \cup \dots \cup V_m'$とする。(これがみたす!)
  \item
  $W\subset \var(I_l)$となる。($W_1\subsetneq V_1'$だし、$\var(I_l)$の分解がある。)
  \item
  \begin{align}
    \var(I_l) - W
    &=
    (V_1' \cup \dots \cup V_m')  - (W_1 \cup V_2' \cup \dots \cup V_m')\\
    &=
    V_1' -  (W_1 \cup V_2' \cup \dots \cup V_m')\\
    &\subset
    V_1'-W_1 \subset \pi_l(V_1) \subset \pi_l(V).
  \end{align}
  \item
  $W\neq \var(I_l)$を示す。$W=\var(I_l)$とする。
  \begin{enumerate}
    \item
    \begin{align}
      W_1 \cup V_2' \cup \dots \cup V_m' = \var(I_l) = V_1' \cup \dots \cup V_m'.
    \end{align}
    \item
    \begin{align}
      V_1' \subset W_1 \cup V_2' \cup \dots \cup V_m'.
    \end{align}
    \item
    $V_1'$は既約なので、$W_1,\, V_2',\, \dots ,\, V_m'$のどれかに含まれなければならない。矛盾。
  \end{enumerate}

\end{enumerate}
