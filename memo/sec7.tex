\label{sec:有限群の不変式論}

\subsection{対称多項式}
\label{sub:対称多項式}


\begin{framed}
  定理3(対称式の基本定理):
  $k[x_1,\dots,x_n]$の任意の対称多項式は、
  基本対称式$\sigma_1,\dots,\sigma_n$の多項式として一意に表すことができる。
\end{framed}
\begin{myproof}
  \begin{enumerate}
    \item $x_1>x_2>\dots>x_n$という順序を使う。
    \item $\forall f$: $f\in k[x_1,\dots,x_n]$を$f\neq 0$とする。
    \item
    $a,\alpha$: $\LT(f) = ax^\alpha$
    \item
    $\alpha_\bullet$: $\alpha = (\alpha_1,\dots,\alpha_n)$
    \item $\alpha_1\ge \alpha_2 \ge \dots \ge \alpha_n$?
    \begin{enumerate}
      \item
      $\exists i$: $\alpha_i < \alpha_{i+1}$と仮定する。
      \item
      $\beta$: $\beta = (\dots,\alpha_{i+1},\alpha_i,\dots)$
      \item
      3より、$ax^\alpha$は$f$の項。
      \item
      $f$は対称式なので、$ax^\beta$も$f$の項。
      \item
      $\beta > \alpha$なので、上は3のLTであることに矛盾。
      \item
      $\alpha_1\ge \alpha_2 \ge \dots \ge \alpha_n$
    \end{enumerate}
    \item $h$: $h = \sigma_1^{\alpha_1-\alpha_2}\sigma_2^{\alpha_2-\alpha_3}\dots \sigma_{n-1}^{\alpha_{n-1}-\alpha_n}\sigma_n^{\alpha_n}$
    とする。
    \item
    5より、
    \begin{align}
      \LT(h)
      &=
      \LT(\sigma_1^{\alpha_1-\alpha_2}\sigma_2^{\alpha_2-\alpha_3}\dots \sigma_{n-1}^{\alpha_{n-1}-\alpha_n}\sigma_n^{\alpha_n})\\
      &=
      \LT(\sigma_1)^{\alpha_1-\alpha_2}\LT(\sigma_2)^{\alpha_2-\alpha_3} \dots \LT(\sigma_n)^{\alpha_n}\\
      &=
      x_1^{\alpha_1-\alpha_2} (x_1x_2)^{\alpha_2-\alpha_3}\dots (x_1 \dots x_n)^{\alpha_n}\\
      &=
      x_1^{\alpha_1}\dots x^{\alpha_n}.
    \end{align}
    \item
    上より、$\LT(f) = \LT(ah)$となる。
    \item
    $f-ah\neq 0$のときは、$f_1 = f-ah$とする。
    \item $\exists t$:
    5-9までの操作を繰替えすと、
    \begin{align}
      \multideg(f) > \multideg(f_1) > \multideg(f_2) > \dots
    \end{align}
    をみたす列が得られる。これは停止するので、
    $f_{t+1} = 0$となる$t$がある。
    \item
    $f=ah + a_1 h_1 + \dots + a_t h_t$となる。存在は示された。
    \item $g_1,g_2$:
    $f=g_1(\sigma_1,\dots,\sigma_n) = g_2(\sigma_1,\dots,\sigma_n)$とする。
    $g_1,g_2 \in k[y_1,\dots,y_n]$とする。$g_1=g_2$を示したい。
    \item
    $g$: $g=g_1-g_2$
    \item
    $g(\sigma_1,\dots,\sigma_n) = 0$
    \item
    $g=0$を示したい。$g\neq 0$と仮定する(背理法)。
    \item
    $a_\bullet$: $g=\sum_\beta a_\beta y^\beta$とする。
    \item
    $g_\bullet$: $g_\beta = a_\beta \sigma_1^{\beta_1} \dots \sigma_n^{\beta_n}$とする。
    $g_\beta \in k[x_1,\dots,x_n]$になっている。
    \item
    $g(\sigma_1,\dots,\sigma_n)$は$g_\beta$たちの和である。
    $g(\sigma_1,\dots,\sigma_n) = \sum_\beta a_\beta g_\beta$である。
    \item
    計算すると、
    \begin{align}
      \LT(g_\beta) = a_\beta x_1^{\beta_1+\dots+\beta_n} x_2^{\beta_2+\dots+\beta_n} \dots x_n^{\beta_n}
    \end{align}
    \item
    \begin{align}
      (\beta_1,\dots,\beta_n) \mapsto (\beta_1+\dots+\beta_n, \beta_2+\dots+\beta_n,\dots,\beta_n)
    \end{align}
    は単射である(尻尾から決めればいい。)。
    \item
    上と19より、$g_\beta$たちはそれぞれ異なる先頭項を持つ。
    \item
    $\LT(g_\beta)$が最高になるものを選べるが、上よりそのようなものは1つしかない。それを$\beta$にする。
    \item
    $\gamma\neq \beta$なら、$\LT(g_\beta)$は$g_\gamma$のすべての項よりおおきい。
    \begin{align}
      \LT(g_\beta) > \LT(g_\gamma) \ge (\forall g_\gamma の項)
    \end{align}
    \item
    $g(\sigma_1,\dots,\sigma_n)$は$k[x_1,\dots,_n]$で零でない
    \footnote{$g$が$k[y_1,\dots,y_n]$のなかで零であることを示したかった。そのこととは違う。}
    。これは14に矛盾。
  \end{enumerate}
\end{myproof}

\begin{framed}
  命題4: 環$k[x_1,\dots,x_n,y_1,\dots,y_n]$において、
  $x_1,\dots,x_n$のうち1つでも含む単項式は、$k[y_1,\dots,y_n]$
  のすべての単項式より大きくなるような単項式順序を1つ固定する。
  $G$をイデアル
  \begin{align}
    \gen{\sigma_1-y_1,\dots,\sigma_n-y_n}\subset k[x_1,\dots,x_n,y_1,\dots,y_n]
  \end{align}
  のグレブナ基底とする。このとき、次のことが成り立つ。
  \begin{enumerate}[label=(\roman*)]
    \item $f$が対称であることと、$g\in k[y_1,\dots,y_n]$は同値である。
    \item $f$が対称ならば、$f=g(\sigma_1,\dots,\sigma_n)$は、
    $f$の基本対称式$\sigma_1,\dots,\sigma_n$の多項式としての一意的な表示である。
  \end{enumerate}
\end{framed}
\begin{myproof}
  \begin{enumerate}
    \item $g_\bullet$:
    $G=\set{g_1,\dots,g_t}$とする。
    \item
    $f,A_\bullet,g$:
    $f$を$G$で割る。
    \begin{align}
      f = A_1 g_1 + \dots + A_t g_t + g.
    \end{align}
    \item
    $\Leftarrow$を示す。$g\in k[y_1,\dots,y_n]$とする。
    \begin{enumerate}
      \item 仮定の$f\in k[x_1,\dots,x_n]$、$y_\bullet$がないことより、$f(x_1,\dots,x_n,\sigma_1,\dots,\sigma_n) = f$である。
      \item
      $y_\bullet \Leftarrow \sigma_\bullet$という代入操作を行うと、
      $\gen{\sigma_1-y_1,\dots,\sigma_n-y_n}$の元はすべて0になる。
      \item 上のことより$y_\bullet \Leftarrow \sigma_\bullet$によって
      $g_1,\dots,g_t \in \gen{\sigma_1-y_1,\dots,\sigma_n-y_n}$は0になる。
      \item
      2に$y_\bullet \Leftarrow \sigma_\bullet$すると、(a)-(c)より、
      \begin{align}
        f=g(\sigma_1,\dots,\sigma_n)
      \end{align}
      である。
      \item $f$は対称である。
    \end{enumerate}
    \item $\Rightarrow$を示す。$f\in k[x_1,\dots,x_n]$が対称であるとする。
    \begin{enumerate}
      \item $g'$\footnote{本だと字がぶつかっていてやばい。}:
      $f=g'(\sigma_1,\dots,\sigma_n)$となるような$g'\in k[y_1,\dots,y_n]$が存在する。
      \item ($f$を$G$でわったあまりが$g'$?)
      \item
      $\alpha_1,\dots,\alpha_n \in \Zge$とすると、
      $B_1,\dots,B_n \in k[x_1,\dots,x_n,y_1,\dots,y_n]$を用いて、
      \begin{align}
        \sigma_1^{\alpha_1}\dots \sigma_n^{\alpha_n}
        &=
        (y_1+(\sigma_1-y_1))^{\alpha_1} \dots (y_n + (\sigma_n-y_n))^{\alpha_n}\\
        &=
        y_1^{\alpha_1}\dots y_n^{\alpha_n} + B_1\cdot (\sigma_1-y_1)  + \dots + B_n\cdot (\sigma_n-y_n).
      \end{align}
      とかける。
      \item
      上より、$g'$の$y_\bullet$たちでできた単項式について上を適用し足し合わせて、
      \begin{align}
        g'(\sigma_1,\dots,\sigma_n) = g'(y_1,\dots,y_n) + C_1\cdot (\sigma_1-y_1) + \dots + C_n\cdot (\sigma_n-y_n).
      \end{align}
      となる$C_1,\dots,C_n \in k[x_1,\dots,x_n,y_1,\dots,y_n]$である。
      \item
      (a)と上より、
      \begin{align}
        f = C_1 \cdot (\sigma_1-y_1) + \dots + C_n\cdot (\sigma_n-y_n) + g'(y_1,\dots,y_n).
      \end{align}
      \item
      ($g'$は$f$を$G$でわった余り?)
      \item
      $g'$のどの項も、$\LT(G)$の項でも割りきれない?
      \begin{enumerate}
        \item $g'$のある項が$\LT(G)$のある項で割り切れるとする。
        \item $\exists i$: $\LT(g_i)$が$g'$を割り切るような$g_i \in G$がある。
        \item $g'\in k[y_1,\dots,y_n]$より、$\LT(g_i)$は$y_1,\dots,y_n$だけを含む。
        \item 上と、順序付の仮定\footnote{$x_\bullet$を含んだら$y_\bullet$だけの単項式より大きい}より
        $g_i \in k[y_1,\dots,y_n]$となる。
        \item
        $g_i \in \gen{\sigma_1-y_1,\dots,\sigma_n-y_n}$なので、
        $g_i(\sigma_1,\dots,\sigma_n) = 0$となる。
        \item
        上より、$g_i$は$k[x_1,\dots,x_n]$として対称多項式である。
        \item
        上と定理3、それにvより、$g_i \in k[y_1,\dots,y_n]$は$k[y_1,\dots,y_n]$の元として0である。
        \item
        上は、$g_i$がグレブナ基底の一個であり、非零であることに矛盾する。

      \end{enumerate}
      $g'$のどの項も、$\LT(G)$のどの項を使っても割り切ることはできない。
      \item
      (e),(g)と、$G$がグレブナ基底であることより、$f$を$G$で割ったあまりは$g'$である。
      \item
        上より、$g=g' \in k[y_1,\dots,y_n]$となり、$g\in k[y_1,\dots,y_n]$である。

    \end{enumerate}
  \end{enumerate}
  後半の(ii)は、$f=g(\sigma_1,\dots,\sigma_n)$となっていることは上
  の考察から従う。それが一意であることは定理3から従う。
\end{myproof}

\begin{framed}
  命題5: $k[x_1,\dots,x_n,y_1,\dots,y_n]$上の
  $x_1>\dots>x_n>y_1>\dots y_n$で決まるlex順序を固定する。
  このとき、$k=1,\dots,n$に対して、多項式
  \begin{align}
    g_k = h_k(x_k,\dots,x_n) + \sum_{i=1}^k (-1)^i h_{k-i}(x_k,\dots,x_n)y_i
  \end{align}
  は、イデアル$\gen{\sigma_1-y_1,\dots,\sigma_n-y_n}$のグレブナ基底をなす。
\end{framed}
\begin{myproof}
  演習問題10をとく。$h_k$は、次数$k$の単項式すべての和である。
  $x^\alpha$は$k$次の単項式であり、$x^\alpha$にあらわれる変数の個数を$a$とする。
  \begin{enumerate}[label=(\alph*)]
    \item

      「$x^\alpha$が$h_{k-i}\sigma_i$のなかに現れるならば、
      $i\le a$を示せ。」


    $x^\alpha$も$h_{k-i}\sigma_i$のすべての項も次数$k$なので次数の心配はいらない。
    仮に$i>a$とする。$\sigma_i$にはちょうど$i$個の変数があらわれるので、
    $h_{k-i}\sigma_i$のすべての項には$i$個以上の変数があらわれ、
    つまり$a$よりも真に大きい個数の変数があらわれる。このとき、
    $x^\alpha$の変数の個数は$a$なのだから、$h_{k-i}\sigma_i$の項たちにあらわれることができない。
    対偶が示された。
    \item
    \begin{framed}
      $i\le a$ならば、$\sigma_i$のなかのちょうど$\tatev{a\\i}$個の項が、
      $x^\alpha$にあらわれる変数だけを含んでいる。
    \end{framed}
    あきらか。
    \item
    \begin{framed}
      $i\le a$ならば、$x^\alpha$は係数$\tatev{a\\i}$を持つ
      $h_{k-i}\sigma_i$の項であることを示せ。
    \end{framed}
    $\sigma_i$のなかから$x^\alpha$に含まれている変数だけを持っているものを選び、
    それに対して適当な$h_{k-i}$の項を選んでかければ(これは$h_{k-i}$の定義より可能である。)多重次数は$\alpha$に一致する。
    また、$x^\alpha$に含まれていない変数を選んでものそのようなことはできない。よって、
    $x^\alpha$の$h_{k-i}\sigma_i$での係数は、$\sigma_i$での$x^\alpha$
    に含まれる係数だけを持つもの全体の個数と一致する。よって、それは上の問題より$\tatev{a\\i}$である。
    \item
    \begin{framed}
      $\sum_{i=0}^k (-1)^i h_{k-i}\sigma_i$\footnote{たぶん誤植}における$x^\alpha$の係数は$\sum_{i=0}^\alpha (-1)^i \tatev{a \\ i}$
      であることを結論せよ。それから2項定理を使って$x^\alpha$の係数が0であることを示せ。
    \end{framed}
    係数は上よりあきらか。係数も、これは$(1-1)^a$なので簡単。
    \item
    以上で、
    \begin{align}
      0=\sum_{i=0}^k (-1)^i h_{k-i}h_i(x_1,\dots,x_n) \sigma_i(x_1,\dots,x_n).
    \end{align}
  \end{enumerate}
  次に、問題11をとく。$S\subset \set{1,\dots,k-1}$のとき、$x^S$
  で変数の積をあらわす。
  \begin{enumerate}[label=(\alph*)]
    \item
    「
      \begin{align}
        \sigma_i(x_1,\dots,x_n) =
        \sum_{S\subset \set{1,\dots,k-1}} x^S \sigma_{i-\myabs{S}} (x_k,\dots,x_n)
      \end{align}
      ここで、$j<0$のとき$\sigma_j = 0$。
      」
    左と右の項を考えれば。
    \item
    \begin{framed}
      \begin{align}
        \sum_{i=0}^k (-1)^i h_{k-i}(x_k,\dots,x_n)\sigma_i(x_1,\dots,x_n)
        =
        \sum_{S\subset \set{1,\dots,k-1}}x^S (\sum_{i=\myabs{S}}^k (-1)^i h_{k-i}(x_k,\dots,x_n)\sigma_{i-\myabs{S}} (x_k,\dots,x_n)).
      \end{align}
    \end{framed}
    (a)の式に$(-1)^i h_{k-i}$をかけて$\sum_{i=0}^k$をとる。$\sigma_{負の数}=0$に注意して、
    \begin{align}
      \sum_{i=0}^k (-1)^i h_{k-i}(x_k,\dots,x_n)\sigma_i(x_1,\dots,x_n)
      &=
      \sum_{S\subset \set{1,\dots,k-1}}x^S (\sum_{i=0}^k (-1)^i h_{k-i}(x_k,\dots,x_n)\sigma_{i-\myabs{S}} (x_k,\dots,x_n))\\
      &=
      \sum_{S\subset \set{1,\dots,k-1}}x^S (\sum_{i=\myabs{S}}^k (-1)^i h_{k-i}(x_k,\dots,x_n)\sigma_{i-\myabs{S}} (x_k,\dots,x_n)).
    \end{align}
    \item
    \begin{framed}
      \begin{align}
        \sum_{i=\myabs{S}}^k (-1)^i h_{k-i}(x_k,\dots,x_n)\sigma_{i-\myabs{S}} (x_k,\dots,x_n)
        =
        0
      \end{align}
    \end{framed}
    \begin{align}
      \sum_{i=\myabs{S}}^k (-1)^i h_{k-i}(x_k,\dots,x_n)\sigma_{i-\myabs{S}} (x_k,\dots,x_n)
      &=
      \sum_{j=0}^{k-\myabs{S}} (-1)^{j+\myabs{S}} h_{k-j-\myabs{S}}(x_k,\dots,x_n)\sigma_{j} (x_k,\dots,x_n)\\
      &=
      (-1)^{\myabs{S}}\sum_{j=0}^{k-\myabs{S}} (-1)^{j} h_{(k-\myabs{S})-j}(x_k,\dots,x_n)\sigma_{j} (x_k,\dots,x_n)\\
      &\desceq{問題10}
      0.
    \end{align}
  \end{enumerate}
  次に演習12をとく。
  \begin{align}
    g_k = h_k(x_k,\dots,x_n) + \sum_{i=1}^k (-1)^i h_{k-i}(x_k,\dots,x_n)y_i
  \end{align}
  としてある。
  \begin{align}
    g_k = (-1)^k (y_k-\sigma_k) +
    \sum_{i=1}^{k-1}(-1)^i h_{k-i}(x_k,\dots,x_n)(y_i-\sigma_i)
  \end{align}
  は既知。
  \begin{enumerate}[label=(\alph*)]
    \item
    「
      \begin{align}
        \gen{\sigma_1-y_1,\dots,\sigma_n-y_n} \subset \gen{g_1,\dots,g_n}
      \end{align}
      」
    $\sigma_1 - y_1= g_1$なので、$\sigma_1 - y_1 \in (右)$となる。
    $(-1)^2\sigma_2 -y_2 = g_2 - g_1 \in (右)$となる。
    以降おなじ。
    \item
    $\LT(g_k) = x_k^k$であること。定義の式からあきらか$y_i$を含まないほうしか見るものがない。
    \item
    $g_1,\dots,g_k$がグレブナ基底?
    (b)より、$i\neq j$のとき、$\LT(g_i)$と$\LT(g_j)$
    は互いに素になっている。よって、命題9-4より、
    $S(g_i,g_j) \toG 0$になる。よって、命題9-3より、
    $\set{g_1,\dots,g_n}$はグレブナ基底になっている。
  \end{enumerate}
  証明する。
  \begin{enumerate}
    \item 演習10と11より、
    \begin{align}
      0=h_k(x_k,\dots,x_n) + \sum_{i=1}^k (-1)^i h_{k-i}(x_k,\dots,x_n)\sigma_i.
    \end{align}
    \item
    $g_1,\dots,g_n$は$\gen{\sigma_1-y_1,\dots,\sigma_n-y_n}$の基底?
    \begin{enumerate}
      \item
      $g_k$の定義
      \begin{align}
        g_k = h_k(x_k,\dots,x_n) + \sum_{i=1}^k (-1)^i h_{k-i}(x_k,\dots,x_n)y_i
      \end{align}
      から1の式を引いて、
      \begin{align}
        g_k = \sum_{i=1}^k (-1)^i h_{k-i}(x_k,\dots,x_n)(y_i-\sigma_i).
      \end{align}
      \item
      よって、$\gen{g_1,\dots,g_n}\subset \gen{\sigma_1-y_1,\dots,\sigma_n-y_n}$
      \item
      (a)から、
      \begin{align}
        g_k = (-1)^k (y_k-\sigma_k) +
        \sum_{i=1}^{k-1} (-1)^i h_{k-i}(x_k,\dots,x_n)(y_i-\sigma_i).
      \end{align}
      \item
      上と演習12より、$\gen{\sigma_1-y_1,\dots,\sigma_n-y_n} \subset \gen{g_1,\dots,g_n}$。
      \item
      (b)(d)より、$\gen{\sigma_1-y_1,\dots,\sigma_n-y_n} = \gen{g_1,\dots,g_n}$となる。
    \end{enumerate}
    \item
    演習問題12で$\LT(g_k)=x_k^k$を示して、
    さらにグレブナ基底であることを示す。おわり。
  \end{enumerate}
\end{myproof}

\begin{framed}
  命題7:多項式$f\in k[x_1,\dots,x_n]$が対称であることと、
  $f$のすべての斉次成分が対称であることは同値である。
\end{framed}
\begin{myproof}
  $\Rightarrow$を示せばよい。$f$が対称であるとする。
  \begin{enumerate}
    \item
    $\forall i_1,\dots,i_n$:
    $x_{i_1},\dots,x_{i_n}$を$x_1,\dots,x_n$の置換とする。
    \item
    置換しても、次数はかわらない。
    \item
    $f(x_{i_1},\dots,x_{i_n}) = f(x_1,\dots,x_n)$
    \item
    上2つより、全次数が$k$の斉次も対称。
  \end{enumerate}
\end{myproof}

\begin{framed}
  定理8:
  $k$が有理数体$\Q$を含む体ならば、$k[x_1,\dots,x_n]$の任意の対称多項式は
  ベキ和$s_1,\dots,s_n$の多項式として表せる。
\end{framed}
\begin{myproof}
  演習14をやる。ニュートン恒等式は
  \begin{align}
    s_k - \sigma_1 s_{k-1} + \dots + (-1)^{k-1}\sigma_{k-1}s_1 + (-1)^k k\sigma_k &= 0 \quad (1\le k \le n),\\
    s_k - \sigma_1 s_{k-1} + \dots + (-1)^{n-1}\sigma_{n-1}s_{k-n+1}+(-1)^n \sigma_n s_{k-n}  &= 0 \quad  (k>n)
  \end{align}
  である。
  \begin{enumerate}
    \item
    「$\sigma_0 = 1$と$i<0,i>n$のときに$\sigma_i=0$としておく。このとき、
    \begin{align}
      \Forall{k\ge 1} s_k-\sigma_1 s_{k-1} + \dots + (-1)^{k-1} \sigma_{k-1}s_1 + (-1)^k k\sigma_k = 0
    \end{align}
    と同値?」
    $k\le n$と$k>n$とで分ける。
    \item
    「上の恒等式を変数の数$n$に関する帰納法で示せ。ただし、
    $n$変数の$\sigma_i$を$\sigma_i^n$、$s_k$を$s_k^n$とする。」
    $n=1$のとき:
    $1\le k \le n$のとき、すなわち$k=1$のときを考える。
    \begin{align}
      \ub{s_k^1 - \sigma_1^1 s_{k-1}^1  \dots + (-1)^{k-1}\sigma_{k-1}^1 s_1^1}_{kコ} + (-1)^k k \sigma_k^1
      =
      s_1^1 + (-1)^1\cdot 1 \cdot \sigma_1^1
      =
      x_1 - x_1 = 0.
    \end{align}
    $k>n$のとき、すなわち$k>1$のときを考える。このときは、
    $\sigma_0,\sigma_1$だけが非零になる。
    \begin{align}
      \ub{s_k^1 - \sigma_1^1 s_{k-1}^1  \dots + (-1)^{k-1}\sigma_{k-1}^1 s_1^1}_{kコ} + (-1)^k k \sigma_k^1
      &=
      s_1^1 + \sigma_1^1 s_0^1 + (-1)^1 \cdot 1\cdot \sigma_1\\
      &=
      x_1 + x_1\cdot 0 - x_1\\
      & = 0.
    \end{align}

    $n-1$変数でうまく行っているとする。???


  \end{enumerate}
\end{myproof}


\subsection{有限行列群と不変式環}
\label{sub:有限行列群と不変式環}
$\Q \subset k$とする。

\begin{framed}
  定義1: 体$k$の元を成分に持つ可逆な$n\times n$行列全体の集合を
  $GL(n,k)$であらわす。
\end{framed}

\begin{framed}
  定義2:
  有限部分集合$G\subset GL(n,k)$が有限行列群であるとは、
  空でなく、行列のかけ算で閉じていることをいう。
  $G$の元の個数を、$G$の位数とよび、$\myabs{G}$であらわす。
\end{framed}

\begin{framed}
  $G\subset GL(n,k)$を有限行列群とする。
  \begin{enumerate}[label=(\roman*)]
    \item $I_n \in G$。
    \item $A\in G$ならば、ある正の整数$m$があって、$A^m = I_n$となる。
    \item $A\in G$ならば、$A\inv \in G$である。
  \end{enumerate}
\end{framed}
\begin{myproof}
  \begin{itemize}
    \item (ii):
    \begin{enumerate}
      \item $A\in G$とする。
      \item $G$が積で閉じているので、$\set{A,A^2,A^3,\dots}\subset G$である。
      \item $i,j$:
      $G$は有限なので、$A^i = A^j$となる$i,j\in \N$がある。$i > j$とする。
      \item
      $m= i-j$とする。
      \item  3より、
      $A^m = A^{i-j} = A^i A^{-j} = E$となる。$m$が条件をみたしたことになる。
    \end{enumerate}
    \item (iii):
    \begin{enumerate}
      \item $I_n = A^{m-1}\cdot A$となる。$m$は上のもの。
      \item $G$は積で閉じているので、$A^{m-1} \in G$となる。
      \item $A^{-1} = A^{m-1} \in G$となる。
    \end{enumerate}
    \item (i):
    $I_n = A^m \in G$となる。
  \end{itemize}
\end{myproof}

\begin{framed}
  定義7:
  $G\subset GL(n,k)$を有限行列群とする。
  多項式$f(\bbold x) \in k[x_1,\dots,x_n]$が、
  すべての$A\in G$に対して、$f(\bbold x) = f(A\cdot \bbold x)$
  をみたすとき、$G$で不変であるという。$G$で不変な多項式全体
  の集合を$k[x_1,\dots,x_n]^G$であらわす。
\end{framed}

\begin{framed}
  例8:
  \begin{align}
    k[x_1,\dots,x_n]^{S_n} =
    \set{k[x_1,\dots,x_n]内のすべての対称多項式}
  \end{align}
\end{framed}

\begin{framed}
  命題9:
  $G\subset GL(n,k)$を有限行列群をする。このとき、
  集合$k[x_1,\dots,x_n]^G$は和と積で閉じており、
  すべての定数多項式を含む。
\end{framed}
\begin{myproof}
  演習10。
  \begin{itemize}
    \item 和:$f(\bbold x),g(\bbold x)\in k[x_1,\dots,x_n]^G$とする。
    \begin{align}
      (f+g)(A\bbold x)
      =
      f(A\bbold x) + g(A\bbold x)
      =
      f(\bbold x) + g(\bbold x)
      =
      (f+g)(\bbold x).
    \end{align}
    \item 積:$f,g$は同様。
    \begin{align}
      (fg)(A\bbold x)
      =
      f(A\bbold x)g(A\bbold x)
      =
      f(\bbold x)g(\bbold x)
      =
      (fg)(\bbold x).
    \end{align}
    \item
    定数を含む:$c\in k$とする。
    \begin{align}
      c(A\bbold x) = c = c(\bbold x).
    \end{align}
    $c\in k[x_1,\dots,x_n]^G$である。
  \end{itemize}
\end{myproof}

\begin{framed}
  命題10:
  $G\subset GL(n,k)$を有限行列群とする。このとき、
  多項式$f\in k[x_1,\dots,x_n]$が$G$で不変であることと、
  その斉次成分がすべて$G$で不変であることとは同値である。
\end{framed}
\begin{myproof}
  $\bbold x\mapsto A\bbold x$は次数を変えないので、
  $A$によって単項式はその次数を変えない。
  よって、$f(\bbold x)$の次数$N$のものは
  $f(A\bbold x)$の次数$N$のものに移ることになる。

  $F\colon \set{f(\bbold x)の項} \to \set{f(A\bbold x)の項}$
  $f$が不変なので、$F$は可逆写像になっている。   $N \in \Zge$とする。
  $F\seigen{\set{次数Nの項}}\colon \set{f(\bbold x)のN次の項} \to \set{f(A\bbold x)の項}$
  だが、先の考察より$\bbold x\mapsto A\bbold x$は次数を変えないので、
  $F\seigen{\set{次数Nの項}}\colon \set{f(\bbold x)のN次の項} \to \set{f(A\bbold x)のN次の項}$
  になっている。$F$が単射だったので、$F\seigen{\set{次数Nの項}}$も単射になっている。
  よって、$\#\set{f(\bbold x)のN次の項} \le \#\set{f(A\bbold x)のN次の項}$となる。
  さらに、$F$が有限集合同士の可逆写像なので、
  \begin{align}
    \# \set{f(\bbold x)の項}
    =
    \# \set{f(A\bbold x)の項}
    =
    \sum_N \#\set{f(A\bbold x)のN次の項}
    \ge
    \sum_N \#\set{f(\bbold x)のN次の項}
    =
    \# \set{f(\bbold x)の項}
  \end{align}
  なので、各$N$について、$\#\set{f(\bbold x)のN次の項} = \#\set{f(A\bbold x)のN次の項}$
  となり、$F\seigen{\set{次数Nの項}}$は同型になる。これは、
  斉次成分が$G$で不変であることを意味する。
\end{myproof}

\begin{framed}
  補題11:
  $G\in GL(n,k)$を有限行列群とし、
  $A_1,\dots,A_m \in G$が存在して、任意の
  $A\in G$を次の形で表すことができる。
  \begin{align}
    A = B_1 B_2 \dots B_t.
  \end{align}
  ここで、各$i$に対して$B_i \in \set{A_1,\dots,A_m}$である。
  (このとき$A_1,\dots,A_m$は群$G$を生成するという。)このとき、
  $f\in k[x_1,\dots,x_n]$が$k[x_1,\dots,x_n]^G$の元であることと、
  \begin{align}
    f(\bbold x) = f(A_1 \bbold x) = \dots = f(A_m \bbold x)
  \end{align}
  が成り立つことは同値である。
\end{framed}
\begin{myproof}
  \begin{enumerate}
    \item $f$が行列$B_1,\dots,B_t$すべての作用で不変であるとする。
    このとき積$B_1\dots B_t$でも$f$は不変?
    \begin{enumerate}
      \item
      $t=1$のときはあきらか。$t-1$のとき成立すると仮定する。$t$で示す。
      \item
      \begin{align}
        f((B_1\dots B_t)\bbold x)
        &=
        f((B_1\dots B_{t-1})\cdot B_t \cdot \bbold x)\\
        &=
        f(B_t \cdot \bbold x) \quad (帰納法の仮定)\\
        &=
        f(\bbold x).
      \end{align}
    \end{enumerate}
    \item
    $\Leftarrow$を示す。$f$は$A_1,\dots,A_m$で不変であるとする。
    \begin{enumerate}
      \item
      $\forall A$: $A\in G$とする。
      \item
      $\exists t,B_\bullet$: 仮定より、$A=B_1 \dots B_t$となる$B_\bullet \in \set{A_1,\dots,A_m}$が存在する。
      \item
      1より、$f$は$A$で不変である。
    \end{enumerate}
    \item
    $\Rightarrow$はあきらか。
  \end{enumerate}
\end{myproof}


\subsection{不変式環の生成元}
\label{sub:不変式環の生成元}
\begin{framed}
  定義1:
  $f_1,\dots,f_m \in k[x_1,\dots,x_n]$に対して、$f_1,\dots,f_m$の
  $k$係数の多項式全体で表される元全体からなる$k[x_1,\dots,x_n]$の部分集合を
  $k[f_1,\dots,f_m]$で表す。

  $\gen{f_1,\dots,f_m}$とは違う。
\end{framed}

\begin{framed}
  定義2:
  有限行列群$G\subset GL(n,k)$に対し、次のように定義される写像
  $R_G\colon k[x_1,\dots,x_n]\to k[x_1,\dots,x_n]$
  を$G$のレイノルズ作用素という。すなわち、$f(\bbold x) \in k[x_1,\dots,x_n]$に対し、
  \begin{align}
    R_G(f)(\bbold x) = \frac{1}{\myabs{G}}\sum_{A\in G} f(A \bbold x).
  \end{align}

\end{framed}

\begin{framed}
  命題3:
  有限行列群$G$のレイノルズ作用素$R_G$に対し、次が成り立つ。
  \begin{enumerate}[label=(\roman*)]
    \item $R_G$は$k$線型写像である。
    \item $f\in k[x_1,\dots,x_n]$ならば$R_G(f)\in k[x_1,\dots,x_n]^G$。
    \item $f\in k[x_1,\dots,x_n]^G$ならば$R_G(f) = f$。
  \end{enumerate}
\end{framed}
\begin{myproof}
  (i)を示す。
  \begin{align}
    R_G(af+bg)(\bbold x)
    &=
    \frac{1}{\myabs{G}}\sum_{A\in G}(af+bg)(A\bbold x)\\
    &=
    \frac{a}{\myabs{G}}\sum_{A\in G}(f)(A\bbold x)
    +
    \frac{b}{\myabs{G}}\sum_{A\in G}(g)(A\bbold x)
    \\
    &=
    a R_G(f)(\bbold x) + b R_G(g)(\bbold x)\\
    &=
    (aR_G(f) + bR_G(f))(\bbold x).
  \end{align}

  (ii)を示す。
  \begin{enumerate}[label]
    \item
    $\forall B$: $B\in G$
    \item
    \begin{align}
      R_G(f)(B\bbold x)
      &=
      \frac{1}{\myabs{G}}\sum_{A\in G}f(A\cdot B\bbold x)\\
      &=
      \frac{1}{\myabs{G}}\sum_{A\in G}(AB\cdot \bbold x).
    \end{align}
    \item
    $\exists A_\bullet$: $G=\set{A_1,\dots,A_{\myabs{G}}}$とする。重複のないようにしておく。
    \item
    $i\neq j$のとき、$A_i B \neq A_j B$になる。
    \item
    上より、$\set{A_1 B,\dots,A_{\myabs{G}}B}$はそれぞれ異なる$\myabs{G}$個の元である。
    \item
    また、$\set{A_1 B,\dots,A_{\myabs{G}}B}$は1の$B\in G$より、$\subset G$である。
    \item
    3,5,6より、
    \begin{align}
      G=\set{A_1,\dots,A_{\myabs{G}}}
      =
      \set{A_1 B,\dots,A_{\myabs{G}} B}
      =
      \set{AB; A\in G}.
    \end{align}
    \item
    \begin{align}
      \frac{1}{\myabs{G}}\sum_{A\in G}f(AB\cdot \bbold x)
      \desceq{7}
      \frac{1}{\myabs{G}}\sum_{A\in G}f(A\cdot \bbold x)
      =
      R_G(f)(\bbold x).
    \end{align}
    \item
    1おわり:
    \begin{align}
      \Forall{B\in G} R_G(f)(B\cdot \bbold x) = R_G(f)(\bbold x).
    \end{align}
    \item
    上より、$R_G(f) \in k[x_1,\dots,x_n]^G$となる。
  \end{enumerate}

  (iii)を示す。
  $f\in k[x_1,\dots,x_n]^G$とする。$f$は不変式なので、
  \begin{align}
    R_G(f)(\bbold x)
    =
    \frac{1}{\myabs{G}}\sum_{A\in G}f(A\bbold x)
    =
    \frac{1}{\myabs{G}}\sum_{A\in G}f(\bbold x)
    =
    f(\bbold x).
  \end{align}
\end{myproof}

\begin{framed}
  定理5:
  有限行列群$G\subset GL(n,k)$に対し、
  \begin{align}
    k[x_1,\dots,x_n]^G  = k[R_G(x^\beta); \myabs{\beta}\le \myabs{G}]
  \end{align}
  が成り立つ。
  特に、$k[x_1,\dots,x_n]^G$は有限個の斉次不変式で生成される。
\end{framed}
\begin{myproof}
  $\subset$を示す。
  \begin{enumerate}
    \item $\forall f$: $f=\sum_{\alpha} c_\alpha x^\alpha \in k[x_1,\dots,x_n]^G$とする。
    \item
    命題3より、
    \begin{align}
      f=R_G(f)
      =
      R_G(\sum_{\alpha}c_\alpha x^\alpha)
      =
      \sum_{\alpha}c_\alpha R_G(x^\alpha).
    \end{align}
    \item
    1おわり:
     すべての不変式は$R_G(x^\alpha)$の$k$上の線形結合である。
    \item
    すべての$\alpha$について、$R_G(x^\alpha)$が$\myabs{\beta} \le \myabs{G}$
    をみたす$R_G(x^\beta)$に関する多項式?
     \begin{enumerate}
       \item $\forall k$:
       $k\in \Zge$とする。
       \item
       $a_\bulllet$:
       \begin{align}
         (x_1+\dots+x_n)^k = \sum_{\myabs{\alpha}=k}a_\alpha x^\alpha
       \end{align}
       \item
       $a_\alpha$が正整数であることを示す。
       演習4。$\alpha=(\alpha_1,\dots,\alpha_n)\in \Zge^n$とし、
       $\myabs{\alpha} = k$とする。
       \begin{align}
         \tatev{k\\ \alpha} = \frac{k!}{\alpha_1! \dots \alpha_n!}.
       \end{align}
       \begin{enumerate}
         \item
         「$\tatev{k\\ \alpha}$は正整数?」2項係数が整数になることは既知とする\footnote{パスカルの三角形の漸化式で多分行ける。}。
         $n=2$のときは成立している。$n$のとき成立していると仮定する。
         \begin{align}
           \tatev{k \\ (\alpha_1,\dots,\alpha_{n+1})}
           &=
           \frac{k!}{\alpha_1 \dots \alpha_{n+1}}\\
           &=
           \frac{(\alpha_1+\dots+\alpha_n)!}{\alpha_1!\dots\alpha_n!} \cdot \frac{k\cdot \dots \cdot (k-(\alpha_1+\dots+\alpha_n)+1)}{\alpha_{n+1}!}\\
           &=
           \tatev{\alpha_1+\dots+\alpha_n \\ (\alpha_1,\dots,\alpha_n)} \cdot \frac{(\alpha_{n+1}+(\alpha_n+\dots+\alpha_1))\cdot \dots \cdot (\alpha_{n+1}+1)}{\alpha_{n+1}!}\\
           &=
           \tatev{\alpha_1+\dots+\alpha_n \\ (\alpha_1,\dots,\alpha_n)} \cdot \frac{(\alpha_{n+1}+(\alpha_n+\dots+\alpha_1))!}{\alpha_{n+1}! (\alpha_n + \dots + \alpha_1)!}\\
           &=
           \tatev{\alpha_1+\dots+\alpha_n \\ (\alpha_1,\dots,\alpha_n)} \cdot \tatev{\alpha_{n+1} + \dots + \alpha_1 \\ (\alpha_n+\dots+\alpha_1, \alpha_{n+1})}.
         \end{align}
         \item
         「
         \begin{align}
           (x_1+\dots+x_n)^k = \sum_{\myabs{\alpha}=k}\tatev{k\\\alpha}x^\alpha.
         \end{align}」あきらか。
       \end{enumerate}
       \item
       記号を整備する。
       \begin{align}
         (A\bbold x)^\alpha
         =
         (A_1\bbold x)^{\alpha_1} \cdot (A_n \bbold x)^{\alpha_n}
       \end{align}
       と$\square^{\alpha}\colon k^n \to k$を定める。
       \item
       \begin{align}
         R_G(x^\alpha)
         =
         \frac{1}{\myabs{G}}\sum_{A\in G}(A\bbold x)^\alpha .
       \end{align}
       \item $u_1,\dots,u_n$:
       不定元$u_1,\dots,u_n$を用意して、(b)に$x_1 \Leftarrow u_i A_i \bbold x$を代入すると、
       \begin{align}
         (u_1 A_1 \bbold x + \dots + u_n A_n \bbold x)^k = \sum_{\myabs{\alpha}=k}a_\alpha (A\bbold x)^\alpha u^\alpha.
       \end{align}
       \item $b_\bullet$:
       上で$A\in G$にわたる和をとり$S_k$とする。
       \begin{align}
         S_k
         &=
         \sum_{A\in G}(u_1 A_1 \bbold x + \dost + u_n A_n \bbold x)^k\\
         &=
         \sum_{\myabs{\alpha}=k} a_\alpha (\sum_{A\in G}(A\bbold x)^\alpha)u^\alpha\\
         &=
         \sum_{\myabs{\alpha}=k}\ub{b_\alpha}_{\exists} R_G(x^\alpha)u^\alpha.
       \end{align}
       ここで、$b_\alpha = \myabs{G}a_\alpha$とした。
       \item
       $U_\bullet$: $A\in G$をインデックスとして、
       \begin{align}
         U_A = u_1 A_1 \bbold x + \dots + u_n A_n \bbold x
       \end{align}
       とする。
       \item $S_k(\square)$:
       $S_k = S_k(U_A: A\in G) = \sum_{A\in G} U_A^k$。
       $S_k$は$U_1,\dots,U_A$の「$k$乗のベキ和」になっている。
       \item
       上と定理1-7-8\footnote{対称式はベキ和で表せる}より、
       $\set{U_A; A\in G}$の対称式は$S_1,\dots,S_{\myabs{G}}$の
       の多項式である。
       \item $\exists F$:
       $S_k$は$\set{U_A; A\in G}$の対称式なので、
       上より
       \begin{align}
         S_k = F(S_1,\dots,S_{\myabs{G}})
       \end{align}
       となる$k$係数$n$変数多項式$F$が存在する。
       \warn{なお、これは$k>\myabs{G}$でもよい!!1}
       \item
       上(k)に(g)を代入$S_k \Leftarrow \sum_{\myabs{\alpha}=k} b_\alpha R_G(x^\alpha)u^\alpha$する。
        \begin{align}
          \sum_{\myabs{\alpha}=k} b_\alpha R_G(x^\alpha)u^\alpha
          =
          F(\sum_{\myabs{\beta}=1}b_\beta R_G(x^\beta)u^\beta, \dots, \sum_{\myabs{\beta}=\myabs{G}}b_\beta R_G(x^\beta)u^\beta)
        \end{align}
      \item $\forall \alpha$: $\myabs{\alpha} = k$とする。
      \item
      (l)の両辺の多重次数$\alpha$の項を取り出して係数比較すると、
      \begin{align}
        b_\alpha R_G(x^\alpha) =
        (\myabs{\beta}\le \myabs{G}となる \beta についての R_G(x^\beta) の多項式).
      \end{align}
      \item
      (g)で$b_\alpha = \myabs{G}a_\alpha$と、4の$a_\alpha > 0$と体$k$の標数が0であることより、
      $b_\alpha \neq 0$である。
      \item (n)(o)より、
      \begin{align}
        R_G(x^\alpha) =
        (\myabs{\beta}\le \myabs{G}となる \beta についての R_G(x^\beta) の多項式).
      \end{align}


     \end{enumerate}
     よって、すべての$\alpha$について、$R_G(x^\alpha)$が$\myabs{\beta} \le \myabs{G}$
     をみたす$R_G(x^\beta)$に関する多項式。
  \end{enumerate}
\end{myproof}

よって、全次数が$\myabs{G}$以下である全ての単項式についてレイノルズ作用素を計算すれば
$G$の不変式環の生成元全体を求めることができる。

\begin{framed}
  多項式$f_1,\dots,f_m \in k[x_1,\dots,x_n]$が与えらえれたとする。
  ここで、$k[x_1,\dots,x_n,y_1,\dots,y_m]$の単項式順序を、
  変数$x_1,\dots,x_n$のうち1つでも含む多項式は$k[y_1,\dots,y_m]$
  のすべての単項式より大きくなるように定める。
  イデアル$\gen{f_1-y_1,\dots,f_m-y_m} \subset k[x_1,\dots,x_n,y_1,\dots,y_m]$
  のグレブナ基底を$G$とする。与えられた$f\in k[x_1,\dots,x_n]$に対し、
  $g=\ovd{f}{G}$を$f$の$G$による割り算の余りとする。このとき次が成り立つ。
  \begin{enumerate}[label=(\roman*)]
    \item $f\in k[f_1,\dots,f_m]$と$g\in k[y_1,\dots,y_m]$は同値。
    \item $f\in k[f_1,\dots,f_m]$ならば、$f=g(f_1,\dots,f_m)$となり、
    これは$f$の$f_1,\dots,f_m$の多項式としての表示を与える。
  \end{enumerate}
\end{framed}
\begin{myproof}
  (i)を示す。
  \begin{enumerate}
    \item $G$: $G=\set{g_1,\dots,g_t}$とし、重複、0はないものとする。
    \item
    $A_\bullet$:
    $f$を$G$で割って、
    \begin{align}
      f = A_1 g_1 + \dots + A_t g_t + g.
    \end{align}
    $A_1,\dots,A_t \in k[x_1,\dots,x_n,y_1,\dots,y_m]$を得る。
    \item
    $\Leftarrow$を示す。$g\in k[y_1,\dots,y_m]$とする。
    \begin{enumerate}
      \item
      2に$y_\bullet \Leftarrow f_\bullet$を代入する。
      $g_\bullet \in \gen{f_1-y_1,\dots,f_m-y_m}$なので、
      $g_\bullet(x_1,\dots,x_n,f_1,\dots,f_m) = 0$となり、
      $f\in k[x_1,\dots,x_n]$なのんで代入するとそのまま$f$である。
      \begin{align}
        f = \tilde g(f_1,\dots,f_m).
      \end{align}
      \item
      上より、$f\in k[f_1,\dots,f_m]$となる。
    \end{enumerate}
    \item
    $\Rightarrow$を示す。$f\in k[f_1,\dots,f_m]$とする。
    \begin{enumerate}
      \item $\exists \tilde g$: $\tilde g\in k[y_1,\dots,y_m]$があって、$f=\tilde g(f_1,\dots,f_m)$とかける。
      \item
      \begin{align}
        f = C_1\cdot (f_1-y_1) + \dots + C_m\cdot (f_m-y_m) + \tilde g(y_1,\dots,y_m).
      \end{align}
      \begin{enumerate}
        \item
        $k[f_1,\dots,f_m]$の$\alpha$次の単項式は、
        \begin{align}
          f_1^{\alpha_1} \dots f_m^{\alpha_m}
          &=
          (y_1 + (f_1 - y_1))^{\alpha_1}\dots (y_m + (f_m - y_m))^{\alpha_m}\\
          &=
          y_1^{\alpha_1}\dots y_m^{\alpha_m} + B_1\cdot (f_1-y_1) + \dots + B_m\cdot (f_m-y_m).
        \end{align}
        と、$B_1,\dots,B_m \in k[x_1,\dots,x_n,y_1,\dots,y_m]$を使ってかける。
        \item
        上を係数をかけて足せば、
        \begin{align}
          \tilde g(f_1,\dots,f_m)
          =
          C_1 \cdot (f_1-y_1)  + \dots + C_m \cdot (f_m-y_m) + \tilde g(y_1,\dots,y_m)
        \end{align}
        と、$C_1,\dots,C_m \in k[x_1,\dots,x_n,y_1,\dots,y_m]$を使ってかける。
        \item
          (a)と上より、
          \begin{align}
            f
            =
            C_1 \cdot (f_1-y_1)  + \dots + C_m \cdot (f_m-y_m) + \tilde g(y_1,\dots,y_m).
          \end{align}
      \end{enumerate}
      \item
      $G'$: $G'= G\cap k[y_1,\dots,y_m]$とする。さらに、$G'=\set{g_1,\dots,g_s}$としてよい。
      \item
      $B_1,\dots,B_s,g'$:
      $\tilde g$を$G'$で割る。
      \begin{align}
        \tilde g= B_1 g_1 + \dots + B_s g_s + g'
      \end{align}
      となる$B_1,\dots,B_s , g' \in k[y_1,\dots,y_m]$が得られる。
      \item $C_1',\dots,C_m'$:
      (b),(d),$g_\bullet \in \gen{f_1-y_1,\dots,f_m-y_m}$より、
      \begin{align}
        f = C_1' \cdot (f_1-y_1) + \dots + C_m'\cdot (f_m-y_m) + g'(y_1,\dots,y_m)
      \end{align}
      となる$C_1',\dots,C_m' \in k[x_1,\dots,x_n,y_1,\dots,y_m]$が得られる。
      \item
      $g'$は$f$の割り算の余り?つまり、$g'$のどの項も$\LT(G)$の元で割り切れない?
      \begin{enumerate}
        \item $g'$のある項が$\LT(G)$のある元で割り切れると仮定する(背理法)。
        \item $\exists i$: $\LT(g_i)$は$g'$のある項を割り切る、となるような$g_i \in G$が存在する。
        \item
        $g' \in k[y_1,\dots,y_m]$なので、$\LT(g_i)$は$y_1,\dots,y_m$のみを含む。
        \item
        上と、順序付より$g_i \in k[y_1,\dots,y_m]$となる。
        \item
        上と、$g_i \in G$より、$g_i \in G'$となる。($G'$は4-(c))。
        \item
        $g'$は$G'$による割り算の余りなので(d)、
        $\LT(g_i)$は$g'$のどの項も割り切らない。
        \item
        上は、iに矛盾する。
      \end{enumerate}
      よって、$g'$は$f$の割り算の余り。$g=g'\in k[y_1,\dots,y_m]$となる。
    \end{enumerate}
  \end{enumerate}

  (ii)を示す。$f\in k[f_1,\dots,f_m]$なら、上の証明の後半の
  (4-e)と(4-f)より、
  \begin{align}
    f = C_1' \cdot (f_1-y_1) + \dots + C_m'\cdot (f_m-y_m) + g(y_1,\dots,y_m)
  \end{align}
  となっている。ここで、$y_\bullet \Leftarrow f_\bullet$とすることで、
  \begin{align}
    f = g(f_1,\dots,f_m).
  \end{align}

\end{myproof}
