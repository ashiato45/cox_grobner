\label{sec:代数と幾何の対応}

\subsection{ヒルベルトの零点定理}
\label{sub:ヒルベルトの零点定理}

問題3:
\begin{enumerate}[label=(\arabic*)]
  \item
  \begin{align}
    f(x_1,x_2,\dots,x_n)
    &=
    f(\tilde x_1,\tilde x_2 + a_2 \tilde x_1,\dots, \tilde x_n + a_n \tilde x_1)\\
    &=
    (\sum_{d=0}^N h_d)(\tilde x_1,\tilde x_2 + a_2 \tilde x_1,\dots, \tilde x_n + a_n \tilde x_1)\\
    &=
    h_N(\tilde x_1,\tilde x_2 + a_2 \tilde x_1,\dots, \tilde x_n + a_n \tilde x_1)
    +
    (ゴミ)
    \\
    &=
    (\sum_{\myabs{\alpha}=N}c_\alpha x^\alpha)(\tilde x_1,\tilde x_2 + a_2 \tilde x_1,\dots, \tilde x_n + a_n \tilde x_1)
    +
    (ゴミ)\\
    &=
    \sum_{\myabs{\alpha}=N}c_\alpha \tilde x_1^{\alpha_1}(\tilde x_2 + a_2 \tilde x_1)^{\alpha_2} \dots (\tilde x_n + a_n \tilde x_1)^{\alpha_n}
    +
    (ゴミ)\\
    &=
    \sum_{\myabs{\alpha}=N} c_\alpha \tilde x_1^{\alpha_1}(a_2\tilde x_1)^{\alpha_2}\dots (a_n \tilde x_1)^{\alpha_n}
    +
    (ゴミ)\\
    &=
    \tilde x_1^N \sum_{\myabs{\alpha}=N}c_\alpha (1\cdot a_2^{\alpha_2}\cdot \dots \cdot a_n^{\alpha_n})
    +
    (ゴミ)\\
    &=
    \tilde x_1^N \sum_{\myabs{\alpha}=N}c_\alpha (1,a_2,\dots,a_n)^\alpha
    +
    (ゴミ)\\
    &=
    \tilde x_1^N h_N(1,a_2,\dots,a_n) + (ゴミ).
    \end{align}
\end{enumerate}

問題4:
代数的閉体$K$を考える。これが仮に有限体であり、
$a_1,\dots,a_n$が$K$のすべての元であるとする。
このとき、
\begin{align}
  (x-a_1)\dots (x-a_n) = 1
\end{align}
という方程式を考える。左辺は
$a_\bullet$のどれを入れても0になるので、
$a_\bullet$はどれも根にならない。
しかし、$K$は代数的閉体なのでこの根は$K$に属さなければならないが、
先の考察よりこれは$a_1,\dots,a_n$のどれでもない。

弱系の零点定理:$I\subset k[x_1,\dots,x_n]$は$\var(I)=\emptyset$
とする。このとき、$I$は全体になってしまう。
証明する。
まず1次元で考える。PIDなので$I=\gen{f}$なる$f$がある。
$\deg f \ge 1$のときには、代数的閉体で考えてるので解が出てしまって、
$\var(I)$は空でなくなるので矛盾。よって、$\deg f = 0$となり、
$f$は定数になる。よって、$1 \in I$である。
次に$n$次元で考える。$I=\gen{f_1,\dots,f_s}$とする。
変数変換
\begin{align}
  x_1 \mapsto \tilde x_1,\quad x_i \mapsto \tilde x_i + a_i \tilde x_1
\end{align}
をかけて、
$f_1$の$\tilde x_1$についての最高次の係数は定数であるとしてよい。
変数変換の性質より、$f$が解を持つ$\iff$ $\tilde f$が解を持つので、
$\var(f) = \emptyset \iff \var(\tilde f)=\emptyset$である。
また、$\tilde \bullet$が定数に影響しないので$1\in I \iff 1\in \tilde I$となる。
\begin{align}
  \var(\tilde I_1)
  &=
  \pi_1(\var(\tilde I))\\
  &
  \fbox{$f_1$の$\tilde x_1$の係数は定数、拡張定理}\\
  &=
  \pi_1(\emptyset)\\
  &
  \fbox{$\var(I)=\emptyset$で$\var(I)=\emptyset \iff \var(\tilde I)=\emptyset$}\\
  &=
  \emptyset.
\end{align}
帰納法の仮定より、$\tilde I_1 = k[\tilde x_2,\dots,\tilde x_n]$である。
よって、$1\in \tilde I_1 \subset \tilde I$となる。
$1\in \tilde I$なので、($\tilde I$で定数は変化しない。)$1\in I$である。
よって、$I$は全体である。

ヒルベルトの零点定理
\begin{enumerate}
  \item $f$は$f_1,\dots,f_s$の共通零点$\var(f_1,\dots,f_s)$で消えるとする。
  \item $f^m = \sum_{i=1}^s A_i f_i$なる$A_i$たちを探す。
  \begin{enumerate}
    \item $\tilde I = \gen{f_1,\dots,f_s,1-yf} \subset k[x_1,\dots,x_n,y]$とする。
    \item $\var(\tilde I) = \emptyset$となる。
    \begin{enumerate}
      \item どの$(a_1,\dots,a_n,a_{n+1})\in  k^{n+1}$も$\var(\tilde I)$に属さなければよい。
      \item 頭$n$個で作った点が$f_1,\dots,f_s$のすべてで消えるとき、
      $1-yf$はこの点で消えない。$yf$がこの点で消えるからである。
      \item そうでないとき、どこかで消えないときはその$f_i$をそのまま使えば
      消えない。
    \end{enumerate}
    \item $1 \in \tilde I$が弱い零点定理からわかる。
    \item
    \begin{align}
      1 = \sum_{i=1}^s p_i(x_1,\dots,x_n,y)f_i
      +
      q(x_1,\dots,x_n,y)(1-yf)
    \end{align}
    となるように$p,q$がとれる。
    \item
    $y=1/f$とする。
    \begin{align}
      1=\sum_{i=1}^s p_i(x_1,\dots,x_n,1/f)f_i.
    \end{align}
    \item
    $f^m$をたくさんかければ、上の$1/f$が消えてのぞむ式が得られる。
  \end{enumerate}
\end{enumerate}

逆のほうは「強い」ほうで言ってる。
