\label{sec:射影代数幾何}


\subsection{射影平面}
\label{sub:射影平面}

\begin{framed}
  定義1:
  $\R$上の射影平面(projective plane)とは、
  $\P^2(\R)$と表記される次の集合。
  \begin{align}
    \P^2(\R)= \R^2 \cup \set{平行な直線からなる同値類ごとに1つの無限遠点}.
  \end{align}
\end{framed}

\begin{framed}
  定義2:
  $R^3-\zeroset$の$\sim$による同値類の全体を$\P^2(\R)$であらわす。
  つまり、
  \begin{align}
    \P^2(\R) = (\R^3-\zeroset)/\sim.
  \end{align}
  3つ組$(x,y,z)\in \R^3-\zeroset$が$p\in \P^2(\R)$に対応するとき、
  $(x,y,z)$を$p$の斉次座標(homogeneous coordinates)という。
\end{framed}

\begin{framed}
  定義3:
  同時にゼロではない実数$A,B,C$が与えられたとき、次の集合
  \begin{align}
    \set{p\in \P^2(\R); pの斉次座標(x,y,z)はAx+By+Cz=0を満たす}
  \end{align}
  を$\P^2(\R)$の射影直線とよぶ。
  これはwell-definedであることは確認できる。
\end{framed}

\begin{framed}
  命題4:
  $R^2 \to \P^2(\R),\quad (x,y)\mapsto i(x,y,1)$は一対一であって、
  その像は$z=0$で定義される射影直線$H_\infty$に一致する。
\end{framed}
\begin{myproof}
  \begin{enumerate}
    \item $\forall p,x,y,x',y'$: $(x,y)$と$(x',y')$が同じ点$p$にうつったとする。
    \item
    $\exists \lambda$ $(x,y,1) = \lambda(x',y',1)$
    \item
    上より、$\lambda = 1$となる。
    \item
    上より、$(x,y)=(x',y')$となる。
    \item
    $p$の斉次座標を$(x,y,z)$とする。
    \item
    $z=0$のとき、$p\in H_\infty$
    \item
    $z\neq 0$のとき、$\pi\colon \R^3 \to \P^2(\R)$を標準的なものとする。
    $p=\pi(x,y,z) = \pi(x/z,y/z,1)$となり、$(x/z,y/z,1)$は
    $p$の斉次座標。
    \item
    上より、$p$は写像$\R^2 \to \P^2(\R)$の像に($(x/z,y/z)$を引数として)なっている。
    \item
    $\pi(\R^2)\cap H_\infty = \emptyset$を示す。
    \begin{enumerate}
      \item $\exists$: $\pi(x,y,z) \in p(\R^2)\cap H_\infty$と仮定する。
      \item
      $\pi(x,y,z) \in H_\infty$なので、$z=0$である。
      \item
      $\pi(x,y,z) \in p(\R^2)$なので、$\pi(x,y,z) = \pi(\xi,\eta,1)$
      なる$\xi,\eta$が存在する。よって、$z\neq 0$である。
      \item
      上2つは矛盾する。
    \end{enumerate}
    よって、$\pi(\R^2)\cap H_\infty = \emptyset$となる。
  \end{enumerate}
\end{myproof}

\subsection{射影空間と射影多様体}
\label{sub:射影空間と射影多様体}
\begin{framed}
  定義1:
  $k^{n+1}-\zeroset$の$\sim$による同値類の集合を体$k$上
  の$n$次元射影空間といい、$\P^n(k)$とあらわす。
  つまり、
  \begin{align}
    \P^n(k) = (k^{n+1}-\zeroset)/\sim
  \end{align}
  である。ゼロでないような$(n+1)$個の$k$の要素の組
  $(x_0,\dots,x_n)\in k^{n+1}$は$\P^n(k)$の点$p$を決めるが、
  $(x_0,\dots,x_n)$を$p$の斉次座標とよぶ。
\end{framed}

\begin{framed}
  $\P^n(k)$の部分集合を
  \begin{align}
    U_0 = \set{(x_0,\dots,x_n) \in \P^n(k); x_0 \neq 0}
  \end{align}
  とすると、$k^n$の点$(a_1,\dots,a_n)$を$\P^n(k)$の斉次座標
  $(1,a_1,\dots,a_n)$に写す写像$\phi$は$k^n$と$U_0\subset \P^n(k)$
  の間の一対一写像である。
\end{framed}
\begin{myproof}
  $\phi(a_1,\dots,a_n)=(1,a_1,\dots,a_n)$の先頭が0でないので、
  $\phi\colon k^n \to U_0$は定まっている。

  $\psi\colon U_0 \to k^n$を
  $\psi(\ub{x_0}_{\neq 0},\dots,x_n) = \psi(1,x_1/x_0,\dots,x_n/x_0) = (x_1/x_0,\dots,x_n/x_0)$となる。
  well-definedと逆写像は示せる。
\end{myproof}

\begin{align}
  \P^n(k) = \ub{k^n}_{無限遠超平面。頭が0のところ} \cup \ub{\P^{n-1}(k)}_{頭が非0のところ}
\end{align}

\begin{framed}
  系3:
  $i=0,\dots,n$それぞれに対して、
  \begin{align}
    U_i = \set{(x_0,\dots,x_n)\in \P^n(k); x_i\neq 0}
  \end{align}
  とおく。
  \begin{enumerate}[label=(\roman*)]
    \item
    $U_i$の点は$k^n$の点と一対一に対応する。
    \item
    補集合$\P^n(k)-U_i$は$\P^{n-1}(k)$同一視できる。
    \item
    $\P^n(k) = \bigcup_{i=0}^n U_i$となる。
  \end{enumerate}
\end{framed}
\begin{myproof}
  i,iiは変数のつけかえで命題2に帰着する。iiiは、
  $\cup$をとることで$x_1\neq 0 \vee \dots \vee x_n \neq 0$で、
  $\P^n(k)$は全部座標が0になることはないので全体になっている。
\end{myproof}

射影空間の多様体は、斉次なものを使わないとうまくいかない。
\begin{framed}
  命題4:
  $f\in k[x_0,\dots,x_n]$を斉次多様体とする。
  もし$f$が点$p\in\P^n(k)$のある斉次座標の組に対して
  消えていれば、$f$は$p$の任意の斉次座標に対して消える。
  とくに$\var(f) = \set{p\in \P^n(k); f(p)=0}$は
  $\P^n(k)$の部分集合として矛盾なく定義される。
\end{framed}
\begin{myproof}
  略。
\end{myproof}

\begin{framed}
  定義5:
  $k$を体とし、$f_1,\dots,f_s \in k[x_0,\dots,x_n]$を斉次多項式とする。
  \begin{align}
    \var(f_1,\dots,f_s)
    =
    \set{(a_0,\dots,a_n) \in \P^n(k); f_i(a_0,\dots,a_n) = 0 \quad (1\le i \le s)}
  \end{align}
  とおいて、$\var(f_1,\dots,f_s)$を$f_1,\dots,f_s$によって定義された射影多様体とよぶ。
\end{framed}

「1つの」斉次多項式で定義された射影多様体は「$n$次超曲面」という。

射影多様体と多様体を考える。$x_0=1$として$V\cap U_0$に斉次多項式を落とすことを
非斉次化という。
\begin{framed}
  命題6:
  $V=\var(f_1,\dots,f_s)$を射影多様体とする。すると
  $W=V\cap U_0$はアフィン多様体$\var(g_1,\dots,g_s) \subset k^n$
  と同一視できる。ここで、$1\le i \le s$に対して、
  $g_i(x_1,\dots,x_n) = f_i(1,x_1,\dots,x_n)$である
  \footnote{$U_0$は頭が非0のやつ。}
  。
\end{framed}
\begin{myproof}
  \begin{enumerate}
    \item $\psi(W) \subset \var(g_1,\dots,g_s)$となる。$\psi\colon U_0 \to k^n$は、射影座標を頭が1になるように正規化して頭を落とす写像であった。
    \begin{enumerate}
      \item $\forall x_\bullet$:
      $(x_1,\dots,x_n)\in \psi(W)$とする。
      $\psi(1,x_1,\dots,x_n) = (x_1,\dots,x_n)$であり、
      $(1,x_1,\dots,x_n) \in V$となっている。
      \item 任意の$i$について、上の$(1,\dots,x_n)\in V$より
      \begin{align}
        g_i(x_1,\dots,x_n)
        =
        f_i(1,x_1,\dots,x_n)
        =0.
      \end{align}
      \item (a)おわり:
      上より、$(x_1,\dots,x_n) \in \var(g_1,\dots,g_s)$となる。
    \end{enumerate}
    \item
    $\supset$を示す。
    \begin{enumerate}
      \item $\forall a_\bullet$: $(a_1,\dots,a_n) \in \var(g_1,\dots,g_s)$とする。
      \item $(1,a_1,\dots,a_n) \in U_0$である。
      \item 任意の$i$について、
      \begin{align}
        f_i(1,a_1,\dots,a_n) = g_i(a_1,\dots,a_n) = 0.
      \end{align}
      \item
      上より、$\phi(\var(g_1,\dots,g_s)) \subset W$となる。
    \end{enumerate}
    \item
    $\phi$と$\psi$は逆写像なので、$W$と$\var(g_1,\dots,g_s)$の点は一対一に対応する。
  \end{enumerate}
\end{myproof}

非斉次化の逆を考える。$f\in k[x_1,\dots,x_n]$について、
すべての項の全次数が$\deg(f)$になるように各項に$x_0$の羃をかけたものを
$f^h$という。

\begin{framed}
  命題7:
  $g(x_1,\dots,x_n) \in k[x_1,\dots,x_n]$を全次数$d$の多項式とする。
  \begin{enumerate}[label=(\roman*)]
    \item
    $g$を斉次成分の和に展開して、$g=\sum_{i=0}^d g_i$とかく。
    ここで$g_i$の全次数は$i$である。すると、
    \begin{align}
      g^h(x_0,\dots,x_n)
      =
      \sum_{i=0}^d g_i(x_1,\dots,x_n) x_0^{d-i}
    \end{align}
    は全次数が$d$であるような$k[x_0,\dots,x_n]$の斉次多項式である。
    この$g^h$を$g$の斉次化という。
    \item
    斉次多項式は次で計算できる。
    \begin{align}
      g^h = x_0^d\cdot g(\frac{x_1}{x_0},\dots,xfra x_n x_0).
    \end{align}
    \item
    $g^h$を非斉次化すると$g$になる。
    \begin{align}
      g^h(1,x_1,\dots,x_n)  = g(x_1,\dots,x_n).
    \end{align}
    \item
    $F(x_0,\dots,x_n)$を斉次多項式とし、$x_0^e$を$F$を割り切るような
    $x_0$の冪乗のうち最高次のものとする。もし$f=F(1,x_1,\dots,x_n)$が
    $F$の非斉次化なら、$F=x_0^e\cdot f^h$がなりたつ。
  \end{enumerate}
\end{framed}
\begin{myproof}
  (i)はあきらか。

  (ii)を示す。
  \begin{align}
    g^h(x_0,\dots,x_n) &=
    \sum_{i=0}^d g_i(x_1,\dots,x_n)x_0^{d-i}\\
    &=
    x^d \sum_{i=0}^d \frac{g_i(x_1,\dots,x_n)}{x_0^i} \\
    &=
    x^d \sum_{i=0}^d g_i(\frac{x_1}{x_0},\dots,\frac{x_n}{x_0}) .
  \end{align}

  (iii),(iv)はあきらか。
\end{myproof}

\subsection{射影化された代数-幾何対応}
\label{sub:射影化された代数-幾何対応}

\begin{framed}
  定義1:
  $k[x_0,\dots,x_n]$のイデアル$I$が斉次であるとは、
  各$f\in I$に対して、$f$の斉次成分$f_i$がまた$I$に属しているときに言う。
\end{framed}

\begin{framed}
  定理2:
  $I\subset k[x_0,\dots,x_n]$をイデアルとする。このとき、次は同値である。
  \begin{enumerate}[label=(\roman*)]
    \item $I$は$k[x_0,\dots,x_n]$の斉次イデアルである。
    \item
    斉次多項式$f_1,\dots,f_s$を用いて、$I=\gen{f_1,\dots,f_s}$とあらわせる。
    \item
    任意の多項式順序に対して、$I$の簡約グレブナ基底は斉次多項式からなる。
  \end{enumerate}
\end{framed}
\begin{myproof}
  (ii)$\implies$(i)を示す。そのために、演習問題2を解く。
  \begin{enumerate}[label=(演習2-\alph*)]
    \item 「$f=\sum_i f_i$と$g=\sum_i g_i$を2つの多項式の斉次成分の和への分解とする。
    このとき、$f=g \iff \Forall{i} f_i = g_i$を示せ。」
    $\Leftarrow$はあきらか。$\Rightarrow$を示す。
    $f-g$を考え、$f=0 \implies \Forall{i}f_i = 0$を示せば十分。
    $i$が次数をあらわしているとする。仮に何か$f_i \neq 0$があるなら、
    それを打ち消すものが他の$f_\bullet$にはない。

    あるいは、$\sum_{j\neq i}f_i = -f_i$と変形し、両方の次数が違うので$f_i=0$である…というのを帰納的にやってもよい。
    \item
    「$f=\sum_i f_i$と$g=\sum_i g_i$を2つの多項式の斉次成分の和への分解とする。
    このとき、$h=fg$の斉次成分は$h_k = \sum_{i+j = k}f_i\cdot g_j$で与えられていることを示せ。」
    $h=fg$の項は$\set{f_i\cdot g_j; i,j\in \Zge^n}$である。
    よって、$h$の多重次数$k$なものは、$i+j=k$となる$(i,j)$達だけであり、成り立つ。
    \item
    「(ii)$\implies$(i)を示せ。」
    $f\in I$とし、これの斉次成分への分解$f=\sum_{d}g_d$とする。各$g_d$は$d$次である。
    $f_i$たちに番号をつけかえて、$f_{ij}$ただし$i$が多重次数で、
    $j$がそれらのうちの番号とする。$f\in I$なので、$f=\sum_{i,j} h_{ij}f_{ij}$と$h_{ij}\in k[x_1,\dots,x_n]$で表示される。
    さらに、$h_{ij}=\sum_{k} h_{ijk}f_{ij}$と斉次成分に分解する。各$f_{ijk}$は$k$次である。
    さらに、(b)より、$f=\sum_d \sum_{i+k=d} \sum_j h'_{ijk}f_{ij}$と和をとりかえる。
    よって、$\sum_d g_d = \sum_{d}\sum_{i+k=d} \sum_j h'_{ijk} f_{ij}$となる。
    (a)より、各々の斉次成分が等しいので、
    $g_d = h'_{ijk}f_{ij}$となっている。よって、$g_d \in I$である。
    示された。
  \end{enumerate}

  (i)$\implies$(ii)を示す。$I$は斉次イデアルであるとする。
  \begin{enumerate}
    \item $\exists F_\bullet$:
    ヒルベルトの基底定理より、
    \begin{align}
      I=\gen{F_1,\dots,F_t}
    \end{align}
    となる$F_1,\dots,F_t \in k[x_1,\dots,x_n]$が存在する。
    (これは斉次とは限らない。)
    \item
    $F_{\bullet \bullet}$:
    各$j$について、$F_j$を斉次成分に分け、$F_j = \sum_{i}F_{ji}$と描く。
    \item
    $I$は斉次イデアルであることと、各$j$について$F_j \in I$であること、
    各$F_{ji}$が$F_j$の項であることから、$F_{ji} \in I$である。
    \item $I'$:
    $I'=\gen{F_{ji}; i,j}$とする。
    \item
    2の$F_j = \sum_i F_{ji}$より、$F_j \in I'$がわかり、$I \subset I'$である。
    \item 3より、$I'\subset I$である。
    \item 5,6より$I=I'$であり、$I$の基底として斉次なもの$F_{ji}$たちが得られた。
  \end{enumerate}

  (ii)$\iff$(iii)を示す。演習問題3を解く。
  \begin{enumerate}[label=(演習3-\alph*)]
    \item
    「割り算アルゴリズムを用いて、斉次多項式$f$を斉次多項式たち
    $f_1,\dots,f_s$で割り算したとせよ。その結果、
    $f=a_1f_1+\dots+a_s f_s + r$という表示が得られる。
    このとき、商$a_1,\dots,a_s$および、余り$r$は斉次多項式(0かもしれない。)
    であることを証明せよ。$r$の全次数はいくらになるか?」
    はじめに暫定的な余り$r$は$f$になっているが、
    ここからある単項式$c\alpha x^\alpha$として$f_i \cdot c_\alpha x^\alpha$を引いて次数を下げても、
    $r$の全次数は変化しない。そしてこのとき、商$a_i$には$cx^\alpha$が追加されるが、
    これの次数は$\deg(c_\alpha x^\alpha)=\deg(f)-\deg(f_i)$である。
    よって、商も余りも(0でなければ)斉次であり続ける。$r$は0でなければ全次数は$\deg(f)$である。
    \item
    「$f,g$を斉次多項式とすると、S多項式$S(f,g)$もまた斉次であることを示せ。」
    $x^\gamma = \LCM(\LM(f),\LM(g))$として、
    \begin{align}
      S(f,g) = \frac{x^\gamma}{\LT(f)}f - \frac{x^\gamma}{\LT(g)}g
    \end{align}
    であった。
    \begin{align}
      (x^\gamma/\LT(f) \cdot f の各項の次数)
      =
      \deg(\frac{x^\gamma}{\LT(f)}f)
      =
      \deg(x^\gamma).
    \end{align}
    同様に、$(x^\gamma/\LT(f) \cdot g の各項の次数)  = \deg(x^\gamma)$である。
    よって、$S(f,g)$のどの項の次数も$\deg(x^\gamma)$であり、斉次である。
    \item
    「ブッフベルガーのアルゴリズムを解析することによって、
    斉次イデアルは斉次多項式からなるグレブナ基底を持つことを示せ。」
    ブッフベルガーのアルゴリズムはS多項式を追加し続ける
    ものだが、スタートが斉次だったので追加したものも(b)より斉次しか追加されない。
    よって、停止時点でも斉次な多項式しかなく、
    斉次イデアルには斉次なグレブナ基底がある。
    \item
    「(ii)$\iff$(iii)を示せ。」
    (iii)$\implies$(ii)はあきらか。(ii)$\implies$(iii)を示す。
    (ii)を仮定する。ここからブッフベルガーのアルゴリズムを使うことで、
    (c)より$I$の斉次なグレブナ基底が得られる。
    斉次であることを保ちながら、これを極小グレブナ基底にすることができる
    \footnote{先頭項係数を1にして、不要なものを除く。}
    。さらにこれに簡約グレブナ基底を作るアルゴリズムを適用しても
    斉次であり続けることを示す。このアルゴリズムは、
    グレブナ基底の各元$g$に対して、$g$を$\ovd{g}{G-\set{g}}$で置換するものであった。
    この操作で、$g$が斉次であることは変化しないことを示す。
    実際、(a)より余りは斉次であり続けるし(割り算をして0になることはない。仮にそうなれば
    極小にしたことに反する。)、その次数は$g$と変わらない。
    よって、$I$の斉次な基底から、斉次な簡約グレブナ基底を作ることができる。
  \end{enumerate}
\end{myproof}

\begin{framed}
  命題3:
  $I\subset k[x_0,\dots,x_n]$を斉次イデアルとして、
  斉次多項式$f_1,\dots,f_s$に対して、
  $I=\gen{f_1,\dots,f_s}$であると仮定する。すると、
  \begin{align}
    \var(I) = \var(f_1,\dots,f_s)
  \end{align}
  であり、したがって$\var(I)$は射影多様体である。
\end{framed}
\begin{myproof}
  演習5を解く。
  \begin{align}
    \var(I)
    &=
    \set{(a_0:\dots:a_n) \in \P^n(k); \Forall{f\in I}f(a_0,\dots,a_n) = 0}\\
    &=
    \set{(a_0:\dots:a_n) \in \P^n(k); \Forall{i=1,\dots,s}f_i(a_0,\dots,a_n) = 0}\\
    &=
    \var(f_1,\dots,f_s).
  \end{align}
\end{myproof}

\begin{framed}
  射影多様体$V\subset \P^n(k)$に対して、
  \begin{align}
    \ideal(V) = \set{f\in k[x_0,\dots,x_n];
    \Forall{(a_0:\dots:a_n)\in V} f(a_0,\dots,a_n) = 0}
  \end{align}
  とおく(ここで$f$は$V$の任意の点のすべての斉次座標に対して消えていなければならないことに注意せよ。)。
  もし$k$が無限体であれば、$\ideal(V)$は$k[x_0,\dots,x_n]$の斉次イデアルである。
\end{framed}
\begin{myproof}
  イデアルであることはあきらか。
  $\ideal(V)$が斉次であることを示す。
  \begin{enumerate}
    \item $\forall f$: $f\in \ideal(V)$とする。
    \item $\forall p$: $p\in V$とする。
    \item 仮定より、$f$は$p$のすべての斉次座標$(a_0,\dots,a_n)$に対して消える。
    \item $f$の任意の斉次成分$f_i$は$(a_0,\dots,a_n)$で消える?
    \begin{enumerate}
      \item     演習問題2-7を解く。(これまでの文字は忘れる。)
      「$k$を無限体とする。もし$f\in k[x_0,\dots,x_n]$
      が斉次多項式ではなく、しかも$p\in \P^n(k)$のすべての斉次座標で
      消え$f$の任意の斉次成分$f_i$も$p$で消えていなければならないことを示そう。
      \begin{enumerate}[label=(演習2-7-\alph*)]
        \item
        「$f$を斉次成分の和として、$f=\sum_i f_i$と表そう。
        $p=(a_0,\dots,a_n)$とするとき、次の式を示せ。
        \begin{align}
          f(\lambda a_0,\dots,\lambda a_n)
          =
          \sum f_i(\lambda a_0, \dots, \lambda a_n)
          =
          \sum_i \lambda^if_i(a_0,\dots,a_n).
        \end{align}
        」
        自明。
        \item
        「$f$がすべての$\lambda\neq 0\in k$に対して消えれば、
        $f_i(a_0,\dots,a_n)=0$がすべての$i$について成り立つことを示せ。」
        $f(\lambda a_0,\dots,\lambda a_n)$を$k[\lambda]$の元と見る。
        これが$\lambda\neq 0$で消えること、
        それに$k$が無限体であることから、この$\lambda$に関する方程式は
        無数の解を持つことになる。そのような多項式は0しかないので、
        $f(\lambda a_0,\dots,\lambda a_n) = 0_{k[\lambda]}$である。
        (a)より、$\sum_i \lambda^i f_i(a_0,\dots,a_n)$も
        $\lambda$に関する0多項式であることがわかる。よって、
        すべての$i$について、
        $f_i(a_0,\dots,a_n) = 0$である。
      \end{enumerate}
      \item
      3は、$\Forall{\lambda \neq 0}f(\lambda a_0,\dots,\lambda a_n) = 0$を主張しているので、
      (a)で解いた問と$k$が無限体であることより、
      $f$のすべての斉次成分$f_i$が$f_i(a_0,\dots,a_n)=0$をみたす。
    \end{enumerate}
    よって、$f$の任意の斉次成分$f_i$は$(a_0,\dots,a_n)$で消える。
    \item
    2おわり: 上より、任意の$i$について、$f_i \in \ideal(V)$である。
    \item
    1おわり: 上より、$f$の任意の斉次成分が$I$に属するので、
    $\ideal(V)$は斉次である。
  \end{enumerate}
\end{myproof}

\begin{framed}
  定理5:
  $k$を無限体とする。写像
  \begin{align}
    射影多様体 \xrightarrow{\ideal} 斉次イデアル
  \end{align}
  と
  \begin{align}
    斉次イデアル \xrightarrow{\var} 射影多様体
  \end{align}
  は包含関係を逆転させる。さらに、任意の射影多様体に対して、
  \begin{align}
    \var(\ideal(V)) = V
  \end{align}
  が成り立つ。特に$\ideal$は単射である。
\end{framed}
\begin{myproof}
  $\ideal$の反転を示す。$\var(f_1,\dots,f_s) \subset \var(g_1,\dots,g_t)$とする。
  \begin{align}
    \ideal(\var(g_1,\dots,g_t))
    &=
    \set{f \in k[x_0,\dots,x_n]; f は \var(g_1,\dots,g_t)を消す}\\
    &\subset
    \set{f \in k[x_0,\dots,x_n]; f は \var(f_1,\dots,f_s)を消す}\\
    &=
    \ideal(\var(f_1,\dots,f_t)).
  \end{align}

  $\var$の反転を示す。$\gen{f_1,\dots,f_s} \subset \gen{g_1,\dots,g_t}$とする。
  \begin{align}
    \var(\gen{g_1,\dots,g_t})
    &\desceq{命題3}
    \var(g_1,\dots,g_t)\\
    &\subset
    \var(f_1,\dots,f_s)\\
    &\desceq{命題3}
    \var(\gen{f_1,\dots,f_s}).
  \end{align}

  $\var(\ideal(V)) = V$を示す。
  \begin{enumerate}
    \item $V=\var(f_1,\dots,f_s)$とする。
    $\var(\ideal(\var(f_1,\dots,f_s))) = \var(f_1,\dots,f_s)$を示せばよい。
    \item $\subset$を示す。
    \begin{enumerate}
      \item $f_1,\dots,f_s$は$\var(f_1,\dots,f_s)$を消す。
      \item 上より、
      $f_1,\dots,f_s \in \ideal(\var(f_1,\dots,f_s))$となる。
      \item 上より、$\gen{f_1,\dots,f_s} \subset \ideal(\var(f_1,\dots,f_s))$がなりたつ。
      \item 上と$\var$の反転より、
      \begin{align}
        \var(\ideal(\var(f_1,\dots,f_s))) \subset \var(\gen{f_1,\dots,f_s}) \desceq{命題3} \var(f_1,\dots,f_s).
      \end{align}
    \end{enumerate}
    \item $\supset$を示す。
    \begin{enumerate}
      \item $\forall (a_0:\dots:a_n)$: $(a_0:\dots:a_n) \in \var(f_1,\dots,f_s)$とする。
      $f_1,\dots,f_s$全ては$(a_0:\dots:a_n)$を消す。
      \item
      $\forall f$: $f\in \ideal(\var(f_1,\dots,f_s))$とする。
      ($f$は$(a_0:\dots:a_n)$を消す?)
      \item
      上より、$f$は$\var(f_1,\dots,f_s)$を消す。
      \item
      上と、(a)より、$f$は$(a_0:\dots:a_n)$を消す。
      \item
      (b)おわり: $(a_0:\dots:a_n)$は$\ideal(\var(f_1,\dots,f_s))$のどれでも消える。
      \item
      (a)おわり: $\var(f_1,\dots,f_s)$は$\ideal(\var(f_1,\dots,f_s))$のどれでも消える。
      \item
      上より、
      \begin{align}
        \var(\ideal(\var(f_1,\dots,f_s)))
        &=
        \set{\ideal(\var(f_1,\dots,f_s))のどれでも消える点}\\
        &\supset
        \var(f_1,\dots,f_s).
      \end{align}
    \end{enumerate}
    \item 2,3よりなりたつ。
  \end{enumerate}
\end{myproof}

\begin{framed}
  定理6:
  $k$を無限体とする。
  \begin{enumerate}[label=(\roman*)]
    \item $\P^n(k)$に含まれる射影多様体の降鎖
    \begin{align}
      V_1 \supset V_2 \supset V_3 \supset \dots
    \end{align}
    に対して、ある整数$N$が存在して、$V_N = V_{N+1} = \dots$が成り立つ。
    \item
    任意の射影多様体$V\subset \P^n(k)$は、有限個の既約な射影多様体
    の和集合として一意的に表される。
    \begin{align}
      V=V_1 \cup \dots \cup V_m.
    \end{align}
    ただし、$i\neq j$に対しては$V_i \not\subset V_j$である。
  \end{enumerate}
\end{framed}
\begin{myproof}
  (i)を示す。鎖に$\ideal$をとって昇鎖を作る。
  ネーター環の昇鎖は安定するので、ある$N$以上$\ideal(V_N) = \ideal(V_{N+1})=\dots$となる。
  定理5の$\ideal$の単射より、$V_N = V_{N+1} = \dots$となる。

  (ii)を示す。2通りに書いて、1個既約多様体とってもう片方のどこに含まれますか~みたいなことを言っていればできる。
\end{myproof}

斉次イデアルの演算と射影多様体の演算を考える。
\begin{framed}
  演習6:
  $I_1,\dots,I_l$を$k[x_0,\dots,x_n]$の斉次イデアルとする。
  \begin{enumerate}[label=(\alph*)]
    \item $I_1 + \dots + I_l$は斉次。
    \item $I_1 \cap \dots \cap I_l$は斉次。
    \item $I_1\dots I_l$は斉次。
  \end{enumerate}
\end{framed}
\begin{myproof}
  (a)を示す。定理2より、各$I_i$には斉次な生成元$f_{i1},\dots,f_{iN_i}$がある。
  \begin{align}
  I_1 + \dots + I_l
  &=
   \gen{f_1 + \dots + f_l; f_1\in I_1,\dots,f_l \in I_l}  \\
   &\desceq{命題4-3-2}
   \gen{f_{11},\dots,f_{1N_1},\dots,f_{i1},\dots,f_{iN_i}, \dots,f_{l1},\dots,f_{lN_l}}.
  \end{align}
  生成元がすべて斉次なので、定理2より$I_1+\dots+I_l$も斉次イデアルである。

  (b)を示す。
  $f\in I_1\cap \dots \cap I_l$とする。$i=1,\dots,l$とする。
  $f\in I_i$である。$I_i$は斉次なので、$f$の各項も$I_i$に属する。
  $i$は任意なので、$f$の各項も$I_1 \cap \dots \cap I_l$に属する。
  $f$は任意なので、$I_1 \cap \dots \cap I_l$は斉次。

  (c)を示す。
  $I_\bullet$の生成元を(a)のときと同様にする。
  命題4-6-3によれば、
  \begin{align}
    I_1 \dots I_l
    &=
    \gen{f_{1i_1}\dots f_{li_l}; 1\le i_1 \le N_1,\dots, 1\le i_N \le N_l}
  \end{align}
  となっている。生成元を斉次になるようにとっておいたので、各$j$について
  $f_{ji_j}$の全次数は$i$に依らず一定であり、
  $f_{1i_1}\dots f_{li_l}$の全次数は$i_1,\dots,i_l$の選び方に依らず一定である。
  よって、$I_1\dots I_l$は斉次な基底で生成されており、
  定理2より$I_1\dots I_l$は斉次イデアルである。
\end{myproof}
アフィンと同様に次が成り立つ。
\begin{framed}
  練習問題7:
  $I_1,\dots,I_l$を$k[x_0,\dots,x_n]$の斉次イデアルとして、
  $V_i = \var(I_i)$を対応する$\P^n(k)$の射影多様体とする。
  \begin{enumerate}[label=(\alph*)]
    \item $\var(I_1+\dots+I_l) = \bigcap_{i=1}^l V_i$である。
    \item
    \begin{align}
      \var(I_1 \cap \dots \cap I_l) = \var(I_1 \dots I_l) = \bigcup_{i=1}^l V_i.
    \end{align}
  \end{enumerate}
\end{framed}

斉次イデアルは
\begin{align}
  \sqrt{I} = \set{f\in k[x_0,\dots,x_n]; ある m\ge 1 について f^m \in I}.
\end{align}

\begin{framed}
  命題7:
  $I\subset k[x_0,\dots,x_n]$を斉次イデアルとする。
  すると、$\sqrt{I}$も斉次イデアルである。
\end{framed}
\begin{myproof}
  \begin{enumerate}
    \item $\forall f$: $f\in \sqrt{I}$とする。斉次成分に興味があるので、$f\neq 0$としてよい。
    \item $\exists m$: $m\ge 1$があって、$f^m \in I$となる。
    \item
    $f_i$: $f$を斉次成分に分解する。$f=\sum_i f_i$としておく。
    \item
    $f_\max$: $f_\max$をゼロでない斉次成分のうち、最大の全次数を持つような成分とする。
    \begin{align}
      f = f_\max + \sum_{i < \max} f_i
    \end{align}
    となる。
    \item 4より、$f^m$を展開することを考えると
    \begin{align}
      (f^m)_\max = (f_\max)^m
    \end{align}
    となる。
    \item
    $I$が斉次イデアルであること、2の$f^m \in I$より、$(f^m)_\max \in I$だえる。
    \item
    5と上より、$(f_\max)^m \in I$である。
    \item
    上より、$f_\max \in \sqrt{I}$である。
    \item
    $g$: $g=f-f_\max$とする。
    \item 1と8より、
    $f,f_\max \in \sqrt{I}$なので、上より$g\in \sqrt{I}$である。
    \item
    2-8の議論を$g$に繰替えすと、$g_\max \in \sqrt{I}$である。
    \item
    以降、9-11の議論を繰替えすことによい、$f$のすべての項が$\sqrt{i}$
    に属することがわかる。
    \item 1おわり:$f \in \sqrt{I}$は任意だったので、$\sqrt{I}$は斉次イデアルである。
  \end{enumerate}
\end{myproof}

射影幾何だと、弱形の零点定理はそのまま持ってこれない。
\begin{framed}
  定理8(射影幾何における弱形の零点定理)
  $k$を代数的閉体として、$I$を$k[x_0,\dots,x_n]$の斉次イデアルとする。
  すると次は同値である。
  \begin{enumerate}[label=(\roman*)]
    \item $\var(I)\subset \P^n(k)$は空である。
    \item $G$を$I$の(ある単項式順序に関する)簡約グレブナ基底とする。
    すると任意の$0\le i \le n$に対して、$\LT(g)$が$x_i$の非負羃であるような
    $g\in G$が存在する。
    \item
    任意の$0\le i \le n$に対して、ある整数$m_i \ge 0$が存在して、
    $x_i^{m_i} \in I$が成り立つ。
    \item
    ある$r\ge 1$が存在して、$\gen{x_0,\dots,x_n}^r \subset I$が成り立つ。
  \end{enumerate}
\end{framed}
\begin{myproof}
  \begin{enumerate}
    \item $C_V$:
    $C_V=\var_a(I) \subset k^{n+1}$を、$I$で定義されるアフィン多様体とする。
    (これは、$I$で定義される射影多様体の各点に対応する点をすべて含む。)
    \item
    (ii)$\implies$(i)を示す。
    簡約グレブナ基底$G$で、任意の$i$に対しある$g\in G$が存在して、
    ある$m_i\ge 0$に対して$\LT(g) = x_i^{m_i}$となるものがあるとする。
    \begin{enumerate}
      \item 上の状況は、$\gen{\LT(I)} = \gen{\LT(G)} = \gen{x_0^{m_0}, x_1^{m_1},\dots,x_n^{m_n}}$となっている
      ($G$は簡約されている。)。
      \item このとき、定理6-3-6によれば$C_V$は有限集合である。
      概略を示す。
      \begin{enumerate}
        \item $\forall i$: $i=0,\dots,n$
        \item $\exists g$: (a)より、$\LT(g)=x_i^{m_i}$となる$g$が存在する。
        \item 上より、$(x_0,\dots,x_n) \in \var(I)$ならば
        $g=0$を満たさなければならない。
        \item $g$を$x_i$に関する方程式と見做せば、
        $x_i$は$m_i$個以下であることがわかる(代数的閉体であることは使っていない。)。
        \item iおわり。任意の$i$について$x_i$は$m_i$個以下なので、
        $x$は$m_0\cdot \dots \cdot m_n$個以下である。
      \end{enumerate}
      よって、$C_V$は有限集合である。
      \item
      $\exists p$: $p\in V$が存在したとする(背理法)。
      \item
      $(a_0,\dots,a_n)$: $(a_0,\dots,a_n)$を$p$の斉次座標とする。
      \item
      任意の$\lambda$について、$\lambda(a_0,\dots,a_n) \in C_V$となる。
      \item
      $k$は代数的閉体ゆえ無限体なので、上より$C_V$は無限に元を含む。
      \item (c)おわり:
      上は、(b)に矛盾する。よって、$\var(I)=V=\emptyset$である。
    \end{enumerate}
    \item (iii)$\implies$(ii):
    \begin{enumerate}
      \item $G$: $G$を $I$のグレブナ基底とする。
      \item $\forall i$: $i=0,\dots,n$とする。
      \item $\exists m_i$:
      (iii)より、$x_i^{m_i} \in I$となる$m_i \ge 0$がある。
      \item
      上より、
      \begin{align}
        x_i^{m_i}= \LT(x_i^{m_i}) \in \LT(I) = \gen{\LT(I)} = \gen{\LT(G)}
      \end{align}
      となる。
      \item $\exists g$:
      上より、$\LT(g) | x_i^{m_i}$となる$g\in G$が存在する。
      \item
      上より、$\LT(g)$は$x_i$の羃である。
      \item (b)おわり: 任意の$i=0,\dots,n$について、
      $g \in G$で$\LT(g_i)$が$x_i$の羃であるものが存在する。
    \end{enumerate}
    \item (iv)$\implies$(iii):
    \begin{enumerate}
      \item $\forall i$: $i=0,\dots,n$
      \item $\exists r$: 仮定より、$r\ge 1$で、$\gen{x_0,\dots,x_n}^r \subset I$となるものがある。
      \item
      上より、$x_i^r \in I$である。
      \item
      (a)おわり: 任意の$i$について、ある整数$m_i \ge 0$が存在して、$x_i^{m_i} \in I$となる。
    \end{enumerate}
    \item (i)$\implies$(iv):
    \begin{enumerate}
      \item 仮定より、$V=\emptyset$である
      \item
      $C_V \subset \set{(\banme{0}{0},\dots,\banme{n}{0})}$?
      \begin{enumerate}
        \item $\exists (a_0,\dots,a_n)$:
        $C_V$がゼロでない点$(a_0,\dots,a_n)$を持つとする(背理法)。
        \item
        上より$(a_0,\dots,a_n) \neq 0$なので、
        $(a_0:\dots:a_n) \in V$となる。
        \item iおわり:
        上は(a)に矛盾。
      \end{enumerate}
      よって、$C_V \subset \set{(\banme{0}{0},\dots,\banme{n}{0})}$。
      \item
      上に$\ideal_a$をかける。$\ideal_a(\set{(0,\dots,0)})\subset \ideal_a(C_V)$となる。
      \item
      $\ideal_a(\set{(0,\dots,0)}) = \gen{x_0,\dots,x_n}$である\footnote{$f\in \ideal_a(\set{(0,\dots,0)})$として、$f$を$x_0,\dots,x_n$で割る。}。
      \item
      $k$は代数的閉体なので、アフィン多様体の強形の零点定理により、
      \begin{align}
        \ideal_a(C_V) = \ideal_a(\var_a(I)) = \sqrt{I}.
      \end{align}
      \item
      (c),(d),(e)より、
      \begin{align}
        \gen{x_0,\dots,x_n}
        \desceq{(d)}
        \ideal_a(\set{(0,\dots,0)})
        \descsubset{(c)}
        \ideal_a(C_V)
        \desceq{(e)}
        \sqrt{I}.
      \end{align}
      \item
      $\exists r$: 上より、$\gen{x_0,\dots,x_n}^r \subset I$となる$r\ge 1$が存在する。
      (なぜなら、各$i$について$x_i^{r_i} \in I$となる$r_i$があるが、
      このとき$r=r_0+r_1 + \dots +r_n$とすれば、$\gen{x_0,\dots,x_n}^r$からどうとっても$x_i^{r_i}$が因子として入っている。)
    \end{enumerate}
  \end{enumerate}
\end{myproof}

\begin{framed}
  定理9(射影幾何における強形の零点定理):
  $k$を代数的閉体として、$I$を$k[x_0,\dots,x_n]$の斉次イデアルとする。
  $V=\var(I)$が$\P^n(k)$の空でない射影多様体であれば、
  $\ideal(\var(I)) = \sqrt{I}$が成り立つ。
\end{framed}
\begin{myproof}
  \begin{enumerate}
    \item $V$:
    $V=\var(I) \subset \P^n(k)$とする。
    \item $C_V$:
    $C_V = \var_a(I) \subset k^{n+1}$とする。
    \item
    仮定より、$V\neq \emptyset$である。
    \item $\ideal_a(C_V) = \ideal(V)$?
    \begin{enumerate}
      \item $\subset$を示す。
      \begin{enumerate}
        \item $\forall f$: $f\in \ideal_a(C_V)$とする。
        \item
        $\forall p$: $p\in V$とする。
        \item
        $\forall (a_0,\dots,a_n)$: $p$の斉次座標を$(a_0,\dots,a_n)$とする。
        \item 上より、
        $(a_0,\dots,a_n)\in C_V$となる。
        \item
        iより、$f$は$C_V$で消える関数なので、
        $(a_0,\dots,a_n)$を消す。
        \item
        iiiおわり:
        $f$は$p$の斉次座標をすべて消す。
        \item
        iiおわり: $f$は$V$を消す。
        \item
        上より、$f\in \ideal(V)$である。
        \item
        iおわり: $\ideal(C_V)\subset \ideal(V)$である。
      \end{enumerate}
      よって、 $\ideal(C_V)\subset \ideal(V)$である。
      \item $\supset$を示す。
      \begin{enumerate}
        \item $\forall f$: $f\in \ideal(V)$とする。
        \item
        $\forall a_0,\dots,a_n$: $(a_0,\dots,a_n) \in C_V-\zeroset$とする。
        \item
        上より、$(a_0:\dots:a_n) \in \ideal(V)$である。
        \item
        上と(b)より、$f$は$(a_0:\dots:a_n) \in V$を消す。
        \item
        (b)おわり: 上より$f$は$C_V-\zeroset$を消す。
        \item
          (a)と$\ideal(V)$は斉次イデアルなので、
          $f$の斉次成分はまた$\ideal(V)$に属し、$V$を消す。
        \item
        上より、$f$の定数項も$V$を消す。
        \item
        上と$V\neq \emptyset$より、$f$の定数項は0である。
        \item
        上より、$f$は原点$0$を消す。
        \item
        上と(e)より、$f$は$C_V$を消す。
        \item
        (a)おわり: 上より$\ideal(V) \subset \ideal(C_V)$となる。
      \end{enumerate}
      \item
      (a),(b)より、$\ideal(C_V) = \ideal(V)$となる。
    \end{enumerate}
    \item
    アフィン幾何の強形の零点定理より、$\sqrt{I} = \ideal_a(\var_a(I))$となる。
    \item
    \begin{align}
      \sqrt{I}
      \desceq{5,零点定理}
      \ideal_a(\var_a(I))
      =
      \ideal_a(C_V)
      \desceq{4}
      \ideal(V)
      =
      \ideal(\var(I)).
    \end{align}

  \end{enumerate}
\end{myproof}

\begin{framed}
  演習問題9(結局使わんかった。):
  \begin{enumerate}[label=(\alph*)]
    \item $k[x_0,\dots,x_n]$の任意の斉次イデアルで真部分集合
    になっているようなものは、$I_0$に含まれることぉお示せ。
    \item
    $r$次の羃$I_0^r$は$k[x_0,\dots,x_n]$の全次数が$r$
    の単項式全体から生成されることを示せ。
    さらに、このことから全次数が$r$以上であるような任意の斉次多項式は
    $I_0^r$に含まれていることを示せ。
    \item
    $V=\var(I_0)\subset \P^n(k), C_V = \var_a(I_0) \subset k^{n+1}$
    とおく。$\ideal_a(C_V)\neq \ideal(V)$であることを示せ。
  \end{enumerate}
\end{framed}
\begin{myproof}
  \begin{enumerate}[label=(\alph*)]
    \item
    斉次イデアル$I\subsetneq k[x_0,\dots,x_n]$とする。
    仮に$I$が定数を含んでいるならば$I$は全体になってしまうので、
    $I$は定数を含まない。$f\in I$とする。先のことより、$f$は定数ではない。
    $f$は斉次なので、これは$f$が定数項を含まないことを意味する。
    したがって、$f\in I_0$であり、$I\subset I_0$である。
    \item
    $I_0^r = \gen{k[x_0,\dots,x_n]の全次数rな単項式}$を示す。
    これは、$I_0=\gen{x_0,\dots,x_n}$なので、イデアルの積の生成元として
    イデアルの生成元の積たち全体が取れることから明らか。

    全次数が$r$以上であるような任意の斉次多項式が$I_0^r$に属することは、
    そのような多項式の各項の出鱈目な$r$次の因子を取れば、
    それが$I_0^r$に属することからわかる。

    \item
    \begin{align}
      \ideal_a(C_V)
      &=
      \ideal_a(\var_a(I_0))\\
      &=
      \gen{x_0,\dots,x_n}.
    \end{align}
    一方、
    \begin{align}
      \ideal(V)
      &=
      \ideal(\var(I_0))\\
      &=
      \ideal(\var(\gen{x_0,\dots,x_n}))\\
      &=
      \ideal(\emptyset)\\
      &=
      k[x_0,\dots,x_n].
    \end{align}
  \end{enumerate}
\end{myproof}

\begin{framed}
  定理10:
  $k$を代数的閉体とする。$\ideal$と$\var$は、空でない射影多様体と、
  $\gen{x_0,\dots,x_n}$に含まれる根基斉次イデアルとの間の、
  包含関係を逆転するような全単射写像を与える。つまり写像
  \begin{align}
    \set{空でない射影多様体}
  &  \xrightarrow{\ideal}
    \set{\gen{x_0,\dots,x_n}に真に含まれる根基斉次イデアル}\\
    \set{\gen{x_0,\dots,x_n}に真に含まれる根基斉次イデアル}
    &\xrightarrow{\var}
    \set{空でない射影多様体}
  \end{align}
  は互いに逆写像を与えている。
\end{framed}
\begin{myproof}
  \begin{enumerate}
    \item $I$を根基斉次イデアルとしたとき、
    $\var(I)\neq \emptyset \iff I \subsetneq \gen{x_0,\dots,x_n}$?
    \begin{enumerate}
      \item 定理8より、$\var(I)=\emptyset \iff \Exists{r\ge 1}\gen{x_0,\dots,x_n}^r \subset I$
      \item 1の仮定より$I$は根基イデアルなので、
      \begin{align}
        \Exists{r\ge 1}\gen{x_0,\dots,x_n}^r \subset I
        \iff
        \gen{x_0,\dots,x_n}\subset I.
      \end{align}
      \item
      (a)(b)より、
      \begin{align}
        \var(I) = \emptyset \iff \gen{x_0,\dots,x_n} \subset I.
      \end{align}
      \item 上の対偶をとり、
      \begin{align}
        \var(I) \neq \emptyset \iff I \subsetneq \gen{x_0\dots,x_n} .
      \end{align}
    \end{enumerate}
    \item
    $I$を$I\subsetneq \gen{x_0,\dots,x_n}$となる根基斉次イデアルとして、$\ideal(\var(I)) = I$?
    \begin{enumerate}
      \item $I\subsetneq \gen{x_0,\dots,x_n}$で、根基斉次イデアルなので、1より
      $\var(I)\neq \emptyset$である。
      \item 上と、$I$が斉次であることから定理9より、$\ideal(\var(I)) = \sqrt{I}$となる。
      \item $I$は根基なので、$\sqrt{I} = I$である。
      \item 上と(b)より、$\ideal(\var(I)) = I$である。
    \end{enumerate}
    \item
    $V$を空でない射影多様体として、$\var(\ideal(V)) = V$?
    \begin{enumerate}
      \item 定理5と、$k$が代数的閉体ゆえ無限体であることから$\var(\ideal(V)) = V$である。
    \end{enumerate}
    \item
    2,3より$\ideal$と$\var$は互いに逆写像である。
  \end{enumerate}
\end{myproof}


\subsection{アフィン多様体の射影完備化}
\label{sub:アフィン多様体の射影完備化}
\begin{framed}
  定義1:
  イデアル$I\subset k[x_1,\dots,x_n]$に対して、$I$の斉次化を
  次のように定義する。
  \begin{align}
    I^h = \gen{f^h; f\in I}\subset k[x_0,\dots,x_n]
  \end{align}
  ここで$f^h$は先の斉次化である。
\end{framed}

\begin{framed}
  命題2:
  任意のイデアル$I\subset k[x_1,\dots,x_n]$に対して、その斉次化
  $I^h$は$k[x_0,\dots,x_n]$の斉次イデアルである。
\end{framed}
\begin{myproof}
  $g\in \gen{f^h; f\in I}$とする。
  $N,F_\bullet$を使って、$g=\sum_{i=1}^N F_i f_i^h$とする。
  $f_\bullet$の番号を付け替えて、$f_{ij}$が、$i$が斉次の次数、
  $j$がそのうちでのインデックスになるようにする。
  $g=\sum_{i=}^N \sum_j F_{ij} f_{ij}^h$となる。
  さらに、$F_{ij}$を斉次分解して、$F_{ij} = \sum_l F_{ijl}$とする。
  $g=\sum_{i=1}^N \sum_j \sum_l F_{ijl} f_{ij}^h$となる。
  これを次数で分けて書くと、(存在しない添字の項は0として、)
  \begin{align}
    g = \sum_{d}\sum_{i+l=d} \sum_j F_{ijl}f_{ij}^h
  \end{align}
  となる。このうち、$\sum_{i+l=d} \sum_j F_{ijl}f_{ij}^h$は
  $d$次斉次成分になっているが、これは$f_{ij}^h$の一次結合なので
  $I^h$に属する。
\end{myproof}

$\gen{f_1^h,\dots,f_s^h}$は斉次な基底でできているので斉次イデアルだが、
上の斉次化はこれよりも大きくなりうる。

\begin{framed}
  次数つきの単項式順序:
  単項式順序のうち、$\myabs{\alpha}  > \myabs{\beta}$なら$x^\alpha > x^\beta$となるもの。
\end{framed}

\begin{framed}
  定理4:
  $I$を$k[x_1,\dots,x_n]$のイデアル、
  $G=\set{g_1,\dots,g_t}$を$k[x_1,\dots,x_n]$の
  次数付き単項式順序に関する$I$のグレブナ基底とする。
  すると$G^h = \set{g_1^h, \dots, g_t^h}$は
  $I^h\subset k[x_0,\dots,x_n]$の基底である。
\end{framed}
\begin{myproof}
  「$G^h$は$k[x_0,\dots,x_n]$の適当な単項式順序について
  $I^h$の\warn{グレブナ基底}であることを示す。」
  \begin{enumerate}
    \item 記号を用意する。$k[x_0,\dots,x_n]$の単項式は、
    $\alpha \in \Zge^n$と$d\in \Zge$を使って、
    \begin{align}
      x_1^{\alpha_1} \dots x_n^{\alpha_n}
      =
      x^\alpha x_0^d
    \end{align}
    と書く。
    \item
    $>_h$: $k[x_0,\dots,x_n]$の順序$>_h$を
    \begin{align}
      x^\alpha x_0^d >_h x^\beta x_0^e
      \iff
      \begin{cases}
        x^\alpha > x^\beta  または\\
        x^\alpha = x^\beta であって、かつd>eが成り立つ。
      \end{cases}
    \end{align}
    と定める。これは単項式順序になっている。
    \item
    任意の$i\ge 1$について、$x_i >_h x_0$が成立する。
    \item
    任意の$f\in k[x_1,\dots,x_n]$について、
    $\LM_{>_h}(f^h) = LM_{>}(f)$?
    \begin{enumerate}
      \item $\forall f$: $f\in k[x_1,\dots,x_n]$
      \item $\alpha$: $x^\alpha = \LM_>(f)$
      \item 上より$x^\alpha$は$f$の最高全次数の斉次部分の単項式である。
      \item 斉次化の定義より、$f^h$の単項式たちにも$x^\alpha$は存在する。
      \item $\forall \beta,e$: $x^\beta x_0^e$を$f^h$にあらわれる他の多項式とする。
      \item $x^\beta$が$f$の$x^\alpha$でない単項式なので、(b)より、$\alpha > \beta$となる。
      \item
      上より、$x^\alpha >_h x^\beta x_0^e$である。
      \item
      (e)おわり: 上より、$x^\alpha = \LM_{>_h}(f^h)$である。
      \item (b)と上より、任意の$f$について、$\LM_>(f) = \LM_{>_h}(f^h)$である。
    \end{enumerate}
    任意の$f\in k[x_1,\dots,x_n]$について、
    $\LM_{>_h}(f^h) = \LM_{>}(f)$である。
    \item 任意の$i$について、
    $I^h$の定義と、$g_i \in G \subset I$より$g_i^h \in I^h$である。
    \item
    上より、$G^h \subset I^h$である。
    \item
    $\gen{\LT_{>_h}(I^h)}$は$\LT_{>_h}(G^h)$で生成される?
    \begin{enumerate}
      \item $\forall F$: $F\in I^h$とする。
      \item
      $I^h$は斉次イデアルなので$F$の各斉次成分は$I^h$に属する。
      \item
      上より、(a)でとった$F$は斉次であると仮定してよい(生成を示したくて、$F$を$\gen{\LT_{>_h}(G^h)}$で書けることさえ言えればいい。)。
      \item $\exists A_j,f_j$:
      (a)より、
      \begin{align}
        F= \sum_j A_j f_j^h
      \end{align}
      と$A_j \in k[x_0,\dots,x_n]$と$f_j \in I$を用いて書ける。
      \item
      $f$: $f$を$F$の非斉次化とする。すなわち:
      $f=F(1,x_1,\dots,x_n)$とする。
      \item
      (d)(e)と命題2-7(iii)の「斉次化の頭に1を入れると戻る」より、
      \begin{align}
        f
        &=
        F(1,x_1,\dots,x_n)\\
        &=
        \sum_j A_j(1,x_1,\dots,x_n)f_j^h(1,x_1,\dots,x_n)\\
        &=
        \sum_j A_j(1,x_1,\dots,x_n)f_j.
      \end{align}
      \item
      (d)で$f_j \in I$としたことを言ったので上より、
      $f\in I \subset k[x_1,\dots,x_n]$となる。
      \item $\exists e$:
      (c)で$F$は斉次としておいたことと、
      命題2-7(iv)の「斉次多項式$F$を$x_0$で$e$回まで割れるなら、
      $f=F(1,x_1,\dots,x_n)$として、$F=x_0^e\cdot f^h$となる
      \footnote{まじ?と思ったが、非斉次化では次数は落ち、
      斉次化では次数は変わらないので、非斉次化→斉次化だと次数は落ちており、$x_0^e$を補わないとまずい。}
      」より、
      \begin{align}
        F= x_0^e\cdot  f^h
      \end{align}
      となる$e \in \Zge$がある。
      \item
      4より、
      \begin{align}
        \LM_{>_h}(F)
        \desceq{(h)}
        x_0^e \cdot \LM_{>_h}(f^h)
        \desceq{4}
        x_0^e \cdot \LM_{>}(f).
      \end{align}
      \item $\exists i$:
      $G$は$I$のグレブナ基底であることと、
      (g)で$f\in I$であることより、
      $\LM_>(f)$はある$\LM_>(g_i)$で割り切れる。
      \item 上と、4の$\LM_>(g_i) = \LM_{>_h}(g_i^h)$
      より、$\LM_{>}(F)$は$\LM_{>_h}(g_i^h)$で割り切れる。
      \item (a)おわり: 任意の$\LM_>(F) \in \LT_{>_h}(I^h)$は
      $\LM_{>_h}(g_i^h)$の倍数になっている。示された。
    \end{enumerate}
    $\gen{\LT_{>_h}(I^h)}$は$\LT_{>_h}(G^h)$で生成される。
    \item
    6の$G^h \subset I^h$と7の$\gen{\LT_{>_h}(I^h)}$が$\LT_{>_h}(G^h)$で生成されることより、
    $G^h$は$I^h$のグレブナ基底。
  \end{enumerate}
\end{myproof}

\begin{framed}
  定義6:
  アフィン多様体$W\subset k^n$に対して、$W$の射影完備化とは、
  射影多様体$\overline{W} = \var(\ideal_a(W)^h)\subset \P^n(k)$
  のことである。
\end{framed}
$W=\zeroset$のときは$\overline W = \emptyset$になってしまう。

$f_\bullet \in k[x_1,\dots,x_n]$は$\var(f_1,\dots,f_s)$は
アフィン多様体$\subset k^n$とも見えるし、
射影多様体$\subset \P^n(k) = k^{n+1}/\sim$とも見える。
また、射影多様体$\subset \P(k^n) = k^{n+1}/\sim$を
$x_0=1$とした
アフィン多様体$k^{n+1}$と同一視することがある。
\begin{framed}
  命題7:

  $W\subset k^n$をアフィン多様体とし、$\overline W \subset \P^n(k)$
  をその射影完備化とする。  \warn{(独自)また、$W\neq \zeroset$とする。←そんなことはなかった。$k^n$と$k^{n+1}$で考えてることに注意。}
  すると、次が成り立つ。
  \begin{enumerate}[label=(\roman*)]
    \item $\overline W \cap U_0 = \overline W \cap k^n = W$。(アフィン多様体とみて)
    \item
    $\overline W$は$W$を含むような$\P^n(k)$における最小の射影多様体である。
    \item
    アフィン多様体$W$が既約ならば、射影多様体$\overline W$もまた既約である。
    \item
    $\overline W$のどの既約成分も無限遠超平面$\var(x_0) \subset \P^n(k)$に完全に含まれることはない。
  \end{enumerate}
\end{framed}
\begin{myproof}
  (i)を示す。
  \begin{enumerate}
    \item $G$: $k[x_1,\dots,x_n]$の次数付き順序に関する
    $\ideal_a(W)$のグレブナ基底とする。
    \item
    定理4と上より、$\ideal_a(W)^h = \gen{g^h; g\in G}$である。
    \item
    $k^n$のサブセットと見て、2より
    \begin{align}
      \overline W \cap U_0
      &=
      \var(\ideal_a(W)^h) \cap U_0\\
      &\desceq{2}
      \var(g^h; g\in G)\cap U_0\\
      &\desceq{同一視}
      \var_a(g^h; g\in G) \cap \var_a(x_0=1)\\
      &=
      \var_a(g^h(1,x_1,\dots,x_n); g\in G).
    \end{align}
    \item
    命題2-7(iii)の「斉次化して$x_0=1$にすると戻る」より、
    $g^h(1,x_1,\dots,x_n) = g$となる。
    \item
    3,4と1より、
    \begin{align}
      \overline W \cap U_0 \desceq{3} \var_a(g^h(1,x_1,\dots,x_n); g\in G) \desceq{4} \var_a(g; g\in G) = \var_a(G) \desceq{1} \var_a(\ideal_a(W)) = W.
    \end{align}
  \end{enumerate}

  (ii)を示す。
  \begin{enumerate}
    \item $\forall V$: $\ub{W}_{\subset k^n} \subset V$なる射影多様体。($V$は一旦$k^{n+1}$と見做すが、射影多様体の条件を満たすものとする。)($\overline W\subset V$?)
    \item $F_1,\dots,F_s$:
    $V=\var(F_1,\dots,F_s)$とする。
    \item $f_1,\dots,f_s$:
    $f_i$は$F_i$の非斉次化$f_i = F_i(1,x_1,\dots,x_n)$とする。
    \item 2より各$F_i$は$V$を消す。
    \item 上と1より$F_i$は$W$も消す。
    \item 上と各$F_i$が射影多様体$V$の定義方程式であることと、
    3で各$f_i$が$F_i$の非斉次化であることから、
    各$f_i$は$W$を消す。
    \item
    上より、各$i$について、$f_i \in \ideal_a(W)$となる。
    \item
    上より、各$i$について、$f_i^h \in \ideal_a(W)^h$となる。
    \item
    上より、各$i$について、$f_i^h$は$\var(\ideal_a(W)^h) = \overline W$を消す。
    \item
    $\exists e_1,\dots,e_s$:
    命題2-7(iv)より、各$i$についてある整数$e_i$があって$F_i=x_0^{e_i}f_i^h$となる。
    \item
    上と9より、各$F_i$は$\overline W$を消す。
    \item
    上と2より、$\overline W \subset \var(F_1,\dots,F_s) = V$となる。
    \item 1おわり:
    任意の$W\subset V$なる任意の射影多様体$V$について$\overline W\subset V$なので、
    $\overline W$は$W$を包む最小の射影多様体になる。
  \end{enumerate}

  (iii)を示す。
  $W$が既約$\iff $ $\overline W$が既約なので、これを示す。
  対偶を示す。
  \begin{enumerate}
    \item $W$が既約でないなら$\overline W$が既約でないことを示す。
    $W=W_1 \cup W_2$と、空でないアフィン多様体$W_1,W_2 \subset k^n$に分解できるとする。
    \begin{align}
      \overline W
      &=
      \var(\ideal_a(W)^h)\\
      &=
      \var(\ideal_a(W_1 \cup W_2)^h)\\
      &=
      \var((\ideal_a(W_1) \cap \ideal_a(W_2))^h)\\
      &=
      \var(\ideal_a(W_1)^h \cap \ideal_a(W_2)^h)\\
      &=
      \var(\ideal_a(W_1)^h) \cup \var(\ideal_a(W_2)^h)\\
      &=
      \overline W_1 \cup \overline W_2.
    \end{align}
    (ii)より、$W_1 \subset \overline W_1$かつ$W_2 \subset \overline W_2$
    であり、$W_1,W_2 \neq \emptyset$なので、$\overline W_1,\overline W_2 \neq \emptyset$
    であり、$\overline W$は既約ではない。
    \item
     $\overline W$が既約でないなら$W$が既約でないことを示す。
     $\overline W = V_1 \cup V_2$と空でない既約多様体$V_1,V_2 \subset \P^n(k)$
     に分解されたとする。すると、(i)より
     \begin{align}
       W = \overline W \cap U_0 = (V_1 \cap U_0) \cup (V_2 \cap U_0)
     \end{align}
     となる。$V_1,V_2$が空でないので、$V_1 \cap U_0,V_2 \cap U_0$は
     空でないアフィン多様体であり、$W$は既約でない。
  \end{enumerate}

  (iv)を示す:
  \begin{enumerate}
    \item $V_\bullet$: $\overline W = V_1 \cup \dots \cup V_m$を既約な分解とする。さらに、
    余計なものは除かれているとする。
    \item どれか既約成分が無限遠超平面$\var(x_0)$に含まれているとする(背理法)。
    それが$V_1$であるとして一般性を失なわない。
    \item
    上の$V_1 \subset \var(x_0)$なので、$V_1 \cap \var(x_0) = \emptyset$である。
    \item
    \begin{align}
      W
      &\desceq{(i)}
      \overline W \cap U_0\\
      &\desceq{1}
      (V_1\cup \dots \cup V_m)\cap U_0\\
      &=
      (V_1 \cap U_0)\cup ((V_2 \cup \dots \cup V_m)\cap U_0)\\
      &\desceq{3}
      (V_2 \cup \dots \cup V_m)\cap U_0.
    \end{align}

    \item
    上より、$W \subset V_2\cup \dots \cup V_m$となる。
    \item
    上と射影完備化の定義より、$\overline W \subset V_2\cup \dots \cup V_m$となる。
    \item
    1より$V_2 \cup \dots \cup V_m\subset \overline W$である。
    \item
    7,8より、$V_2 \cup \dots \cup V_m = \overline W$である。
    \item
    1と上より、$V_1 \subset V_2 \cup \dots \cup V_m$である。
    \item
    上で$V_1$との$\cap$をとり、
    \begin{align}
      V_1 = (V_2 \cup \dots \cup V_m) \cap V_1
      =
      (V_2 \cap V_1)\cup \dots \cup(V_m \cap V_1).
    \end{align}
    \item $\exists i$:
    上と1の$V_1$の既約性より、ある$i$について $V_1 = V_1 \cap V_i$である。
    \item
    上より、$V_1 = V_1 \cap V_i \subset V_i$となる。
    \item 2おわり:
    上は、1で余計なものを除いたことに矛盾する。
    よって、どの$\overline W$の既約成分も無限遠超平面$\var(x_0)$に含まれることはない。
  \end{enumerate}

\end{myproof}

\begin{framed}
  定理8:
  $k$を代数的閉体とし、$I\subset k[x_1,\dots,x_n]$を任意の\footnote{これまでは斉次だった。}イデアルとする。
  すると$\var(I^h) \subset \P^n(k)$は$\var_a(I) \subset k^n$の射影完備化である。
\end{framed}
\begin{myproof}
  \begin{enumerate}
    \item $W$: $W=\var_a(I)$とする。
    \item $Z$:
    $Z=\var(I^h) \subset \P^n(k)$とする。
    \item 命題7(i)と$k$が代数的閉体であることより、
    \begin{align}
      W
      &\desceq{命題7(i)}
      \overline W \cap V_0\\
      &=
      \var(\ideal_a(W)^h)\cap V_0\\
      &=
      \var(\ideal_a(\var_a(I))^h)\cap V_0\\
      &\subset
      \var(\ideal_a(\var_a(I))^h)\\
      &=
      \var(\sqrt{I}^h)\\
      &\descsubset{☆}
      \var(I^h)\\
      &=
      Z.
    \end{align}
    よって、$W\subset Z$となっている。
    \item (最小性を示す)
    \item
    $\forall V, F_1,\dots,F_s$: $V=\var(F_1,\dots,F_s)$を$W\subset V$なる射影多様体とする。
    \item
    上より、$f_i = F_i(1,x_1,\dots,x_n)$と非斉次化として、
    \begin{align}
      W &\subset
      U(F_1,\dots,F_s)\cap U_0\\
      &=
      \var(F_1,\dots,F_s,x_0-1)\\
      &=
      \var(f_1,\dots,f_s).
    \end{align}
    \item
    上より、各$i$について$f_i \in \ideal_a(W)$となる。
    \item
    上と1と、$k$が代数的閉体であることから$f_i \in \ideal_a(W) = \sqrt{I}$である。
    \item
    $\forall i$:
    \item
    8より、$f_i \in \sqrt{I}$となる。
    \item
    $\exists m$:
    $m\ge 1$が存在して、$f_i^m \in I$となる。
    \item
    上より、$(f_i^m)^h \in I^h$となる。
    \item
    上と2より、$(f_i^m)^h$は$Z$を消す。
    \item
    $(f_i^m)^h = (f_i^h)^m$?
    \begin{enumerate}
      \item $f,g\in k[x_1,\dots,x_n]$として、$(fg)^h = f^h g^h$?
      \begin{enumerate}
        \item $f,g$の全次数を$d,e$としておく。
        \item  命題2-7(ii)より、
        \begin{align}
          (fg)^h
          &=
          x_0^{d+e} \cdot (fg)(\frac{x_1}{x_0},\dots,\frac{x_n}{x_0})\\
          &=
          (x_0^d \cdot f(\frac{x_1}{x_0},\dots,\frac{x_n}{x_0}))\cdot
          (x_0^e \cdot g(\frac{x_1}{x_0},\dots,\frac{x_n}{x_0}))\\
          &=
          f^h\cdot g^h.
        \end{align}
      \end{enumerate}
      よって、$(fg)^h = f^h \cdot g^h$となる。
      \item
      上よりあきらか。
      よって、$(f_i^m)^h = (f_i^h)^m$となる。
    \end{enumerate}
    \item
    13と14より、$(f_i^h)^m$は$Z$を消す。
    \item
    上より、$f_i^h$は$Z$を消す。
    \item
    命題2-7(iv)より、$F_i$は$f_i^h$の倍数である。
    \item
    上と16より、$F_i$は$Z$を消す。
    \item
    9おわり: 上より、$Z \subset \var(F_1,\dots,F_s) = V$である。
    \item
    $Z$は$W$を含む最小の射影多様体である。
    \item
    命題7より、$\overline W$も$W$を含む最小の射影多様体なので、
    上より$\overline W = Z$である。
  \end{enumerate}
\end{myproof}

\subsection{射影的消去理論}
\label{sub:射影的消去理論}

以下、
\begin{align}
  (x_1,\dots,x_n,y_1,\dots,y_m)\mapsto
  (1,x_1,\dots,x_n,y_1,\dots,y_m)
\end{align}
によって、$k^n\times k^m$を$\P^n\times k^m$の$x_0\neq 0$
で決まる部分集合と同一視する。

\begin{framed}
  定義2:
  $k$を体とする。
  \begin{enumerate}[label=(\roman*)]
    \item 多項式$F\in k[x_0,\dots,x_n,y_1,\dots,y_m]$は、
    ある整数$l\ge 0$に対して
    \begin{align}
      F=\sum_{\myabs{\alpha}=l}h_\alpha(y_1,\dots,y_m)x^\alpha
    \end{align}
    と表されるとき、$(x_0,\dots,x_n)$斉次であるという。
    ただし上の式で、$x^\alpha$は$x_0,\dots,x_n$の多重次数
    $\alpha$の単項式、また$h_\alpha \in k[y_1,\dots,y_m]$である。
    \item
    $(x_0,\dots,x_n)$斉次多項式$F_1,\dots,F_s \in k[x_0,\dots,x_n,y_1,\dots,y_m]$によって
    定義された多様体$\var(F_1,\dots,F_s) \subset \P^n\times k^m$とは、
    次の集合
    \begin{align}
      \set{(a_0,\dots,a_n,b_1,\dots,b_m)\in \P^n\times k^m;
      F_i(a_0,\dots,a_n,b_1,\dots,b_m)= 0 \quad (1\le i \le s)}
    \end{align}
    のことである。
  \end{enumerate}
\end{framed}
($\P^n \times k^m$の多様体を定義するために斉次多項式の概念を拡張した。)
(ii)のほうはwell-definedである。


\begin{framed}
  定義4:
  $(x_0,\dots,x_n)$斉次多項式で生成されたイデアル
  $I\subset k[x_0,\dots,x_n,y_1,\dots,y_m]$が与えられたとき、
  $I$の射影的消去イデアルとは集合
  \begin{align}
    \hat I =
    \set{f\in k[y_1,\dots,y_m];
    \Forall{0\le i \le n}\Exists{e_i\ge 0} x_i^{e_i}f \in I}.
  \end{align}
  のことである。
\end{framed}
$k[y_1,\dots,y_m]$のイデアルになっていることはあきらか。

\begin{framed}
  命題5:
  $V=\var(F_1,\dots,F_s) \subset \P^n \times k^m$を
  $(x_0,\dots,x_n)$斉次多項式で定義された多様体とし、
  $\pi\colon \P^n \times k^m \to k^m$を射影とする。
  このとき、$\hat I$を$I=\gen{F_1,\dots,F_s}$の
  射影的消去イデアルとすれば、$k^m$の中で
  \begin{align}
    \pi(V)\subset \var(\hat I)
  \end{align}
  が成り立つ。
\end{framed}
\begin{myproof}
  \begin{enumerate}
    \item $\forall a_\bullet,b_\bullet$: $(a_0,\dots,a_n,b_1,\dots,b_m) \in V$とする。
    \item
    $\forall f$: $f\in \hat I$
    \item $\forall i$:
    \item $\exists e_i$:
    2と射影的消去イデアルの定義により、$ x_i^{e_i}f(y_1,\dots,y_m) \in I$となる
    $e_i \ge 0$が存在する。
    \item 上より、
    $x_i^{e_i}f(y_1,\dots,y_m)$は$V$を消す。
    \item
    3おわり: 上より、
    任意の$i$について$a_i^{e_i} f(b_1,\dots,b_m) = 0$となる。
    \item $\exists i$:
    1より、$(a_0,\dots,a_n)$は斉次座標なので$a_i \neq 0$となる$i$がある。
    \item
    上と4より、$f(b_1,\dots,b_m) = 0$となる。
    \item
    1おわり: 上より、$f$は$\pi(V)$を消す。
    \item 2と上より、
    $\pi(V) \subset \var(\hat I)$となる。
  \end{enumerate}
\end{myproof}

\begin{framed}
  定理6(射影化された拡張定理):
  $k$を代数的閉体とし、$V=\var(F_1,\dots,F_s) \subset \P^n \times k^m$を
  $k[x_0,\dots,x_n,y_1,\dots,y_m]$の$(x_0,\dots,x_n)$
  斉次多項式で定義された多様体とする。$I=\gen{F_1,\dots,F_s}$の射影的消去イデアルを、
  $\hat I \subset k[y_1,\dots,y_m]$で表す。また
  \begin{align}
    \pi\colon \P^n \times k^m \to k^m
  \end{align}
  を後半の$m$個への座標空間への射影とするならば、
  \begin{align}
    \pi(V) = \var(\hat I)
  \end{align}
  が成り立つ。
\end{framed}
\begin{myproof}
   命題5より$\pi(V)\subset \var(\hat I)$は成立しているので、
   $\var(\hat I) \subset \pi(V)$を示したい。
   \begin{enumerate}
     \item $\forall \bbold c, c_\bullet$:
     $\bbold c = (c_1,\dots,c_m) \in \var(\hat I)$とする。
     \item 任意の$i$について、$F(x_0,\dots,x_n,\bbold c)$は
     $x_0,\dots,x_n$についての斉次多項式である。
     \item $d_\bullet$:
      各$i$について、 $F_i(x_0,\dots,x_n,\bbold c)$の
      $x_0,\dots,x_n$の多項式としての全次数を$d_i$とする。
    \item $\bbold c \notin \pi(V)$とする。(背理法)
    \item
    上より、
    \begin{align}
      F_1(x_0,\dots,x_n,\bbold c) = \dots =
      F_s(x_0,\dots,x_n,\bbold c) = 0
    \end{align}
    は$\P^n$の空集合を定義する。(これを満足するような斉次座標の$x_\bullet$は存在しない。)
    ($\var(F_1(x_0,\dots,x_n,\bbold c),\dots, F_s(x_0,\dots,x_n,\bbold c)) = \emptyset$である。)
    \item $\exists r$:
    $k$は代数的閉体であることから、上の空集合であることに
    射影化された零点定理の弱形(定理3-8)を使って、
    ある$r\ge 0$が存在して
    \begin{align}
      \gen{x_0,\dots,x_n}^r
      \subset
      \gen{F_1(x_0,\dots,x_n,\bbold c),\dots,F_s(x_0,\dots,x_n,\bbold c)}.
    \end{align}
    \item $\forall \alpha$: $\myabs{\alpha} = r$となる多重次数とする。
    \item $H_\bullet$:
    上と6より、$\myabs{\alpha} = r$なる単項式$x^\alpha$は
    $F_i(x_0,\dots,x_n,\bbold c)$たちの多項式係数線形結合としてあらわされる:
    \begin{align}
      x^\alpha = \sum_{i=1}^s H_i(x_0,\dots,x_n) F_i(x_0,\dots,x_n,\bbold c).
    \end{align}
    となる$H_\bullet$がある。
    \item
    3($d_i$の次数),7($\alpha$の次数),8($x^\alpha$の表現)より、
    $H_i$は全次数$r-d_i$の斉次多項式として一般性を失なわない。($F_i(x_0,\dots,x_n,\bbold c)$はもともと斉次。)
    \item
    7おわり:
    各$H_i$を$\myabs{\beta_i} =r-d_i$なる$x^{\beta_i}$の$k$係数線形結合で書くと、
    8より、任意の$\myabs{\alpha} = r$なる$\alpha$について、
    \begin{align}
      x^\alpha = \sum_{i=1}^s \sum_{\myabs{\beta_i}=r-d_i} (なにか係数)x^{\beta_i} F_i(x_0,\dots,x_n,\bbold c)
    \end{align}
    となっている。
    \item
    上より、
    \begin{align}
      x^{\beta_i}F_i(x_0,\dots,x_n,\bbold c),\quad i=0,\dots,s,\quad \myabs{\beta_i} = r-d_i
    \end{align}
    が$x_0,\dots,x_n$に関する全次数$r$の斉次多項式すべての空間を張る。
    \item $N_r$:
    「$x_0,\dots,x_n$に関する全次数$r$の斉次多項式すべての空間」の次元を$N_r$とする。
    実際はこれは$n+1$文字から$r$文字選ぶ方法の数になっている。
    \item $G_\bullet$:
    11の基底たちを、12より$N_r$個選んでこれを「$x_0,\dots,x_n$に関する全次数$r$の斉次多項式すべての空間」の
    基底にすることができる。その基底を
    \begin{align}
      G_j(x_0,\dots,x_n,\bbold c),\quad j=1,\dots,N_r
    \end{align}
    とする($x^{\beta_i}$の部分は、これが$x_\bullet$の関数なので折り込んでよい。)。
    \item $a_{\bullet \bullet}$:
    13の$G_\bullet$の定義より、これは11のような$x_\bullet$について$r$次の斉次多項式式
    $x^{\beta_i} F_i(x_0,\dots,x_n,\bbold c)$たちでできているのだから、
    これを斉次分解することにより
    各$j=1,\dots,N_r$について
    \begin{align}
      G_j = \sum_{\myabs{\alpha}=r} a_{j\alpha}(y_1,\dots,y_m)x^{\alpha}
    \end{align}
    となる$a_{j\alpha} \in k$が存在する。
    \item
    上について、$\myabs{\alpha} = r$となる単項式の個数は12の定義より
    $N_r$だったので、$a_{j\alpha}$は$j$を固定すると$N_r$個ある。
    \item $D(y_1,\dots,y_m)$:
    上より、$a_\bullet$は正方形に並べられるので行列式
    \begin{align}
      D(y_1,\dots,y_m) =
      \det(a_{j\alpha}(y_1,\dots,y_m); 1\le j \le N_r,\, \myabs{\alpha}=r)
    \end{align}
    が考えられる。
    \item
    14の式で$(y_1,\dots,y_m)=\bbold c$と代入し$j$を走らせながら並べることにより、
    \begin{align}
    \ub{(G_1, \dots ,G_{N_r})(\bbold c)}_{N_r \times N_r}
    =
    \ub{(a_{j\alpha}; 1\le j\le N_r, \, \myabs{\alpha}=r)(y_1,\dots,y_m)}_{N_r \times N_r}
    \cdot
    \ub{(x_\alpha; \myabs{\alpha}=r)}_{N_r \times N_r}.
    \end{align}
    となる。ただし、左の行列と右の行列は多項式なので、それを項毎に
    $\alpha$についての適当な順序で並べたベクトルと見做している。
    \item
    上は、「$x_0,\dots,x_n$に関する全次数$r$の斉次多項式すべての空間」の基底
    の変換の式になっているので($G_\bullet$が基底なのは13による。$x_\bullet$は標準的な基底。)
    $(a_{j\alpha}; 1\le j\le N_r, \, \myabs{\alpha}=r)(y_1,\dots,y_m)$
    は正則である。
    \item
    上より、$D(\bbold c)\neq 0$である。
    \item
    $D(y_1,\dots,y_m) = 0$だとすると$D(\bbold c) = 0$となるので
    上に矛盾する。したがって、$D(y_1,\dots,y_m) \neq 0_{k[y_1,\dots,y_m]}$となる。
    \item
    17の式
    \begin{align}
      {(a_{j\alpha}; 1\le j\le N_r, \, \myabs{\alpha}=r)(y_1,\dots,y_m)}
      \cdot
      {(x_\alpha; \myabs{\alpha}=r)}
      =
      {(G_1, \dots ,G_{N_r})(\bbold c)}
    \end{align}
    は$x_\alpha$たちについての$k(y1,\dots,y_m)$上での
    \item $M_\alpha$:
    ${(a_{j\alpha}; 1\le j\le N_r, \, \myabs{\alpha}=r)(y_1,\dots,y_m)}$
    の第$\alpha$列を$\tatev{G_1 \\ \vdots \\ G_{N_r}}$に取り替えたものとする。
    \item 上と21とクラメルの公式と20の非零より、
    \begin{align}
      x^\alpha = \frac{\det(M_\alpha)}{D(y_1,\dots,y_m)}.
    \end{align}
    \item $H_{j\alpha}$:
    $H_{j\alpha}$を${(a_{j\alpha}; 1\le j\le N_r, \, \myabs{\alpha}=r)(y_1,\dots,y_m)}$
    から$j$列$\alpha$行を除いた小行列の行列式とする。($j$列$\alpha$行での展開のやつ。)
    \item 23で、$\det(M_\alpha)$を$\alpha$列目で展開することで、24も利用し
    \begin{align}
      x^\alpha D(y_1,\dots,y_m)
      =
      \sum_{j=1}^{N_r} H_{j\alpha}(y_1,\dots,y_m) G_j(x_0,\dots,x_n,y_1,\dots,y_m).
    \end{align}
    \item
    13の定義より、各$G_j$は$x^{\beta_i}F_i$の形をしていたので、
    \begin{align}
      x^\alpha D(y_1,\dots,y_m)
      \in
      \gen{F_1,\dots,F_s} = I.
    \end{align}
    \item
    上より、$D(y_1,\dots,y_m)\in \hat I$となる。
    \item 上より、$D(y_1,\dots,y_m)$は$\var(\hat I)$を消す。
    \item 1より$\bbold c \in \var(\hat I)$なので、上より、
    $D(y_1,\dots,y_m)$は$\bbold c$を消す: $D(\bbold c) = 0$となる。
    \item 上と19は矛盾する。
    \item 4おわり: $\bbold c \in \pi(V)$となる。
    \item 1おわり: $\var(\hat I) \subset \pi(V)$となる。
   \end{enumerate}
\end{myproof}

\begin{framed}
  命題7:
  $I\subset k[x_0,\dots,x_n,y_1,\dots,y_m]$をイデアルとする。
  十分大きなすべての整数$e$に対して、
  \begin{align}
    \hat I = (I: \gen{x_0^e,\dots,x_n^e})\cap k[y_1,\dots,y_m]
  \end{align}
  が成り立つ。
\end{framed}
\begin{myproof}
  \begin{enumerate}
    \item イデアル商の定義より、
    \begin{align}
      \Forall{0\le i \le n}
      f\in I:\gen{x_0^e,\dots,x_n^e}
      \implies
      x_i^e f \in I
    \end{align}
    である。
    \item
    上より、任意の$e\ge 0$に対して、
    \begin{align}
      (I:\gen{x_0^e,\dots,x_n^e}) \cap k[y_1,\dots,y_m] \subset \hat I
    \end{align}
    となる。($f\in \hat I \subset k[y_1,\dots,y_m]$のための条件は、$x_0,\dots,x_n$の適当な羃を$f$にかければ$I$に入るということだった。)
    \item
    \begin{align}
      I:\gen{x_0,\dots,x_n}
      \subset
      I:\gen{x_0^2,\dots,x_n^2}
      \subset
      \dots
    \end{align}
    という昇鎖がある。
    \item $\exists e$:
    上の昇鎖は安定するので、$e$以降安定するとする:
    \begin{align}
      I:\gen{x_0^e,\dots,x_n^e}
      =
      I:\gen{x_0^{e+1},\dots,x_n^{e+1}}=\dots
    \end{align}
    である。
    \item
    \begin{align}
      \Forall{d\ge 0} I:\gen{x_0^d,\dots,x_n^d}
      \subset
      I:\gen{x_0^e,\dots,x_n^e}.
    \end{align}
    \item $\forall f$:
    $f\in \hat I$とする。
    \item
    上より、
    \begin{align}
      \Forall{0\le i\le n}\Exists{e_i\ge 0}x_i^{e_i} f \in I.
    \end{align}
    \item $d$:
    $d = \max(e_0,\dots,e_n)$とする。
    \item
    上の定義と7より、$x_i^d f \in I$がすべての$i$で成立する。
    \item
    上より、$f\in I:\gen{x_0^d,\dots,x_n^d}$である。
    \item
    6で、$f\in \hat I \subset k[y_1,\dots,y_n]$と
    5より、
    \begin{align}
      f\in (I:\gen{x_0^e,\dots,x_n^e}) \cap k[y_1,\dots,y_m].
    \end{align}
    \item
    上より、
    \begin{align}
      \hat I \subset \gen{I:\gen{x_0^e,\dots,x_n^e}}\cap k[y_1,\dots,y_m].
    \end{align}
    \item 2,12より、
    \begin{align}
      \hat I = \gen{I:\gen{x_0^e,\dots,x_n^e}} \cap k[y_1,\dots,y_m].
    \end{align}
  \end{enumerate}
\end{myproof}


\begin{framed}
  定義:
  $F\in k[x_0,\dots,x_n,y_1,\dots,y_m]$において$x_i = 1$について、
  \begin{align}
    F\ovparen{i} =
    F(x_0,\dots,\banme{i}{1},\dots,x_n,y_1,\dots,y_m) \in
    k[x_0,\dots,\hat x_i,\dots,x_n,y_1,\dots,y_m].
  \end{align}

  イデアルの非斉次化を
  \begin{align}
    I\ovparen{i} = \set{F\ovparen{i}; F\in I}\subset
    k[x_0,\dots,\hat x_i,\dots,x_n,y_1,\dots,y_m].
  \end{align}
  とする。$I=\gen{F_1,\dots,F_s}$のときには、
  $I\ovparen{i} = \gen{F_1\ovparen{i},\dots,F_s\ovparen{i}}$が成立する。
\end{framed}

\begin{framed}
  命題8:
  $I\subset k[x_0,\dots,x_n,y_1,\dots,y_m]$を
  $(x_0,\dots,x_n)$斉次多項式で生成されたイデアルとする。すると、
  \begin{align}
    \hat I = I_n\ovparen{0}\cap I_n\ovparen{1} \cap \dots \cap I_n\ovparen{n}
  \end{align}
  が成り立つ。
\end{framed}
\begin{myproof}
  \begin{enumerate}
    \item $\hat I = I\ovparen{0}\cap \dots \cap I\ovparen{n}\cap k[y_1,\dots,y_m]$を示す。
    \item
    $\forall f$: $f\in \hat I$とする。
    \item $\forall i$:
    \item
    2より、$x_i^{e_i} f(y_1,\dots,y_m) \in I$
    \item
    上で$x_i = 1$として、$f(y_1,\dots,y_m)\in I\ovparen{i}$となる。
    \item
    3おわり: $f(y_1,\dots,y_m)\in I\ovparen{0}\cap \dots \cap I\ovparen{n}$
    \item
    2おわり: $\hat I \subset I\ovparen{0}\cap \dots \cap I\ovparen{n} \cap k[y_1,\dots,y_m]$となる。
    \item $\forall i$:
    \item $\forall f$:
    $f\in I\ovparen{i}$
    \item $\exists F$:
    $I\ovparen{i}$の定義より、$f$はある$F\in I$で$x_i = 1$とおくことで得られる。
    \item $F$は$(x_0,\dots,x_n)$斉次として一般性を失なわない?
    \begin{enumerate}
      \item  $F_\bullet$:
      $F$を$(x_0,\dots,x_n)$斉次に分解する:
      $F=\sum_{j=0}^d F_j$となる。
      \item $I$が$(x_0,\dots,x_n)$斉次多項式で生成されていることから、
      各$F_j \in I$となる。 これは定理3-2と同様に示せる。概略としては、
      $F\in I$から$F$を$I$の生成元の一次結合で書き、
      それを$x_\bullet$斉次なものたちで整理する。あとは、
      次数で$F=\sum_{j=0}^d F_j$と比較し、$F_j \in I$を得る
      \item
      $\sum_{j=0}^d x_i^{d-j}F_j$は$I$に含まれる
      $(x_0,\dots,x_n)$斉次多項式。(全次数は、$F_j$が(a)より$j$となるようにしておいたので、$x_i^{d-i}$をかけて$d$次。)
      \item
      上の$\sum_{j=0}^d x_i^{d-j}F_j$は
      $x_i=1$とすれば非斉次化され$f$に一致する。
      \item 上より、$F\in I$は$(x_0,\dots,x_n)$斉次と仮定してよい。
      もとの$F$から、斉次になるように作り直すことができた。
    \end{enumerate}
    \item 9おわり: $f\in I\ovparen{i}$について、$(x_0,\dots,x_n)$斉次多項式$F$が存在して、
    $f=F(x_1,\dots,\banme{i}{x_i},\dots,x_n,y_1,\dots,y_m)$となる。
    \item $\bullet^h$:
    記号を定める。$f\in k[x_0,\dots,\hat x_i, \dots, x_n, y_1,\dots,y_m]$に対して、
    $x_i$を用いて$(x_0,\dots,x_n)$斉次多項式$f^h \in k[x_0,\dots,x_n,y_1,\dots,y_m]$を
    作ることで斉次化作用素$\bullet^h$を定める。
    \item
    $(x_0,\dots,x_n)$ 斉次多項式$F$について、$f=F(x_0,\dots,\banme{i}{1},\dots,x_n,y_1,\dots,y_m)$と
    非斉次化するとき、ある整数$e\ge 0$について$F=x_i^e f^h$となる?
    \begin{enumerate}
      \item $i=0$として一般性を失なわない。
      \item $F=\sum_{\myabs{\alpha}=l}h_\alpha(y_1,\dots,y_n)x^\alpha$と分解する。
      \item さらに、$x_0$だけ特別扱いする。
      \begin{align}
        F = \sum_{j=0}^l \sum_{\myabs{\beta}=l-j} h_{(j,\beta)}(y_1,\dots,y_n) x^{(j,\beta)}
      \end{align}
      とする。
      \item
      \begin{align}
        f = \sum_{j=0}^l \sum_{\myabs{\beta}=l-j} h_{(j,\beta)}(y_1,\dots,y_n) x^{(0,\beta)}
      \end{align}
      \item
      $l'$を、$h_{(l',\beta)}(y_1,\dots,y_n) \neq 0$なる$\beta$があるような
      $l'$のうち最小のものとする。すると、
      \begin{align}
        f = \sum_{j=l'}^l \sum_{\myabs{\beta}=l-j} h_{(j,\beta)}(y_1,\dots,y_n) x^{(0,\beta)}
      \end{align}
      となり、$f$の$(x_1,\dots,x_n)$についての全次数は$l-l'$となる。
      \item 上より、
      \begin{align}
        f^h = \sum_{j=l'}^l \sum_{\myabs{\beta}=l-j} h_{(j,\beta)}(y_1,\dots,y_n) x^{(j-l',\beta)}
      \end{align}
      となる。全次数は変わらず$l-l'$である。
      \item (c),(f)より、
      \begin{align}
        F = f^h\cdot x_0^{l'}
      \end{align}
      となる。
    \end{enumerate}
    \item
    $\forall f$: $f\in I\ovparen{i} \cap k[y_1,\dots,y_m]$とする。
    \item $\exists F$:
    9-12より、$f=F(x_i=1)$と、非斉次化して$f$になる$(x_0,\dots,x_n)$斉次多項式 $F\in I$がある。
    \item
    15より、$f$は$x_0,\dots,x_n$を含まないので、
    ($(x_0,\dots,x_n)$に関する全次数が$-\infty$であり、)
    $f=f^h$である。
    \item $\exists e$:
    14より、
    $F=x_i^e f^h$となる$e\ge 0$が存在する。
    \item 上と17より、$F=x_i^e f$である。
    \item 上と16より、$x_i^e f \in I$である。
    \item 15おわり: 任意の$i$について、$e_i\ge 0$があって、
    $x_i^{e_i} f \in I$となる。
    \item 上より、$f \in \hat I$である。
    \item 15おわり:
    \begin{align}
      I\ovparen{0}\cap \dots \cap I\ovparen{n} \cap k[y_1,\dots,y_m] \subset \hat I.
    \end{align}
    \item
    7,23より、
    \begin{align}
      I\ovparen{0} \cap \dots \cap I\ovparen{n} \cap k[y_1,\dots,y_m] = \hat I.
    \end{align}
  \end{enumerate}
\end{myproof}

\begin{framed}
  定義:
  $I\subset k[x_0,\dots,x_n,y_1,\dots,y_m]$について、
  $I$の$(x_0,\dots,x_n)$斉次化$I^h$を
  \begin{align}
    I^h = \gen{f^h; f\in I}\subset
    k[x_0,\dots,x_n,y_1,\dots,y_m].
  \end{align}
\end{framed}

\begin{framed}
  命題9:
  イデアル$I \subset k[x_1,\dots,x_n,y_1,\dots,y_m] $の
  \xs 斉次化を$I^h$とする。
  \begin{enumerate}[label=(\roman*)]
    \item $I^h$の射影的消去イデアルは$I$の第$n$消去イデアルに一致する。
    つまり$\hat{I^h} = I_n \subset k[y_1,\dots,y_m]$が成り立つ。
    \item
    $k$が代数的閉体のとき、$\overline V= \var(I^h)$は、
    アフィン多様体$V  = \var_a(I) \subset k^n \times k^m$
    を含む$\P^n \times k^m$における最小の多様体である。
    $\overline V$を$V$の$\P^n\times k^m$における射影完備化とよぶ。
  \end{enumerate}
\end{framed}
\begin{myproof}
  (i)を示す。
  \begin{enumerate}
    \item $I^h$を$x_0$について非斉次化すれば、$(I^h)\ovparen{0} = I$
    となる($h$で$x_0$をつけたし、それを1にしたのでもとにもどる。)。
    \item
    命題8より(あるいはその証明より)
    \begin{align}
      \hat {I^h} = {(I^h)}_n\ovparen{0} \cap \dots \cap {(I^h)}_n\ovparen{n} \cap k[y_1,\dots,y_n]
    \end{align}
    である。
    \item 1と2より、
    \begin{align}
      \hat {I^h} \subset (I^h)_n\ovparen{0} = I_n.
    \end{align}
    \item $\forall f$:
    $\forall f \in I_n$
    \item
    $f\in k[y_1,\dots,y_n]$なので、$f$
    $(x_0,\dots,x_n)$斉次になっている。($-\infty$次。)
    \item
    上と4の$f\in I_n \subset I$より、$f=f^h \in I^h$となる。
    \item
    上より、
    任意の$i$について$x_i^0 f \in I^h$である。(1をかけただけ)
    \item
    上と射影的消去イデアルの定義より、$f\in \hat{I^h}$となる。
    \item 4おわり:
    $I_n \subset \hat{I^h}$となる。
    \item 3,9より、$I_n = \hat{I^h}$となる。
  \end{enumerate}

  (ii)を示す。
  略証。
  \begin{enumerate}
    \item (まず$\var_a(I)=V\subset \overline V = \var(I^h)$を示す。)
    \item
    \begin{align}
      V
      &=
      \var_a(I)\\
      &=
      \var_a(I^h, x_0-1)\\
      &\subset
      \var_a(I^h).
    \end{align}
    \item
    (次に、最小性を示す。)
    \item $\forall F_1,\dots,F_s$:
    $V=\var_a(I)\subset \var(F_1,\dots,F_s)$と、
    $F_1,\dots,F_s \in k[x_0,\dots,x_n,y_1,\dots,y_m]$
    となる\xs 斉次多項式を考える。
    \item $f_\bullet$:
    各$i$について、$f_i$を$F_i$の非斉次化とする。
    \item
    \begin{align}
      V
      &\subset
      \var(F_1,\dots,F_s) \cap (U_0\times k^m)\\
      &=
      \var(F_1,\dots,F_s,x_0-1)\\
      &=
      \var(f_1,\dots,f_s).
    \end{align}
    \item
    上より、$f_1,\dots,f_s \in \ideal_a(V) = \sqrt{I}$となる。
    \item $\exists m_i$:
    \begin{align}
      \Forall{i}\Exists{m_i \ge 0} f_i^{m_i}\in I.
    \end{align}
    \item
    上より、
    \begin{align}
      \Forall{i} \Exists{m_i \ge 0} (f_i^{m_i})^h \in I^h.
    \end{align}
    \item
    上で、羃と斉次化を交換して(多分できる)
    \begin{align}
      \Forall{i} \Exists{m_i \ge 0} (f_i^h)^{m_i} \in I^h.
    \end{align}
    \item
    上より、各$(f_i^h)^{m_i}$は$\var(I^h)$を消す。
    \item
    上より、
    各$f_i^h$は$\var(I^h)$を消す。
    \item
    各$F_i$は$f_i^h$の倍数である。
    \item
    上と12より、各$F_i$は$\var(I^h)$を消す。
    \item
    上より、$\var(I^h) \subset \var(F_1,\dots,F_s)$である。
    \item 4おわり:
    4より、$\var_a(I)$を包む$\P^n\times k^m$の多様体は
    $\var(I^h)$を包む。
    \item
    2,16より、$\var(I^h)$は$\var_a(I)$を包む最小の$\P^n\times k^m$の多様体である。
  \end{enumerate}
\end{myproof}

\begin{framed}
  系10:
  $k$を代数的閉体とし、イデアル$I\subset k[x_1,\dots,x_n,y_1,\dots,y_m]$
  に対して$V=\var_a(I)\subset k^n \times k^m$とおく。すると
  \begin{align}
    \var(I_n) = \pi(\overline V)
  \end{align}
  が成り立つ。ただし$\overline V \subset \P^n \times k^m$
  は$V$の射影完備化であり、$\pi\colon \P^n \times k^m \to k^m$は射影である。
\end{framed}
\begin{myproof}
  \begin{enumerate}
    \item 命題9より、$\overline V = \var(I^h)$である。
    \item
    命題9より、$\hat{I^h} = I_n$である。
    \item
    定理6を多様体$\var(I^h)$、イデアル$I^h$に使うと、
    $\pi(\var(I^h)) = \var(\hat{I^h})$がわかる。
    \item
    \begin{align}
      \var(I_n)
      \desceq{2}
      \var(\hat{I^h})
      \desceq{3}
      \pi(\var(I^h))
      \desceq{1}
      \pi(\overline V).
    \end{align}
  \end{enumerate}
\end{myproof}

\begin{framed}
  命題11:
  $>$を$k[x_1,\dots,x_n,y_1,\dots,y_m]$の単項式順序であって、
  $x_1,\dots,x_n,y_1,\dots,y_m$の単項式
  $x^\alpha y^\gamma, x^\beta y^\delta $に対して
  \begin{align}
    \myabs{\alpha} > \myabs{\beta} \implies
    x^\alpha y^\gamma > x^\beta y^\delta
  \end{align}
  となるようなものとする。
  $G=\gen{g_1,\dots,g_s}$が$I\subset k[x_1,\dots,x_n,y_1,\dots,y_m]$の
  $>$に関するグレブナ基底であれば、$G^h = \set{g_1^h,\dots,g_s^h}$は
  $I^h \subset k[x_0,\dots,x_n,y_1,\dots,y_m]$の基底である。
\end{framed}
\begin{myproof}
  Pending.
\end{myproof}

先に演習問題を解く。
\begin{framed}
  演習16:
  \xs を$\P^n$の斉次座標、$(y_0,\dots,y_m)$を$\P^m$の斉次座標とする。
  \begin{enumerate}[label=(\alph*)]
    \item 「
    $h\in k[x_0,\dots,x_n,y_0,\dots,y_m]$は
    \begin{align}
      h=\sum_{\myabs{\alpha}=k,\myabs{\beta}=l}a_{\alpha \beta}x^\alpha y^\beta
    \end{align}
    と書けるとき、双斉次とよばれる。このとき、$h$は双次数$(k,l)$を持つという。
    $h_1,\dots,h_s$が双斉次のとき、多様体
    \begin{align}
      \var(h_1,\dots,h_s) \subset \P^n  \times \P^m
    \end{align}
    は矛盾なく定義できることを示せ。
    また$J\subset k[x_0,\dots,x_n,y_0,\dots,y_m]$が双斉次多項式で生成された
    イデアルであるとき、$\var(J)\subset \P^n \times \P^m$をどのように定義すればよいかを説明せよ。
    また、$\var(J)$は多様体であることを証明せよ。」
    明らか。
    \item
    「
    $J$が双斉次多項式で生成されていれば、$V=\var(J)\subset \P^n \times \P^m$が
    定義できた。$J$は$(x_0,\dots,x_n)$斉次でもあるから、
    その射影的消去イデアル$\hat J \subset k[y_0,\dots,y_m]$が構成できる。
    $\hat J$は斉次イデアルであることを証明せよ。」
    $(x_0,\dots,x_n)$斉次化を$I^h$とする。命題9より
    $\hat{J^h}=J_n$となる。また、$J$はすでに$(x_0,\dots,x_n)$斉次なので、
    $J^h = J$である。よって、$\hat J = J_n$となる。
    $J$は斉次イデアルになっているので(グレブナ基底を考えれば)
    $J_n$も斉次イデアルになっている。よって、$\hat J$は斉次イデアルである。
    \item
    $\var(J)$を$\P^n\times k^{m+1}$の多様体とみなし、定理6より
    $\pi(V) = \var(\hat J)$を得る。
  \end{enumerate}
\end{framed}

\begin{framed}
  定理12:
  $k$を代数的閉体とする。すべて同じ全次数を持ち、$\P^n$において
  共通零点を持たないような斉次多項式$f_0,\dots,f_m \in k[x_0,\dots,x_n]$
  によって定義される写像を$F\colon \P^n \to \P^m$とする。
  $k[x_0,\dots,x_n,y_0,\dots,y_m]$における
  イデアル$I=\gen{y_0-f_0,\dots,y_m-f_m}$を考え、
  $I_{n+1} = I\cap k[y_0,\dots,y_m]$とおく。すると$I_{n+1}$は
  $k[y_0,\dots,y_m]$の斉次イデアルであって、
  \begin{align}
    F(\P^n)  = \var(I_{n+1})
  \end{align}
  が成り立つ。
\end{framed}
\begin{myproof}
  \begin{enumerate}
    \item $I_{n+1}$は斉次イデアル?
    \item $d$: 各$f_i$の全次数(これらは仮定より一致する)を$d$とする。
    \item 各$x_i$にウェイト1を与え、$y_j$にウェイト$d$を与える。
    \item $f \in k[x_0,\dots,x_n,y_0,\dots,y_m]$がウェイト付き
    斉次多項式とは、$f$の単項式がすべて同じウェイトを持つ場合をいう。
    \item
    $I$の生成元$y_i-f_i$たちはすべてウェイト$d$を持つ。
    \item
    上より、$I$はウェイトつき斉次イデアルである。
    \item $G$:
    $G$を$I$のグレブナ基底とする。
    \item 定理3-2と同様の定理が示せて、上と6より$G$はウェイトつきの
    斉次多項式からなる。
    \item
    lex順序をとると、消去定理により、
    $G\cap k[y_0,\dots,y_m]$は$I_{n+1} = I\cap k[y_0,\dots,y_m]$の基底になる。
    \item
    8,9より、$I_{n+1}$はウェイト付き斉次多項式からなる基底を持つ。
    \item
    3より$y_i$たちは同じウェイトを持つ。
    \item
    上より、$k[y_0,\dots,y_m]$の多項式について、
    「これが斉次である$\iff$これがウェイト付き斉次である」となる。
    \item
    $h\in k[x_0,\dots,x_n,y_0,\dots,y_m]$はもしそれが
    \begin{align}
      h=\sum_{\myabs{\alpha}=k,\myabs{\beta}=l} a_{\alpha \beta} x^\alpha y^\beta
    \end{align}
    とかけていれば、双斉次であるという。
    このとき、$h_1,\dots,h_s$が双斉次なら、$\var(h_1,\dots,h_s) \subset \P^n \times \P^m$が定義され、多様体になる。
    $J\subset k[x_0,\dots,x_n,y_0,\dots,y_m]$が双斉次多項式で生成されたイデアルなら、多様体$\var(J) \subset \P^n\times \P^n$が定義される。
    \item
    $J$が双斉次多項式で生成されたイデアルなら、演習16より、$\hat J \subset k[y_0,\dots,y_m]$は斉次イデアルである。
    \item
    $J$が双斉次多項式で生成されたイデアルなら
    $\pi\colon \P^n\times \P^m \to \P^m$で$\pi(\var(J)) = \var(\hat J)$となる。(演習16より。)
    \item $J$:
    $J = \gen{y_if_j - y_j f_i}$とする。
    \item
    $\var(J) \subset \P^n\times \P^m$が$F\colon \P^n \to \P^m$のグラフ?
    \begin{enumerate}
      \item
      \begin{enumerate}
        \item $(Fのグラフ) \subset \var(J)$?
        \begin{enumerate}
          \item $\forall p$: $p\in \P^n$とする。
          \item $\forall i$:
          \item $y_i = f_i(p)$である。
          \item A,Bおわり:$(p,F(p))\in \var(J)$となる。
        \end{enumerate}
        \item $\var(J)\subset (Fのグラフ)$?
        \begin{enumerate}
          \item $\forall p,q$: $(p,q) \in \var(J)$とする。
          \item $p_\bullet, q_\bullet$: $p$の第$i$斉次座標を$p_i$とし、$q$についても同様にする。
          \item
          16より、任意の$i,j$について$q_if_j(p) = q_jf_i(p)$となる。
          \item $\exists j$:
          $q_\bullet$は斉次座標なので、$q_j\neq 0$となる$j$がある。
          \item $\exists i$:
          仮定より$f_\bullet$たちがすべて0になるようなことはないので、
          、$f_i(p)\neq 0$となる$i$がある。
          \item
          C,D,Eより、$q_if_j(p) = q_jf_i(p) \neq 0$となる。
          \item
          上より、$q_i \neq 0$である。
          \item $\lambda$: $\lambda = q_i / f_i(p)$とする。
          \item
          16より、$q=\lambda f(p)$である。
          \item Aおわり:
          上より、$(p,q) \in (Fのグラフ)$となる。
        \end{enumerate}
      \end{enumerate}
      よって、$\var(J) = (Fのグラフ)$となる。
    \end{enumerate}
    \item 上とグラフ、射影の性質より、$\pi(\var(J)) = F(\P^n)$である。
    \item
    15より、$\var(\hat J) = F(\P^n)$である。
    \item
    上より、$F$の像は$\P^m$の多様体である。
    \item
    $\var(\hat J) = \var(I_{n+1})$?
    つまり、$\var_a(\hat J) = \var_a(I_{n+1})$?
    \begin{enumerate}
      \item $\var_a (I)\subset k^{n+1}\times k^{n+1}$は
      $(f_0,\dots,f_m)$で決まる写像$k^{n+1}\to k^{n+1}$のグラフである。
      \item
      $\pi\colon k^{n+1} \times k^{m+1}\to k^{m+1}$とする。
      \item $\pi(\var_a(I)) = \var_a(\hat J)$?
      \begin{enumerate}
        \item 15より、$\var(\hat J) = F(\P^m)$である。
        \item 上より、原点を除いて考えれば
        \begin{align}
          q \in \var_a(\hat J) \iff \Exists{p\in k^{n+1}} q = F(p).
        \end{align}
        \item  $\exists \lambda$:
        上より、$q=\lambda F(p)$となる$\lambda \neq 0$がある。
        \item $\lambda'$: $\lambda' = \sqrt[d]{\lambda}$とする。
        \item
        $q=F(\lambda' p)$となる。
        \item 上より、
        \begin{align}
          \Exists{p\in k^{n+1}} q = F(p)\iff q \in \pi(\var_a(I)).
        \end{align}
        \item ii,viより、
        \begin{align}
          q\in \var_a(\hat J) \iff q \in \pi(\var_a(I)).
        \end{align}
      \end{enumerate}
      よて、$\var_a(\hat J) = \var_a(I_{n+1})$となる。
      \item
      閉包定理により、$\var_a(I_{n+1})$は$\pi(\var_a(I))$を含む最小の多様体。
      \item
      (c)(d)より、$\var_a(\hat J) = \var_a(I_{n+1})$となる。
    \end{enumerate}
    よって、$\var_a(\hat J ) = \var_a(I_{n+1})$であり、
    $\var(\hat J)= \var(I_{n+1})$となる。
    \item
    19と21より、$F(\P^n) = \var(I_{n+1})$である。
  \end{enumerate}
\end{myproof}

\subsection{2次超曲面の幾何}
\label{sub:2次超曲面の幾何}
$A\in GL(n+1,k)$は$A\colon \P^n \to \P^n$を引き起こす。

\begin{framed}
  命題1:
  $A\in GL(n+1,k)$とし、$V\subset \P^n$を多様体とする。
  すると$A(V)\subset \P^n$はまた多様体となる。
  このとき$V$と$A(V)$は射影同値であるという。
\end{framed}
\begin{myproof}
  \begin{enumerate}
    \item $\exists s, f_\bullet$: $V=\var(f_1,\dots,f_s)$とする。ただしこれらは斉次多項式。
    \item $B$: $B=A\inv{}$
    \item $g_\bullet$: 各$i$について、$g_i = f_i\circ B$とする。
    \item $b_{\bullet \bullet}$ : $B=(b_{ij})$とする。
    \item
    \begin{align}
      g_i(x_0,\dots,x_n)
      =
      f_i(\sum_{j=0}^n b_{0j}x_0,\dots, \sum_{j=0}^n b_{nj}x_n)
    \end{align}
    となる。
    \item
    上の各$g_i$は斉次
    \item
    $f_i$と全次数は一致する。
    \item
    $A(\var(f_1,\dots,f_s)) = \var(g_1,\dots,g_s)$となる。(計算する。)
  \end{enumerate}
\end{myproof}

\begin{framed}
  命題2:
  すべての超平面$H \subset \P^n$は射影同値である。
\end{framed}
\begin{myproof}
  $H$が$\var(x_0)$に射影同値なことを示す。
  \begin{enumerate}
    \item $\exists x_\bullet,f$
    $H$は$f=a_0x_0 + \dots + a_n x_n$で定義されるとする。
    $a_0\neq 0$と仮定する。
    \item
    \begin{align}
      X_0 &= a_0 x_0 + a_1 x_1 + \dots + a_n x_n,\\
      X_1 &= x_1,\\
      &\vdots\\
      X_n &= x_n.
    \end{align}
    と斉次座標をとる。
    \item
    $\var(f) = \var(X_0)$である。
    \item $A$:
    $\var(f)$と$\var(x_0)$が
    \begin{align}
      A=
      \begin{pmatrix}
        a_0 & a_1 & \ldots & a_n\\
        0 & 1 & \ldots & 0\\
        \vdots & \vdots & \ddots & \vdots \\
        0 & 0 & \ldots & 1
      \end{pmatrix}
    \end{align}
    によって射影同値。
    \item
    上は、1の$a_0\neq 0$より可逆。
    \item
    $A(\var(f)) = \var(x_0)$となる。
    \item
    上より、射影同値。
  \end{enumerate}
  $x_0$でなく$x_i$のときには$\var(x_i) \simeq \var(x_0)$を示せばよい。
\end{myproof}

\begin{framed}
  定義3:
  $f$がゼロではない全次数2の斉次多項式のとき、多様体$V=\var(f) \subset \P^n$
  を2次超曲面、あるいはもっと簡単に2次曲面とよぶ。
\end{framed}

\begin{framed}
  定理4(2次曲面の標準形):
  $f=\sum_{i,j=0}^n a_{ij}x_i x_j \in k[x_0,\dots,x_n]$を
  ゼロでない全次数が2の斉次多項式とし、$k$は標数が2でない体とする。
  すると$\var(f)$は
  \begin{align}
    c_0 x_0^2 + \dots + c_n x_n^2 = 0
  \end{align}
  で定義された2次曲面に射影同値である。ここで
  $c_0,\dots,c_n$は$k$の元であって、同時にゼロにはならない。
\end{framed}
\begin{myproof}
  \begin{enumerate}
    \item
    座標変換$X_i = \sum_{j=0}^n b_{ij}x_j$であって、$f$がの座標に関して
    \begin{align}
      c_0 X_0^2 + \dots +c_n X_n^2
    \end{align}
    と現わされるようなものを見つけてくればよい。
    \item
    帰納法を使う。1変数のときは自明。$n$変数のとき正しいとする。
    \begin{enumerate}
      \item $\forall f$: $f = \sum_{i,j=0}^n a_{ij}x_i x_j$とする。
      \item $a_{00} \neq 0$としてよい?
      \begin{enumerate}
        \item $a_{00}=0$であるが、ある$1\le j \le n$については
        $a_{jj}\neq 0$であるとき:
        \begin{enumerate}
          \item
          \begin{align}
            X_0 = x_j,\quad _Xj = x_0,\quad X_i = x_i (i\neq 0,j)
          \end{align}
          とする。
          \item
          上とiより、座標$X_0,\dots,X_n$に関して$f$の$X_0^2$の係数はゼロでない。
        \end{enumerate}
        \item すべての$a_{ii}=0$のとき:
        \begin{enumerate}
          \item $\exists i,j$:
          $f\neq 0$なので(仮定)、ある$i\neq j$について$a_{ij} \neq 0$となる。
          \item
          座標の入れ替えを行なうことにより、$a_{01}\neq 0$として一般性を失なわない。
          \item
          \begin{align}
            X = x_0,\quad X_1 = x_1-x_0,\quad X_i = x_i (i\ge 2)
          \end{align}
          とする。
          \item
          $f = \sum_{i,j=0}^n c_{ij}X_i X_j$とあらわすと、
          計算すれば$c_{00} = a_{01}\neq 0$となる。
        \end{enumerate}
        よって、適当な座標変換で$a_{00} \neq 0$としてよい。
      \end{enumerate}
      \item
      \begin{align}
        \frac{1}{a_{00}}(a_{00}x_0 + \sum_{i=1}^n \frac{a_{i0}}{2}x_i)^2
        =
        a_{00}x_0^2 + \sum_{i=1}^n a_{i0}x_0 x_i
        +
        \sum_{i,j=1}^n \frac{a_{i0}a_{j0}}{4a_{00}}x_i x_j
      \end{align}
      である。標数は2ではないので、上の2での割り算は正当である。
      \item
      座標変換
      \begin{align}
        X_0 = x_0 + \frac{1}{a_{00}}\sum_{i=1}^n \frac{a_{i0}}{2}x_i,\quad
        X_i = x_i (i\ge 1).
      \end{align}
      \item
      上で計算すると、
      \begin{align}
        a_{00}X_0^2 + \sum_{i,j=1}^n d_{ij}X_i X_j.
      \end{align}
      となる。
      \item
      帰納法の仮定より、$\sum_{i,j=1}^n d_{ij}X_i X_j$
      を$e_1 X_1^2 + \dots + e_n X_n^2$に写せる。
    \end{enumerate}
  \end{enumerate}
\end{myproof}

\begin{framed}
  定義5:
  $V\subset \P^n$を2次超曲面とする。
  \begin{enumerate}[label=(\roman*)]
    \item $V$が
    \begin{align}
      c_0 x_0^2 + \dots + c_p x_p^2 = 0,\quad c_0,\dots,c_p \neq 0
    \end{align}
    によって定義されているとき、$V$は階数$p+1$を持つという。
    \item
    もっと一般に、$V$が任意の2次曲面のとき、$V$が階数$p+1$を持つとは、
    $V$が上で定義される2次曲面と射影同値であるときにいう。
  \end{enumerate}
\end{framed}
上はwell-definedになる。

\begin{framed}
  命題6:
  $Q$を$(n+1)\times (n+1)$対称行列として、$f=\bbold x^t Q \bbold x$とおく。
  \begin{enumerate}[label=(\roman*)]
    \item $A\in GL(n+1,k)$が与えられたとする。
    $B=A\inv{} $とおくと、$g=\bbold x^t \bbold B^t Q B \bbold x$に対して、
    $A(\var(f)) = \var(g)$が成り立つ。
    \item
    2次超曲面$\var(f)$の階数は行列$Q$の階数に一致する。
  \end{enumerate}
\end{framed}
\begin{myproof}
  (i)を示す。
  \begin{enumerate}
    \item $g$: $g=f\circ B$
    \item 計算すると、$A(\var(f)) = \var(g)$となる。
    \item
    \begin{align}
      g(\bbold x) =
      f(B\bbold x)
      =
      (B\bbold x)^t Q (B\bbold x)
      =
      \bbold x^t B^t Q B \bbold x.
    \end{align}

  \end{enumerate}

  (ii)を示す。
  \begin{enumerate}
    \item $\rank Q = \rank B^t QB$となる。
    \item $\exists A,c_\bullet$:
    $g=c_0 x_0^2 + \dots + c_p x_p^2$となる$A,C_\bullet$がある。
    ここで、$c_0,\dots,c_p \neq 0$である。((i)のなかで、$B$も定まる。)
    \item
    \begin{align}
      B^t Q B =
      \begin{pmatrix}
        c_0 & & & & & \\
        & \ddots & & & &\\
        & & c_p & & & \\
        & & & 0 & & \\
        & & & & \ddots & \\
        & & & & & 0
      \end{pmatrix}.
    \end{align}
    \item
    $\rank B^t Q  B = p+1$となる。
  \end{enumerate}
\end{myproof}

\begin{framed}
  命題7:
  $k$が代数的閉体であって、その標数が2でないとする。
  すると、階数が$p+1$の2次超曲面は、次の式で定義される2次曲面に射影同値である。
  \begin{align}
    \sum_{i=0}p x_i^2 = 0.
  \end{align}
  特に、2つの2次曲面は、等しい階数を持つとき、かつそのときに限り射影同値である。
\end{framed}
\begin{myproof}
  階数$p+1$の2次超曲面が、それに射影同値であることを示す。
  \begin{enumerate}
    \item $c_\bullet,p$: 定理4より、2次曲面は$c_0 x_0^2 + \dots + c_p x_p^2=0$としてよい。$p+1$が階数。どの係数も0でない。
    \item 任意の$i$について、$k$が代数的閉体ということから、
    $x^2 - c_i = 0$は$k$に根を持つので、その1つを$\sqrt{c_i}$とする。
    \item
    1で各$c_i$は0でないので、上の$\sqrt{c_i} \neq 0$である。
    \item
    \begin{align}
      X_i = \sqrt{c_i}x_i \, (0\le i \le p),\quad
      X_i = x_i,\, (p < i\le n).
    \end{align}
    と変換する。
    \item
    上の変換を計算すると射影同値である。
  \end{enumerate}

  射影同値なら同じ階数を示す。
  \begin{enumerate}
    \item $f,g$: $\var(f)$と$\var(g)$が射影同値であるとする。
    \item $Q, B$
    命題7より、$f$は$Q$に対応しており、$g$は$B^t Q B$に対応する。
    $B$は可逆。
    \item
    上で$B$は可逆なので、$\rank Q = \rank (B^t Q B)$となる。
    \item
    上より、$\var(f)$と$\var(g)$は同じ階数である。
  \end{enumerate}
\end{myproof}

\begin{framed}
  定義8:
  $\P^n$の2次曲面はその階数が$n+1$であるとき、非特異であるという。
\end{framed}

\begin{framed}
  系9:
  $k$を代数的閉体とする。このとき$\P^n$のすべての非特異2次曲面は射影同値である。
\end{framed}

\begin{framed}
  命題10:
  セグレ写像$\sigma\colon \P^1 \times \P^1 \to \P^3$は1対1であって、
  その像は非特異2次曲面$\var(z_0z_3-z_1z_2)$と一致する。
  ただし、$\sigma$は$(x_0,x_1,y_0,y_1)\mapsto (x_0y_0,x_0y_1,x_1y_0,x_1y_1)$である。
\end{framed}
\begin{myproof}
  $\sigma(\P^1\times \P^1) \subset \var(z_0z_3-z_1z_2)$を示す。
  \begin{enumerate}
    \item
    $z_\bullet$: $(z_0,z_1,z_2,z_3)\in \sigma(\P^1\times \P^1)$を$\P^3$の斉次座標とする。
    \item $\exists x_0,x_1,y_0,y_1$:
    上より、
    \begin{align}
      x_0 y_0 &= z_0,\\
      x_0 y_1 &= z_1,\\
      x_1 y_0 &= z_2,\\
      x_1 y_1 &= z_3.
    \end{align}
    となる$x_0,x_1,y_0,y_1\in k$が存在する。
    \item
    上より、$z_0z_3-z_1 z_2 = 0$となる。
  \end{enumerate}

  等号を示す。つまり、$\sigma(\P^1\times \P^1) \supset \var(z_0z_3-z_1z_2)$を示す。
  \begin{enumerate}
    \item $\forall w_0,w_1,w_2,w_3$:
    $(w_0,w_1,w_2,w_3) \in \var(z_0z_3-z_1z_2)$とする。
    \item 上のどれか1つが0でないので、$w_0\neq 0$とする(他は同様。)
    \item 上より、
    $(w_0,w_2,w_0,w_1) \in \P^1\times \P^1$となる。
    \item
    $\sigma(w_0,w_2,w_0,w_1) = (w_0^2,w_0w_1,w_0w_2,w_1w_2)$となる。
    \item
    1より、$w_0w_3-w_1w_2 = 0$である。
    \item
    4,5より、
    \begin{align}
      \sigma(w_0,w_2,w_0,w_1)
      =
      (w_0^2,w_0w_1,w_0w_2,w_0w_3)
      =
      w_0(w_0,w_1,w_2,w_3)
      =
      (w_0,w_1,w_2,w_3).
    \end{align}
  \end{enumerate}

  よって、$\sigma(\P^1\times \P^1) = \var(z_0z_3-z_1z_2)$を示す。
\end{myproof}

\begin{framed}
  $p=(a_0,a_1,a_2,a_3) \in k^4$とし、
  $q=(b_0,b_1,b_2,b_3) \in k^4$とし、
  $\Omega = \tatev{a_0 & a_1 & a_2 & a_3 \\ b_0 & b_1 & b_2 & b_3}$とする。
  \begin{align}
    w_{ij} &= \det \tatev{a_i & a_j \\ b_i & b_j}
  \end{align}
  を$(i,j)=(0,1),(0,2),(0,3),(1,2),(1,3),(2,3)$で考え、
  \begin{align}
    w(p,q) = (w_{01},w_{02},w_{03},w_{12},w_{13},w_{23}) \in k^6
  \end{align}
  とする。これは、$p-q$を通る$\P^3$の直線にのみ依存する。$\omega(L)$と書く。
  また、この像は$\var(z_{01}z_{23}-z_{02}z_{13}+z_{03}z_{12})$となる。
\end{framed}

\begin{framed}
  定理11:
  直線$L\subset \P^3$をそのプリュッカー座標
  $\omega(L) \in \var(z_{01}z_{23}-z_{02}z_{13}+z_{03}z_{12})$に対応させる写像
  \begin{align}
    \set{\P^3 の直線全体}
    \to
    \var(z_{01}z_{23}-z_{02}z_{13}+z_{03}z_{12})
  \end{align}
  は全単射である。(プリュッカー座標はあたりまえだが射影化されている。)
\end{framed}
\begin{myproof}
  \begin{enumerate}
    \item $\forall L$: $L$を$\P^3$の直線とする。
    \item $\forall p,q$: $p,q$を$L$上の点とする。
    \item $\exists a_\bullet, b_\bullet$:
    $p=(a_0,a_1,a_2,a_3)$とし、$q=(b_0,b_1,b_2,b_3)$とする。
    \item $w_\bullet$を
    $F\colon \P^1 \to \P^3$で
    \begin{align}
      F(u,v) = u\cdot (a_0,\dots,a_3) - v\cdot (b_0,\dots,b_3)
    \end{align}
    を用いて、
    \begin{align}
      b_0 p - a_0 q &= F(b_0,a_0) = (0,-w_{01},-w_{02},-w_{03}),\\
      b_1 p - a_1 q &= F(b_1,a_1) = (w_{01},0,-w_{12},-w_{13}),\\
      b_2 p - a_2 q &= F(b_2,a_2) = (w_{02},w_{12},0,-w_{23}),\\
      b_3 p - a_3 q &= F(b_3,a_3) = (w_{03},w_{13},w_{23},0).
    \end{align}
    とする。
    \item
    上は$F$で書けており、演習13より$F$の像は$p,q$を通る直線となるので、
    上の4つが0でないならば、それらは$p,q$を通る直線すなわち$L$を通る。
    \item $\omega$が単射であることを示す。
    \begin{enumerate}
      \item $\forall L, L'$: $\omega(L)$と$\omega(L')$が射影的に等しいような$L,L' \subset \P^3$を考える。
      \item $\exists \lambda$:
      $\omega(L)= \lambda \omega(L')$がある非零な$\lambda$になりたつ。
      \item $\omega(L),\omega(L')$のプリュッカー座標を
      $w_{ij},w_{ij}'\quad (0\le i < j \le 3)$とする。
      (プリュッカー座標の定義を確認!)
      \item
      (b)と(c)より、$w_{ij} = \lambda w_{ij}'$が$0\le i < j\le 3$
      となる。
      \item
      $w_{01}\neq 0$とする。(他のプリュッカー座標の成分では、次で4からの選びかたを変えれば同様に通る。)
      \item
      $P,Q$: (d)より、
      \begin{align}
        P
        &=
        (0,-w'_{01},-w'_{02},-w'_{03})
        =
        (-0,-\lambda w_{01},-\lambda w_{02}, -\lambda w_{03})
        =
        (0,-w_{01},-w_{02},-w_{03}),\\
        Q
        &=
        (w'_{01},0,-w'_{12},-w'_{13})
        =
        (\lambda w_{01},0,-\lambda w_{12},-\lambda w_{13})
        =
        (w_{01},0,-w_{12},-w_{13}).
      \end{align}
      \item (f)より、
      $P = (0,-w'_{01},-w'_{02},-w'_{03})$と、$L'$のプリュッカー座標の
      4のような並びで書けているので、5より$P \in L'$となる。
      \item (f)より、
      $P = (0,-w_{01},-w_{02},-w_{03})$と、$L$のプリュッカー座標の
      4のような並びで書けているので、5より$P\in L$となる。
      \item
      (g)(h)と同様に$Q$も(f)より、$Q\in L,L'$となる。
      \item
      演習問題14より、2点を通る直線は一意に定まる。
      \item
      (g),(h),(i),(j)より、$L=L'$である。
      \item
      (a)おわり: $\omega$は単射である。
    \end{enumerate}

    全射であることを示す。
    \begin{enumerate}
      \item
      $\forall w_{\bullet}$:
      $(w_{01},w_{02},w_{03},w_{12},w_{13},w_{23}) \in
      \var(z_{01}z_{23}-z_{02}z_{13}+z_{03}z_{12})$とする。
      \item
      上はプリュッカー座標なので、どれか1つの成分は0ではなく、
      したがって4の4つのベクトルから0でないものを2つ選ぶことができる。
      ここでは仮に$w_{01} \neq 0$とし、上2つの
      $(0,-w_{01},-w_{02},-w_{03})$と$(w_{01},0,-w_{12},w_{13})$
      を選んだとする。
      \item $L$:
      5より、上の2ベクトルが直線$L$を定める。
      \item
      $\omega(L)$の定義と$L$のプリュッカー座標から、$L$のプリュッカー座標が
      (a)で定めた
      $(w_{ij})$と一致することが計算できる(演習16)。
      \item
      (a)おわり: $\omega$は全射。
    \end{enumerate}
  \end{enumerate}
\end{myproof}

\subsection{ベズーの定理}
\label{sub:ベズーの定理}

\begin{framed}
  定義(曲線):
  $f\in \C[x,y,z]$によって定義された射影多様体
  $\var(f)$を曲線とよぶ。
\end{framed}

\begin{framed}
  演習3:
  $f=gh$を斉次多項式、
  $g=g_m + \dots +g_0$を$g_i$が全次数$i$であるような
  斉次成分への分解で、$g_m \neq 0$とする。
  $h = h_n + \dots + h_0$も同様とすう。
  このとき、$f=g_m h_n$であることは命題4で示す。
  このとき、$g=g_m$かつ$h=h_n$となることを示せ?
\end{framed}
\begin{myproof}
  $m_0$を$g_{m_0}\neq 0$となる最小の$m_0$とする。
  $n_0$も$h$について同様とする。
  この定義により、$f=gh$の$gh$の全次数最小の項は$g_{m_0}h_{n_0}$となる。
  $f$の全次数最小の項は$f=g_m h_n$より、$g_m h_n$となる。
  よって、$g_{m_0}h_{n_0} = g_m h_n$となり、$m=m_0,\,n=n_0$となる。
  よって、$g=g_m$かつ$h=h_n$となる。
\end{myproof}

\begin{framed}
  演習6:
  \begin{enumerate}[label=(\alph*)]
    \item
    $f\in \C[x_1,\dots,x_n]$を0でない(\warn{非定数もつけないとだめだよね?})多項式とする。
    このとき、$\var(f)$と$\C^n-\var(f)$はどちらも空集合でない。
    \item
    上を用いて、$q\notin C \cup D \cup \bigcup_{i<j} L_{ij}$となる点がある?
    \item
    $q\in \P^2(\C)$について、
    $A\in GL(3,\C)$であって、$A(q) = (0,0,1)$となるものを見つけよ。
    \item
    $(0,0,1)$と$(u,v,w)$を結ぶ射影直線が、直線$z=0$と点$(u,v,0)$で交わることを示せ。
  \end{enumerate}
\end{framed}
\begin{myproof}
  (a)を示す。(定理の証明としては射影的なほうを使わなければいけないはずなのでなんで解いたのかよくわからない。)
  $\var(f)= \emptyset \iff \gen{f} = k[x_1,\dots,x_n] \iff f\in \C$となる。
  よって、$f\notin \C \iff \var(f) \neq \emptyset$となる。
  次に、$\var(f) = \C^n$とする。すると、$f$は無数の根を持つことになる。
  $k$は代数的閉体ゆえ無限体で、$f=0$となる。よって、$f\neq 0$ならば
  $\var(f) \subsetneq \C^n$であり、$\C^n-\var(f) \neq \emptyset$である。

  (b)を示す。

  Pending...
\end{myproof}

\begin{framed}
  命題4:
  $f\in \C[x,y,z]$を0でない斉次多項式(しかも定数でない)とすると、
  $f$の既約因子はまた斉次である。$f$の因数分解を
  \begin{align}
    f= f_1^{a_1}\dots f_s^{a_s}
  \end{align}
  とする。ここに$i\neq j$なら、$f_i$と$f_j$は互いに
  定数倍でない既約因子である。このとき、
  \begin{align}
    \var(f) = \var(f_1) \cup \dots \cup \var(f_s)
  \end{align}
  は$\var(f)$の$\P^2$における既約成分への無駄のない分解であり、
  \begin{align}
    \ideal(\var(f))  = \sqrt{\gen{f}} = \gen{f_1,\dots,f_s}
  \end{align}
  が成り立つ。
\end{framed}
\begin{myproof}
  \begin{enumerate}
    \item $f$が$g,h \in \C[x,y,z]$によって$f=gh$と分解し、
    $f$が斉次ならば、$g,h$も斉次である?
    \begin{enumerate}
      \item $g_\bullet,m$: $g=g_m + \dots + g_0$と書く。$g_m\neq 0$としておく。添字は全次数とする。
      \item $h_\bullet,n$: $h=h_n + \dots + h_0$と書く。$h_n \neq 0$としておく。添字は全次数とする。
      \item
      \begin{align}
        f = gh
        =
        g_m h_n + (全次数がmn未満の項).
      \end{align}
      \item
      上と、仮定の$f$が斉次であることから、$f=g_mh_n$となる。
      \item
      $m_0$:
      0でない$g$の最低全次数の添字とする。
      \item
      $n_0$: 0でない$h$の最低全次数の添字とする。
      \item
      (e),(f)より、$gh$の最低全次数は$g_{m_0}h_{n_0}$である。
      \item
      仮定の$f=gh$と(d)の$f=g_m h_n$で、両辺の最低全次数を比較し、上より
      $g_m h_n = g_{m_0}h_{n_0}$となる。
      \item
      上より、$m=m_0,\, n=n_0$となる。
      \item
      上より、$g=g_m,\, h=h_n$となる。
    \end{enumerate}
    $f$が$g,h \in \C[x,y,z]$によって$f=gh$と分解し、
    $f$が斉次ならば、$g,h$も斉次である。
    \item
    \begin{align}
      \var(f) = \var(f_1^{a_1}\dots f_s^{a_s})
      =
      \var(f_1^{a_1}) \cup \dots \cup \var(f_s^{a_s})
      =
      \var(f_1)\cup \dots \cup \var(f_s).
    \end{align}
    \item
    上は無駄のない分解である。なぜなら、$i\neq j$について$\var(f_i) \subset \var(f_j)$
    なら$\gen{f_j} \subset \gen{f_i}$となるが、これは$f_i | f_j$となり、これは既約因子への分解に矛盾する。
    \item
    $\var(f) \neq \emptyset$である?
    \begin{enumerate}
      \item $f$の先頭項係数を正規化しておく。$\gen{f}$は簡約グレブナ基底になっている。
      \item 射影幾何の零点定理より、
      $\var(f)= \emptyset$と「$G$を$\gen{f}$のグレブナ基底として、$\LT(g)$が$x$のべきであるような$g\in G$、
      $y$のべきであるような$g\in G$、$z$のべきであるような$g\in G$
      がすべて存在する」となる
      \item (a)(b)より、$\var(f)\neq \emptyset$となる。
    \end{enumerate}
    $\var(f) \neq \emptyset$である。
    \item
    上と強形の零点定理より、$\ideal(\var(f)) = \sqrt{\gen{f}}$である。
    \item
    $\sqrt{\gen{f_1\dots f_s}} = \gen{f_1 \dots f_s}$である。
    なぜなら、左辺から$[g(f_1\dots f_s)]^N$をとると、
    $g^N(f_1 \dots f_s)^N \in \gen{f_1\dots f_s}$となるから。
    \item
    2と上より、
    \begin{align}
      \ideal(\var(f))
      =
      \ideal(\var(f_1)\cup \dots \cup \var(f_s))
      =
      \ideal(\var(\gen{f_1}\dots \gen{f_s}))
      =
      \ideal(\var(\gen{f_1\dots f_s}))
      =
      \sqrt{\gen{f_1\dots f_s}}
      =
      \gen{f_1\dots f_s}.
    \end{align}
    \item 5,7より、
    \begin{align}
      \ideal(\var(f)) = \sqrt{f} = \gen{f_1 \dots f_s}.
    \end{align}
  \end{enumerate}
\end{myproof}

\begin{framed}
  補題5:
  $f,g \in \C[x,y,z$を斉次多項式で、
  全次数がそれぞれ$m,n$のものとする。
  $f(0,0,1)$と$g(0,0,1)$がゼロでなければ、
  終結式$\Res(f,g,z)$は$x,y$に関して斉次$mn$次である。
\end{framed}
\begin{myproof}
  \begin{enumerate}
    \item $\exists a_\bullet,b_\bullet$:
    \begin{align}
      f&=a_0 z^m + \dots + a_m,\\
      g&= b_0 z^n + \dots + b_n.
    \end{align}
    となる斉次多項式$a_\bullet,\, b_\bullet \in \C[x,y]$がある。
    ここで、$f,g$は斉次なので、$a_i$は全次数$i$次、
    $b_j$は全次数$j$次としてある。
    \item
    各$a_i$が全次数$i$次斉次であったことから、
    \begin{align}
      f(0,0,1)
      =
      a_0(0,0)1^m + \dots + a_m(0,0)
      =
      a_0.
    \end{align}
    となる。$a_0$が全次数0で、定数であることに注意。
    \item
    上と仮定$f(0,0,1) \neq 0$より、$a_0 \neq 0$である。
    \item
    2-3と同様の議論を$g,b_\bullet$にもして、$g(0,0,1) \neq 0$より、
    $b_0 \neq 0$である。
    \item $c_{ij}$: $c_{ij}$を$Syl(f,g,z)$の$ij$要素とする。
    \item
    シルベスター行列の定義により、ゼロでない$c_{ij}$は、
    \begin{align}
      c_{ij} =
      \begin{cases}
        a_{i-j} &; j\le n\\
        b_{n+i-j} & ; j>n
      \end{cases}.
    \end{align}
    となる。(ちなみにこれは、
    \begin{align}
      c_{i,1} =
      \begin{cases}
        a_{i-1} & ; 1\le i\le m+1 \\
        0 & ; ほか
      \end{cases},\quad
      c_{i,j} = c_{(i-1),(j-1)} = \dots = c_{i-(j-1),j-1}
    \end{align}
    はあきらかなので、これを使うと詳しく調べられる。
    )
    \item
    上より、$c_{ij}$はゼロでなければ、
    \begin{align}
      \deg c_{ij} =  \deg
      \begin{cases}
        i-j &; j \le n\\
        n+i-j &; j>n
      \end{cases}.
    \end{align}
    であり、1より斉次多項式である。
    \item
    行列式の定義より、
    \begin{align}
      \Res(f,g,z) = \sum_{\sigma \in S_{m+n}} \mathrm{sgn}(\sigma) \prod_{i=1}^{m+n}c_{i\sigma(i)}
    \end{align}
    である。(この各項の次数を数えたい。)
    \item $\forall \sigma$: $\sigma \in S_{m+n}$とする。
    \item
    \begin{align}
      \prod_{i=1}^{m+n} c_{i\sigma(i)} =
      \prod_{\sigma(i)\le n}c_{i\sigma(i)}\prod_{\sigma(i)>n}c_{i\sigma(i)}.
    \end{align}
    \item
    上と7より、この式の全次数は、
    \begin{align}
      \deg(\prod_{i=1}^{m+n} c_{i\sigma(i)})
      &=
      \sum_{\sigma(i)\le n}(i-\sigma(i)) + \sum_{\sigma(i)>n} (n+i-\sigma(i))\\
      &=
      \sum_{i=1}^{m+n}(i-\sigma(i)) + \sum_{\sigma(i)>n}n\\
      &=
      \sum_{\sigma(i)>n}n\\
      &=
      mn.
    \end{align}
    \item
    9おわり:8の各項の全次数は$mn$である。
    \item
    上より、$\Res(f,g,z)$は次数$mn$の斉次多項式である。
  \end{enumerate}
\end{myproof}

\begin{framed}
  定義:
  曲線$C\subset \P^2$が斉次多項式$f$で$C=\var(f)$と書けているとし、
  $f$の相異なる既約因子が$f_1,\dots,f_s$であるとき、
  $f_1\dots f_s$を被約多項式といい、$f_1\dots f_s = 0$を$C$の被約多項式方程式という。
\end{framed}

\begin{framed}
  補題6:
  $h\in \C[x,y]$をゼロでない斉次多項式とする。
  すると$h$は次の形に因数分解される。
  \begin{align}
    h= c(s_1 x - r_1 y)^{m_1} \dots (s_tx -r_t y)^{m_t}.
  \end{align}
  ここで$c\neq 0$は複素数であり、
  $(r_1,s_1),\dots,(r_t,s_t)$は$\P^1$の相異なる点である。さらに
  \begin{align}
    \var(h) = \set{(r_1,s_1),\dots,(r_t,s_t)} \subset \P^1
  \end{align}
  が成り立つ。
\end{framed}
\begin{myproof}
  非斉次化$h(x,1)\in \C[x]$なので、代数的閉体の性質より
  \begin{align}
    h(x,1) = c(x-a_1)^{m_1} \dots (x-a_t)^{m_t}
  \end{align}
  となる$a_\bullet\in \C$と$m_\bullet \in \N$と$c \in \C$が存在する。
  ここで、$a_\bullet$は相異なる。
  $e$を、$h$が$y^e$で割り切れ、それより高い羃では割り切れないような自然数とする。
  命題2-7(iv)より、
  \begin{align}
    h = y^e h(x,1)^h
  \end{align}
  となるので、
  \begin{align}
    h = c y^e (x-a_1 y)^{m_1}\dots (x-a_t y)^{m_t}
  \end{align}
  となっている。よって、$e>0$ならば
  \begin{align}
    \var(h) = \set{(1:0),(a_1:1),\dots,(a_t:1)} \subset \P^1
  \end{align}
  であり、$e=0$ならば
  \begin{align}
    \var(h) = \set{(a_1:1),\dots,(a_t:1)} \subset \P^1
  \end{align}
  である。$a_\bullet$が相異なることより、これらも相異なる。
\end{myproof}

\begin{framed}
  定理7:
  $C$と$D$を$\P^2$の射影曲線であって、共通の既約成分を持たないものとする。
  $C$と$D$の被約多項式方程式の次数をそれぞれ$m$および$n$とすれば、
  $C\cap D$は有限集合であって、高々$mn$個の点しか持たない。
\end{framed}
\begin{myproof}
  \begin{enumerate}
    \item $\implies$:
    $C\cap D$が交点を$mn$個より多く持ったとする(背理法)。
    \item $\exists p_\bullet$:
    上より、交点が$mn+1$個選べるので、$p_1,\dots,p_{mn+1}$と番号付けられる。
    \item $L_\bullet$:
    $1\le i < j \le mn+1$について、$L_{ij}$を$p_i$と$p_j$を通る直線とする。
    \item $f$: $f$を$C=\var(f)$となる被約多項式とする(命題4より斉次となる。)
    \item $g$: $g$を$D=\var(g)$となる被約多項式とする(命題4より斉次となる。)。
    \item $f$は仮定より次数$m$である。
    \item $g$は仮定より次数$n$である。
    \item $l_{ij}$:
    $l_{ij}$を
    \begin{align}
      l_{ij} =
      \begin{vmatrix}
        \tatev{x \\ y \\ z} & p_{i} & p_j
      \end{vmatrix} \in \C[x,y,z]
    \end{align}
    と定義する。ただし、$p_i,p_j$は何か斉次座標をとったものと考える。
    \item $l_{ij}$は上の定義より、$p_i, p_j$を通る直線になっており、
    3より
    $\var(l_{ij}) = L_{ij}$となっている。
    \item $\exists q$:
    $q \notin C \cup D \cup \bigcup_{i<j} L_{ij}$となる$q\in \P^2$が存在する?
    \begin{enumerate}
      \item $\P^2 \setminus (C\cup D \cup \bigcup_{i<j}L_{ij}) \neq \emptyset$ であることを示せばよい。
      これは、アフィン多様体では
      \begin{align}
      [\C^3 \setminus (\var_a(f)\cup \var_a(g)\cup \bigcup_{i<j} \var_a(l_{ij})) ]\cap [\C^3\setminus \zeroset] \neq \emptyset
      \end{align}
      を示せばよい。
      \item
      上の否定を仮定する。(背理法)
      \begin{align}
        \var_a(f) \cup \var_a(g) \cup \bigcup_{i<j}\var_a(l_{ij}) \cup \zeroset = \C^3
      \end{align}
      となっている。
      \item
      $\zeroset \subset \var(x)$なので、
      上より、
      \begin{align}
        \C^3
        &=
        \var_a(f) \cup \var_a(g) \cup \bigcup_{i<j}\var_a(l_{ij}) \cup \zeroset \\
        &\subset
        \var_a(f) \cup \var_a(g) \cup \bigcup_{i<j}\var_a(l_{ij}) \cup \var_a(x)
      \end{align}
      となり、これは$\C^3$である。
      \item
      上は、
      \begin{align}
        \var_a(fg \prod_{i<j}l_{ij} x) = \C^3
      \end{align}
      を意味する。
      \item
      上はは$fg \prod_{i<j}l_{ij}x = 0$が$\C^3$全体で成立することを意味するので、
      $k$が無限体であることから、$fg \prod_{i<j}l_{ij}x$が多項式として0になる。これは矛盾である。
      \item
      (b)おわり: (a)の式が示された。
    \end{enumerate}
    よって、$q \notin C\cup D \cup \bigcup_{i<j}L_{ij}$となる$q\in \P^2$が存在する。
    \item $\exists A$:
    $Aq = \tatev{0 \\ 0 \\ 1}$となる$A$が存在する。
    なぜなら、$q\neq 0$なので、3次元回転と拡大を組み合わせればよい
    \footnote{あるいは、$q$と、それと一次独立な$q',q''$をとって$(q,q',q'')$を作り、それを$(1,0,0)^t$にかけると$q$になる。$(q,q',q'')$は可逆なのでOK。}
    。
    \item
    上より、$A$で変換することにより10の$q$を$\tatev{0 \\ 0\\ 1}$として一般性を失なわない。
    \item
    4($f$について),10($q$の条件),12($q$のとりなおし)より、
    $f(0,0,1)\neq 0$となる。
    \item
    上と同様に、$g(0,0,1)\neq 0$となる。
    \item
    13,14と補題5より、$\Res(f,g,z)$は$x,y$に関する斉次$mn$次の多項式である。
    \item
    命題3-6-1の終結式の性質と、$f,g$が$z$について正の次数であることと、仮定の$f,g$が共通の既約成分を持たないことより、
    $\Res(f,g,z) \neq 0$である。
    \item
    命題3-6-1より、$\Res(f,g,z)\in \gen{f,g}$である。
    \item $u_\bullet,v_\bullet,w_\bullet$:
    各$p_i$について、$p_i = (u_i,v_i,w_i)$と$u_i,v_i,w_i \in \C$を定める。
    \item
    2の定義より、$f,g$が上の各$p_i=(u_i,v_i,w_i)$を消し、さらに
    17より、各$i$について$\Res(f,g,z)(u_i,v_i)  = 0$である。
    \item
    $q=(0,0,1)$と$p_i=(u_i,v_i,w_i)$を結ぶ挑戦は、$z=0$と
    $(u_i,v_i,0)$で交わる。実際内分の式を考えればよい。
    \item
    10より、$(0,0,1)$は$p_i,p_j$をむすぶどの直線$L_{ij}$上にもない。
    \item
    上より、$(u_i,v_i,0)$は各$i$についてすべて異なる(射影平面上で異なる直線は1点で交わる。$(0,0,1)$は共有しているのだからもう共有点はない。)。
    \item
    上より、$\set{(x,y,0) \in \P^2} \simeq \P^1$の同一視によって、
    相異なる$(u_i,v_i)$が得られる。
    \item
    19と上より、$\Res(f,g,z)$は$mn+1$個の相異なる点$(u_i,v_i)$で消える。
    \item
    15,16によれば$\Res(f,g,z)$は$mn$次の0でない多項式だが、これは
    上に矛盾する。
  \end{enumerate}

\end{myproof}

\begin{framed}
  定義8:
  $C$と$D$は$\P^2$の曲線で、共通の成分がなく、
  被約方程式$f=0$および$g=0$でそれぞれ定義されているとする。
  $\P^2$の座標を
  \begin{align}
    (0,0,1) \notin C\cup D \cup \bigcup_{p\neq q \in C\cap D}L_{pq}
  \end{align}
  をみたすように選ぶ\footnote{先と同じようにまず条件をみたす点をもとの座標でとって、それを$(0,0,1)$に写す可逆変換を考えればよい。}。 $p=(u,v,w) \in C\cap D$に対して、
  交わりの重複度$I_p(C,D)$を、$\Res(f,g,z)$の因数分解における
  因子$vx-uy$の指数と定義する。
\end{framed}

\begin{framed}
  定理10(ベズーの定理):
  $C$と$D$を$\P^2$における曲線で、共通の成分を持たないものとし、
  $m$と$n$をそれぞれの被約定義多項式の次数とする。すると
  定義8で定義された$p$における交わりの重複度$I_p(C,D)$に対して、
  \begin{align}
    \sum_{p\in C\cap D}I_p(C,D) = mn
  \end{align}
  が成り立つ。
\end{framed}
\begin{myproof}
  \begin{enumerate}
    \item  $f$:
    $f=0$を$C$の被約方程式とする。
    \item $g$: $g=0$を$D$の被約方程式とする。
    \item 適当な座標変換をかけることにより、
    \begin{align}
      (0,0,1)\notin
      C\cup D \cup \bigcup_{p\neq q \in C\cap D} L_{pq}
    \end{align}
    であると仮定してよい。
    \item $u_\bullet,v_\bullet,w_\bullet$: 記号を定める。$p\in C\cap D$について、$u_p,v_p,w_p \in \C$を
    $p=(u_p,v_p,w_p)$と定める。
    \item $\exists c$:
    \begin{align}
      \Res(f,g,z) =
      c \prod_{p\in C\cap D} (v_p x - u_p y)^{I_p(C,D)},\quad c\neq 0
    \end{align}
    が成立?
    \begin{enumerate}
      \item
      各$p$について、重複度$I_p(C,D)$の定義より、
      終結式$\Res(f,g,z)$を割り切る$v_p x - u_p y$の最大の羃は
      $(v_p x - u_p y)^{I_p(C,D)}$である。
      \item
      上より、曲線の交点$(u_p,v_p,w_p)$から作った$(u_p,v_p)$はすべて終結式の根になっている。
      \item
      終結式のすべての根は曲線の交点から得られる?
      終結式の根になってるのに曲線の交点になってないやつとかあったりしない?
      \begin{enumerate}
        \item $\forall u,v$: $(u,v) \in \P^1$が$\Res(f,g,z)(u,v) = 0$をみたすとする。
        \item
        $f$の$z$についての先頭項の係数($k[x,y]$の元になる)は0でない定数である?
        \begin{enumerate}
          \item $m$: $f$の全次数を$m$とする。
          \item $a_\bullet$: $f=a_0 z^m + \dots + a_m$とする。$a_\bullet$は斉次。
          \item $f$は斉次多項式なので、上の$a_0$は定数である。
          \item
          1と3より、
          \begin{align}
            a_0 = a_0(0,0)\cdot 1^m + \dots + a_m(0,0) = f(0,0,1)  \neq 0.
          \end{align}
        \end{enumerate}
        よって、$f$の$z$についての先頭項係数は0でない定数である。
        \item
        上と同様に、$g$の$z$についての先頭項の係数は0でない定数である。
        \item $\exists$:
        iiとiiiを命題3-6-3を適用し、$w\in \C$が存在して、$f,g$が$(u,v,w)$で消える。
        \footnote{
        命題3-6-3は、「$f,g\in \C[x_1,\dots,x_n]$に対して、$a_0,b_0 \in \C[x_2,\dots,x_n]$を
        $f,g$の$x_1$についての先頭項係数とする。もし$\Res(f,g,x_1) \in \C[x_2,\dots,x_n]$が
        $(c_2,\dots,c_n) \in \C^{n-1}$において消えるとすれば次のいずれかが成立する。
        (i)$a_0$または$b_0$が$(c_2,\dots,c_n) \in \C^{n-1}$で消える。
        (ii)$c_1 \in \C$が存在して、$f$と$g$は$(c_1,\dots,c_n) \in \C^n$で消える。」だった。
        証明は、(i)を否定しておいて終結式を計算する。
        }
        \item
        上より、
        $(u,v,w)\in C\cap D$となり、$(u,v)$は曲線の交点になっている。
      \end{enumerate}
      終結式のすべての根は曲線の交点から得られる。
      \item $\exists c$: (b),(c)より、$\Res(f,g,z)$を割りきるものは各曲線の交点から
      作った$(v_px-u_py)^{I_p(C,D)}$だけだと分かったので、ある定数$c\neq 0$が存在して、
      \begin{align}
        \Res(f,g,z) = c\prod_{p\in C\cap D} (v_p x - u_p y)^{I_p(C,D)}
      \end{align}
      となる。
    \end{enumerate}
    \item
    5の両辺の次数を計算すると、3で$f,g$が$(0,0,1)$で消えないことから補題5より、
    \begin{align}
      mn &= \deg(\Res(f,g,z))\\
      & = \deg(c \prod_{p\in C\cap D}(v_p x - u_p y)^{I_p(C,D)})\\
      &=
      \sum_{p\in C\cap D}\deg((v_px-u_py)^{I_p(C,D)})\\
      &=
      \sum_{p\in C\cap D} I_p(C,D).
    \end{align}
  \end{enumerate}
\end{myproof}

いま、「$A$で座標変換」と言ったとき、曲線全体を動かして条件を満たすようにすることを考えている。
つまり、「$A$が条件
\begin{align}
  (0,0,1)\notin C\cup D \cup \bigcup_{p\neq q \in C\cap D}L_{pq}
\end{align}
をみたす座標変換」ということは、「
\begin{align}
  (0,0,1) \notin A(C\cup D \cup \bigcup_{p\neq q \in C\cap D}L_{pq})
\end{align}
」ということである。
\begin{align}
  A(C) &=
  A(\var(f)) \\
  &=
  A\set{p; f(p)=0}\\
  &=
  \set{Ap; f(p)=0}\\
  &=
  \set{q; f(A\inv q)=0}\\
  &=
  \set{q; [f\circ (A\inv)](q)=0}\\
  &=
  \var(f\circ (A\inv)).
\end{align}
となっていることには注意する。

\begin{framed}
  補題11:
  定義8において、
  \begin{align}
    (0,0,1)\notin C\cup D \cup \bigcup_{p\neq q \in C\cap D}L_{pq}
  \end{align}
  をみたすどの座標変換行列を用いても、$p\in C\cap D$に
  対して同じ交わりの重複度$I_p(C,D)$を定める。
\end{framed}
\begin{myproof}
  \begin{enumerate}
    \item $l_{pq}$: $L_{pq}$の方程式を$l_{pq}$とする。
    \item $A$が座標変換の条件を満たすとは、
    \begin{align}
      A\inv(0,0,1) \notin C\cup D \cup \bigcup_{p\neq q \in C\cup D} L_{pq}
    \end{align}
    となることである。
    \item
    \begin{align}
      h=f\cdot g\cdot \prod_{p\neq q \in C\cap D}l_{pq}
    \end{align}
    とする。
    \item
    1,2,3より、$A$が座標変換の条件を満たすとは、
    \begin{align}
      A\inv(0,0,1) \notin \var(h)
    \end{align}
    である。
    \item
    1,2,3より、$A$が座標変換の条件を満たすとは、
    \begin{align}
      h(A\inv(0,0,1))\neq 0
    \end{align}
    である。
    \item $H$:
    $H\colon M_{3\times 3}(\C) \to \C$を
    \begin{align}
      H(B) = \det(B) \cdot h(B(0,0,1))
    \end{align}
    とする。
    \item
    行列が可逆になることと、その行列が0でないことは同値なので、
    6の記号を用いると、
    \begin{align}
      H(B) \neq 0
      &\iff
      \det(B)\neq 0 \quad かつ \quad h(B(0,0,1))\neq 0 \\
      &\iff
      Bは可逆 \quad かつ \quad h(B(0,0,q)) \neq 0.
    \end{align}
    \item
    上より、$M_{3\times 3}(\C)-\var(H)$と条件を満たす座標変換とは
    逆行列を取る操作で一対一である。
    \item $p_\bullet$:
    $C\cap D = \set{p_1,\dots,p_s}$とする。
    \item $u_{\bullet},v_\bullet,w_\bullet$:
    $B\in M_{3\times 3}(\C) - \var(H)$に対して、
    $B\inv(p_i) = (u_{i,B},v_{i,B},w_{i,B})$とする。
    \item $\forall B$: $B\in M_{3\times 3}(\C)- \var(H)$とする。
    \item $A$: $A=B\inv$とする。
    \item $\exists c_B, m_\bullet$:
    定理10の議論のように、$B(C),B(D)$について終結式を
    書く。このとき、
    \begin{align}
    A(C) = \var(f\circ B),\, A(D) = \var(g\circ B),\,
    A(p_i) = (u_{i,B},v_{i,B},w_{i,B})
    \end{align}
    に注意すると、
    \begin{align}
      \Res(f\circ B,g\circ B,z)
      =
      c_B\prod_{i=1}^s (v_{i,B}x-u_{i,B}y)^{m_{i,B}}.
    \end{align}
    となる。10より$B(u_{i,B},v_{i,B},w_{i,B}) = p_i$
    となっているので、$(u_{i,B},v_{i,B},w_{i,B})$は
    曲線$A(C) \cap A(D)$の点たちになっていることに注意する。
    \item
    $m_{i,B}$がすべての$B\in M_{3\times 3}(\C)-\var(H)$
    に対して同じ値をとる?
    \begin{enumerate}
      \item 斉次多項式$G(x,y)$と$H(x,y)$と$(u,v)\neq (0,0)$に対して
      \begin{align}
        G(x,y) = (vx-uy)^M H(x,y),\quad M\in \Zge
      \end{align}
      のとき、
      \begin{align}
        \frac{\pd^{i+j}G}{\pd x^i \pd y^j}(u,v)
        =
        \begin{cases}
          0 &; 0\le i+ j < M\\
          M! v^i (-u)^j H(u,v) &; i+j = M
        \end{cases}
      \end{align}
      となる。計算する。
      $G$にライプニッツルールを適用して微分していくうち、
      $(vx-uy)$がかかった項は$(u,v)$を代入して消えてしまう。
      $i+j < M$のときは$(vx-uy)$が残って消えて0になる。
      $i+j=M$のときには、$vx-uy$のほうを微分しきった項だけ残り、
      その項は$M! v^{i} (-u)^j H(x,y)$となる。

      特に、$H(u,v) \neq 0$なら$(u,v)\neq (0,0)$なので、
      $G$のある$M$階の偏微分係数が$(u,v)$では消えない。
      ($u=0$になってしまったら$i=0$にして0乗を考え、$1$にしてしまう。)

      \item
      $M_{3\times 3}(\C)$の間の距離を、
      $B=(b_{ij}),\, C=(c_{ij}) \in M_{3\times 3}(\C)$について
      \begin{align}
        d(B,C)
        =
        \sqrt{\sum_{i,j=1}^3 \myabs{b_{ij}-c_{ij}}^2}
      \end{align}
      と定義する。
      \item
      上の定義から明らかに、$F\colon M_{3\times 3}(\C) \to \C$
      が、行列の9要素でできた多項式写像ならば、これは(b)で定義した距離のもとで
      連続である。
      \item
      上より、多項式写像$F\colon M_{3\times 3}(\C) \to \C$と
      $F(B_0) \neq 0$なる
      $B_0 \in M_{3\times 3}(\C)$について、
      $B \in M_{3\times 3}(\C)$を$B_0$に十分(上の距離で)近付けることで
      $F(B) \neq 0$とできる。
      \item
      さらに、有理関数についても、定義されていれば同様のことが言える。
      \item $\forall B_0,i$: $B_0 \in M_{3\times 3}(\C) - \var(H)$とする。7より、$B_0 \in GL(3,\C)$である。
      \item $M$: $M=m_{i,B_0}$
      \item $B \in M_{3\times 3}(\C)$が$B_0$に十分近いなら$m_{i,B} =  M$である。
      \begin{enumerate}
        \item $\exists M階の偏微分$: (a)より、$\Res(f\circ B_0, g\circ B_0,z)$の
        $M$階の偏微分には$(u_{i,B_0},v_{i,B_0})$で消えていないものがある。
        ($\Res(f\circ B_0, g\circ B_0,z)$は$(v_{i,B_0}x-u_{i,B_0}y)^{M}\cdot (ほか)$であるから、これに適用する。)
        \item
        13で
        \begin{align}
          (u_{i,B},v_{i,B},w_{i,B}) &=
          B\inv \cdot p_i,\\
          \Res(f\circ B, g\circ B,z)
          &=
          c_B \prod_{i=1}^s (v_{i,B}x-u_{i,B}y)^{m_{i,B}}.
        \end{align}
        \item
        上より、$\Res(f\circ B, g\circ B, z)$の$M$階の偏微分に
        $(u_{i,B},v_{i,B})$を代入したものは($p_i$の要素は定数とみなして)、
        $B$の9要素の有理関数となる。分母は$B$の行列式しか出ず、
        11より$B\in GL(3,\C)$なので$M_{3\times 3}(\C) - \var(H)$
        全体で定義されている。
        \item
        上と(e)の有理関数の連続性、それにiより、
        $B$が$B_0$に十分近ければ$\Res(f\circ B,g\circ B,z)$
        のiで取った$M$階の偏微分は$(u_{i,B},v_{i,B})$でゼロでない。
        \item
        仮に$m_{i,B} > M$であるとする(背理法)。
        \item
        14より、$m_{i,B}$階未満の$\Res(f\circ B, g\circ B, z)$の偏微分は
        $(u_{i,B},v_{i,B})$で消える。
        \item
        vと上より$M$階の$\Res(f\circ B, g\circ B, z)$の偏微分は
        $(u_{i,B},v_{i,B})$で消えるが、これはivに矛盾する。
        \item
        vおわり: 上より、$m_{i,B}\le M = m_{i,B_0}$である。
        \item
        定理10のベズーの定理は座標を固定したまま証明できるので、
        ある座標にだけ定義された「仮の」重複度のまま利用できる。そこで、
        viiiの和を$i$にわたって取り、
        \begin{align}
          mn = \sum_{i=1}^s m_{i,B} \le \sum_{i=1}^s m_{i,B_0} = mn.
        \end{align}
        \item
        上より、各$i$について$m_{i,B} = m_{i,B_0}$である。
      \end{enumerate}
      よって、$B$が$B_0$に十分近ければ各$i$について$m_{i,B} = m_{i,B_0}$である。
    \end{enumerate}
    \item
    上より、$B\mapsto m_{i,B}$は局所定数である。
    \item
    $M_{3\times 3}(\C)-\var(H)$は弧状連結?
    \begin{enumerate}
      \item $f\in \C[x]$が0でないなら、$\C-\var(f)$は弧状連結である。これは、
      $\var(f)$が有限集合であることから明か。
      \item $f\in \C[x_1,\dots,x_n]$が0でないなら、
      $\C^n - \var(f)$は弧状連結である。
      $a,b\in \C^n$とし、それを結ぶ複素直線$t\mapsto ta + (1-t)b$を考える。
      $t\mapsto f(ta+(1-t)b)$の像は$\C^n - \var(f)$の部分集合である。
      $t\mapsto ta+(1-t)b$は$\C$内の1変数の多様体とみなせるので、
      (a)より$\C^n - \var(f)$の$a$と$b$を結ぶpathが得られる。
      \item 上を、$\C^9$でのことと考えて$M_{3\times 3}(\C) - \var(H)$
      に適用すればOK。
    \end{enumerate}
    \item
    15と16より、$B\mapsto m_{i,B}$は
    弧状連結な集合上の局所定数関数なので、
    定数関数である。
  \end{enumerate}
\end{myproof}

\begin{framed}
  定理12(パスカルの神秘の六角形):
  既約2次曲線上の6点に対し、それらを結ぶ6本の直線を上のように決める。
  すると3組の対置する2本の直線の交点は、ある1本の直線上にある。
\end{framed}
\begin{myproof}
  \begin{enumerate}
    \item $C$: 2次曲線。
    \item $p_\bullet$: $p_1,\dots,p_6$を$C$上の6点とする。
    \item $L_\bullet$: $L_i$を$p_i$と$p_{i+1}$を繋ぐ直線とする。
    \item $C_1$: $C_1 = L_1 \cup L_3 \cup L_5$とする。
    これは3本の直線の$\cup$なので、3次曲線になっている。
    \item $C_2$: $C_2 = L_2 \cup L_4 \cup L_6$とする。
    これは3本の直線の$\cup$なので、3次曲線になっている。
    \item
    ベズーの定理と4,5より、$C_1 \cap C_2$は
    重複度を含めて9個の点からなる。
    \item
    $C_1 \cap C_2 = \set{p_1,\dots,p_6}$である。
    なぜなら、
    \begin{align}
      C_1 \cap C_2
      =
      (L_1 \cup L_3 \cup L_5)\cap (L_2 \cup L_4 \cup L_6)
      =
      \set{p_1,\dots,p_6} \cup (L_1\cap L_4)\cup  (L_2 \cap L_5) \cup (L_3 \cap L_6)
    \end{align}
    となっている。
    \item
    上の$C_1\cap C_2$の9つの点が全て異なるので、これらの重複度は1である。
    \item
    $f$: $C=\var(f)$とする。
    \item
    $g_1$: $C_1 = \var(g_1)$とする。
    \item
    $g_2$: $C_2 = \var(g_2)$とする。
    \item
    1と9より$f$の全次数は2。
    \item
    4,5,10,11より、$g_1,g_2$の全次数は3。
    \item
    $\forall p$ : $p\in C$を$p_1,\dots,p_6$と異なる点とする。
    \item
    $g_1(p)$と$g_2(p)$は0ではない。
    なぜなら、$g_1(p)=0$とすると$p\in C_1$となり、
    $p\in C_1 \cap C$となる。ベズーの定理より$C_1 \cap C$は重複度込みで
    6点からなる。$\set{p_1,\dots,p_6}\subset C_1 \cap C$なので、
    $\set{p_1,\dots,p_6} = C_1 \cap C$となる。よって、$p$は$p_1,\dots,p_6$のいずれかになるが、
    これは矛盾である。よって、$g_1(p) \neq 0$である。同様に、$g_2(p) \neq 0$となる。
    \item $g$:
    $g=g_2(p)g_1 - g_1(p)g_2$とする。これは13より3次になる。(斉次なので先頭項だけ消えるということもない。)
    \item
    上より、$g$は$p,p_1,\dots,p_6$で消える。
    \item
    $g_1$と$g_2$は1次独立である。なぜなら、1次従属だとすると$C_1 = \var(g_1) = \var(g_2) = C_2$となってしまうから。
    \item
    上と15と16より、$g$は0でない。
    \item
    17より$\var(g)$は$C$と7点以上で交わる。
    \item
    ベズーの定理を$\var(g)$と$C$とに適用すると、上より
    「$g$と$f$が被約であり、しかも共通の成分を持たないならば、
      $\sum_{q \in \var(g)\cap C} 1 = 6$」となる。
    \item
    20,21より、
    $g$か$f$かが被約でないか、あるいは$\var(g)$と$C$とが共通の既約成分を持つことになる。
    \item
    $f$は被約にしておいたので、上より
    $g$が被約でないか、あるいは$g$と$f$とが共通の成分を持つことになる。
    \item
    $\var(g)$と$C$は共通の既約成分を持つ?
    \begin{enumerate}
      \item $\var(g)$と$C$が共通の既約成分を持たないとする。(背理法)
      \item
        上と23より、$g$は被約でないということになる。
      \item
        16より、$g$が3次なので$\var(g)$は2次以下の多項式で定義される。
      \item
        (a)(b)(c)と$C$を定義する$f$が被約であることから、
        ベズーの定理より$\var(g)\cap C$は高々4点になる。
      \item
      上は20と矛盾する。
    \end{enumerate}
    よって、$\var(g)$と$C$は共通の既約成分を持つ。
    \item
    $C$は既約なので、上より$C=\var(f)$が$\var(g)$の既約成分である。
    \item $\exists l$:
    上より、$g=f\cdot l$となる全次数1の斉次多項式$l$がある。
    \item
    4,10,5,11より、$g_1$は$L_1,L_3,L_5$を消し、
    $g_2$は$L_2,L_4,L_6$を消すので、16より$g$は
    $L_1\cap L_4,\, L_2 \cap L_5,\, L_3 \cap L_6$を消す。
    \item
    $L_1 \cap C$はベズーの定理より2個しか点がなく、
    それらは$p_1,p_2$なので、$L_1\cap L_4$は$L_1\cap C$上にない。
    \item
    上同様の考察を$L_4\cap C$に対してもして、$L_1 \cap L_4$は
    $L_4\cap C$上にない。
    \item
    28,29より、$L_1\cap L_4$は$C$上にない。
    \item
    28-30と同様の考察をして、
    $L_2\cap L_5$と$L_3\cap L_6$も$C$上にない。
    \item
    30,31より$C$を定義する$f$は$L_1\cap L_4,\, L_2\cap L_5,\, L_3\cap L_6$
    を消さない。
    \item
    26,27,32より、$L_1\cap L_4,\, L_2 \cap L_5,\, L_3\cap L_6$は
    $g$で消えるのに$f$で消えない。よって、
    $l$は$L_1\cap L_4,\, L_2 \cap L_5,\, L_3\cap L_6$
    を消す。
    \item
    16より$g$の次数は3で、仮定より$f$の次数は2なので、26より$l$の次数は1になっている。
    よって、$\var(l)$は射影直線になっている。
    \item
    33,34より、$L_1\cap L_4,\, L_2\cap L_5,\, L_3\cap L_6$は同一直線$l$上である。
  \end{enumerate}
\end{myproof}
