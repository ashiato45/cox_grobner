\label{sec:消去理論}
\subsection{消去および拡張定理}
\label{sub:消去および拡張定理}
グレブナ基底をlex順序で計算すると、変数の消去が起こることをみた。このことを示す。
そのために、「消去イデアル」を定義する。
$k[x_1,\dots,x_n]$のイデアル$I$について、「$I$の$l$次の消去イデアル$I_l$」を
$I_l \defeq I \cap k[x_{l+1},\dots,x_n]$と定める。これが\warn{$k[x_1,\dots,x_n]$の}イデアルになっていることを示す必要はある。
\begin{myproof}
  $I$は$k[x_1,\dots,x_n]$のイデアルであり、$k[x_{l+1},\dots,x_n]$は$k[x_{l+1},\dots,x_n]$のイデアルなので、イデアルの交わりがイデアルになることは使えない。個別にイデアルの条件を示す必要がある。
  \begin{itemize}
    \item 和で閉じる:
    $f, g\in I_l$とする。$f,g \in I$なので、$f+g \in I$となる。
    また、$f,g \in k[x_{l+1},\dots,x_n]$なので、$f+g \in k[x_{l+1},\dots,x_n]$となっている。よって、$f+g \in (I\cap k[x_{l+1},\dots,x_n]) = I_l$となっている。
    \item 積で飲み込む:$f \in I_l$とし、$g\in k[x_{l+1},\dots,x_n]$とする。
    $gf \in I_l$であることを示す。$f\in k[x_{l+1},\dots,x_n]$なので、
    $gf \in k[x_{l+1},\dots,x_n]$となっている。また、$I$がイデアルであり$f\in I$
    なので、$gf\in I$となっている。よって、$gf\in I_l$となっている。
  \end{itemize}
\end{myproof}
さらに、$l$次の消去イデアル$I_{l}$の$1$次の消去イデアル$(I_l)_1$は$I$の$l+1$次の消去イデアル$I_{l+1}$になっている:$(I_l)_1 = I_{l+1}$である。
\begin{myproof}
  $I_l = I \cap k[x_{l+1},\dots,x_n]$であり、
  $I_{l+1} = I\cap k[x_{l+2},\dots,x_n]$であり、
  $(I_l)_1 = I_l \cap k[x_{l+2},\dots,x_n]$である。よって、
  \begin{align}
    (I_l)_1
    &=
    I_l \cap k[x_{l+2},\dots,x_n]\\
    &=
    (I\cap k[x_{l+1},\dots,x_n]) \cap k[x_{l+2},\dots,x_n]\\
    &=
    I \cap k[x_{l+1},\dots,x_n]\\
    &=
    I_l.
  \end{align}
  示された。
\end{myproof}
つまり、高次の消去イデアルを考えたいときには、1次ずつ消去イデアルを計算すればよいことがわかった。

消去イデアルはその定義から、イデアル$I$のうち文字を消したもののあつまりであり、
$G$を$I$の基底とするなら、この$G$をたしひきかけ算して文字を消したもののあつまりとなっている。
$\var(I)$を考えると、これに属する点は$I$の式を0にしなくてはならず、特に
$I_l$の式を0にしなくてはならない。これは、$\var(I)$に点が属するには
$k[x_{l+1},\dots,x_n]$のなかではどうでなければならないかという必要条件を与える。
$I_l$の式を0にするときを考えるには$I_l$の基底がわかっていれば必要十分なので、
$I_l$の基底を求める方法を知りたいが、これにはGroebner基底が便利である。
次のことが言える。これを消去定理とよぶ。
「$G$を$I\subset k[x_1,\dots,x_n]$のlex順序でのGroebner基底とすると、
$G\cap k[x_{l+1},\dots,x_n]$は$I_l$のGroebner基底となる。」
\begin{myproof}
  $G\cap k[x_{l+1},\dots,x_n] \subset I_l$なので、
  $\gen{\LT(G\cap k[x_{l+1},\dots,x_n])} = \gen{\LT(I_l)}$
  となることを示せばよい。この$\subset$は、$G\cap k[x_{l+1},\dots,x_n]\subset I_l$は自明なので、生成元$\LT(I_l)$が$\gen{\LT(G\cap k[x_{l+1},\dots,x_n])}$に包まれることを示せばよい。$f\in I$なので、$\LT(f)\in \LT(I) \subset \gen{\LT(I)} = \gen{\LT(g_1),\dots,\LT(g_s)}$となっていて、
  $\LT(g_i) | \LT(f)$となる$i$が存在する。示したいのは
  $\LT(f)$が$\LT(G\cap k[x_{l+1},\dots,x_n])$のどれかで割り切れることであり、$\LT(G)$
  である$\LT(g_i)$で割り切れることは示したので、あとは$g_i \in k[x_{l+1},\dots,x_n]$を示せばよい。

  $f\in k[x_{l+1},\dots,x_n]$なので、
  $\LT(f) \in k[x_{l+1},\dots,x_n]$である。多項式順序の性質から、
  $\LT(g_i) \le \LT(f)$であり、いまはlex順序を採用しているので、
  $\LT(g_i) \in k[x_{l+1},\dots,x_n]$となる。さらにlex順序を採用しているので、
  $g_i \in k[x_{l+1},\dots,x_n]$となり、$g_i \in G \cap k[x_{l+1},\dots,x_n]$となる。
  よって、$g_i \in G\cap k[x_{l+1},\dots,x_n]$であり、
  $\LT(g_i) | \LT(f)$であり、$\LT(f)\in \gen{\LT(G\cap k[x_{l+1},\dots,x_n])}$となっている。

\end{myproof}
これで消去のほうは議論できた。あとは後退代入に相当するところを考える。

文字を消した結果の式を満たすことは、多様体に点が属することの必要条件でしかない。
それを満たす点のことを部分解という。つまり、「$(a_{l+1},\dots,a_n) \in k[x_{l+1},\dots,x_n]$が$\var(I)$の部分解である」とは、
「$(a_{l+1},\dots,a_n) \in \var(I_l)$となる」ことである。
したがって、多様体に属するように整合性が取れるように他の点が取れるかどうかは分からない。そのような拡張ができるための十分条件として、次の拡張定理がある。
「
体は$k=\C$で考えることにする。
$(a_{2},\dots,a_n)\in \var(I_1)$を部分解とする。
$I$のGroebner基底を$G$とする。
$k[x_1,\dots,x_n]$の多項式について、その多項式を$(\C[x_2,\dots,x_n])[x_1]$の元、すなわち$x_1$だけを不定元とみなした多項式とみなしたときの最高次の係数を$\LC'$とよぶことにする。ただし、$\LC'(0)$は考えないことにする。この条件のもとで、
\begin{align}
  (a_2,\dots,a_n) \notin \var(\LC'(G))
  \implies
  (a_1,a_2,\dots,a_n) \in \var(I)となるa_1(\in \C) が存在する
\end{align}
となる。証明は後の節でやる。
先に、$(I_l)_1 = I_{l+1}$であることは示したので、必要なら繰り返し使えばよい。

ここで、2つの特徴的な条件がある。
\begin{itemize}
  \item 体を$\C$にしていること:$x^2 = z,\, x^2 = y$を$\R$上で考えて先の定義をナイーブに適用すると、$x$を消去した$y=z$上、つまり$(a,a)$は、
  $\var(\LC'(x^2-z),\LC'(x^2-y))=\var(1,1)=\emptyset$に入らない限り、
  つまりいつでも拡張できるということになるが、実際は$a\ge0 $のときだけ拡張できる。
  \item $(a_2,\dots,a_n) \notin \var(\LC'(G))$としていること:
  $xy=1,\, xz=1$を考える。
  \insertcalc{calc_3_1.tex}
  という計算で、このGroebner基底が$y-z,\, xz-1$である。よって、$I_1 = \gen{y-z}$である。よって、$\var(I_1) = \set{(a,a); a\in \C}$となる。
  よって、これをナイーブに拡張すると、$(1/a,a,a)$となる。

  ここで先の条件を考えてみる。
  $\LC'(xz-1)=z$なので、拡張できるための十分条件として$(a,a) \notin \var(z)$が得られる。つまり、拡張できないかもしれない場合というのは、$(a,a) \in \var(z)$になる。このときというのは、$a=0$のときである。このときは実際、$1/a$が考えられない。
  また図を考えて、$y=z,\, xz=1$というときを考える。これは、
  平面$y=z$と双曲線$xz=1$を$y$方向に延ばしたやつの共有点全体だが、
  この点のうち$z=0$となっているものはあきらかに存在しない。

\end{itemize}

$\LC'(g_1),\dots,\LC'(g_s)$のうち定数があったときはあきらかに$\var(\LC'(g_1),\dots,\LC'(g_s)) = \emptyset$となるので、
部分解全体が拡張できることが保証される。つまり、系として「
体は$k=\C$とする。部分解$(a_2,\dots,a_n)\in \var(I_1)$
があったとする。さらに、$G$を$I$のGroebner基底とし、
$\LC'(G)$のうち(当然非0の)定数があったとすると、
部分解$(a_2,\dots,a_n)$は常に$(a_1,\dots,a_n) \in \var(I)$に
拡張できる。」が得られる。仮に$g_1,\dots,g_s$に定数があったとすると、元の
$\gen{I}$が全体集合になり、$\var(I)$は空集合になる。このときは、拡張もなにもなくなってしまうので自明に正しい。また、仮に$g_1,\dots,g_s$に0があったとすると、
そのような0は外しておけばよいので考える必要がない。このときは$\LC'$を考えることができなくなってしまう。

\subsection{消去の幾何}
\label{sub:消去の幾何}
頭$l$個落とす写像$\pi_l\colon \C^n \to \C^{n-l}$を射影写像(projection map)という。
すると、消去イデアルとについて、次の関係がある。「
$f_\bullet \in k[x_1,\dots,x_n]$とする。
\begin{align}
  \pi_l(\var(f_1,\dots,f_s)) \subset \var(\gen{f_1,\dots,f_s}_l)
\end{align}
となる。言い換えるなら、多様体の($l$次の)射影は($l$次の)部分解に包まれる。
」
\begin{myproof}
$(a_1,\dots,a_n)\in \var(f_1,\dots,f_s)$とする。
$f_\bullet(a_1,\dots,a_n) = 0$となっている。
$\pi_l(a_1,\dots,a_n) = (a_{l+1},\dots,a_n)$である。

一般に、$f\in k[x_{l+1},\dots,x_n]$のとき、これを$f\in k[x_1,\dots,x_n]$とみなすと、
$f(\xi_1,\dots,\xi_n)$は$\xi_1,\dots,\xi_{l}$の値に依存せず、$\xi_{l+1},\dots,\xi_n$の値のみによって定まる。これは、$f\in k[x_1,\dots,x_n]$ではあるが、
$k[x_{l+1},\dots,x_n]$からの埋め込みだったので、式のなかに$x_1,\dots,x_n$の文字があらわれず、これらに対応する値$\xi_1,\dots,\xi_{l}$に値が依存しないからである。
よって、$f(\pi(\xi_{1},\dots,\xi_{n}))$と$\pi_l$による同値類で定めれば、これはwell-definedである。

$f\in \gen{f_1,\dots,f_s}_l$とする。
$f\in k[x_{l+1},\dots,x_n]$なので、先の考察より$f$のとる値は$\pi_l$の同値類で定ま
り、
\begin{align}
  f(a_{l+1},\dots,a_n) = f(\pi_l(a_1,\dots,a_n)) = f(0,\dots,0,a_{l+1},\dots,a_n)
\end{align}
である。また、$f\in \gen{f_1,\dots,f_s}$なので、$f = \sum_i h_i f_i$となる
$h_\bullet \in k[x_1,\dots,x_n]$が存在し、
\begin{align}
  f(a_{l+1},\dots,a_n)
  &\desceq{頭$l$個はなんでもいい(well-defined)}
  f(a_1,\dots,a_n)\\
  &=
  \sum_{i=1}^s h_i(a_1,\dots,a_n) \ub{f_i(a_1,\dots,a_n)}_{=0,\,はじめの設定}\\
  &=
  0.
\end{align}
\end{myproof}
言い換えるなら、多様体の射影は、部分解のうち拡張できるもの全体に一致する(そりゃそうだ、射影が部分解をはみ出ることがないというほうが重要情報っぽい。)。
例えば、$(y=z,xy=1)$を考えると、これの射影$\pi_1(\var(y-z,xy-1))$は$\set{(a,a); a\neq 0}$であり、消去イデアルのなす多様体は$\var(\gen{y-z})$になって、$\set{(a,a)}$になる。

ただし、多様体の射影がかならず多様体になるとは限らない。実際先の例だと、
$\pi_1(\var(y-z,xy-1))= \set{(a,a); a\neq 0}$であり、これは多様体でない。
この状況を考えるために、次の分解を用意しておく。
「$f_1,\dots,f_s \in k[x_1,\dots,x_n]$について、
\begin{align}
  \var(f_1,\dots,f_s)
  =
  \pi_1(\var(f_1,\dots,f_s))
  \cup
  (\var(f_1,\dots,f_s) \cap \var(\LC'(f_1),\dots,\LC'(f_s))).
\end{align}
となる。」
\begin{myproof}
  \begin{itemize}
    \item $\supset$:
    $a=(a_1,\dots,a_n)$とする。$a\in \pi_1(\var(f_1,\dots,f_s))$のときは、
    先の「多様体の射影は部分解に含まれる」より、$a\in \var(f_1,\dots,f_s)$となる。
    $a\in \var(f_1,\dots,f_s)\cap \var(\LC'(f_1),\dots,\LC'(f_s))$
    のときは自明に$a\in \var(f_1,\dots,f_s)$となる。
    \item $\subset$:
    $a\in \var(f_1,\dots,f_s)$とする。
    $a\notin \var(f_1,\dots,f_s)\cap \var(\LC'(f_1),\dots,\LC'(f_s))$であるとする。
    このときは、$a\in \var(f_1,\dots,f_s)$なので、$a\notin \var(\LC'(f_1),\dots,\LC'(f_s))$なので、先の拡張定理により$a\in \pi_1(f_1,\dots,f_s)$となる。
  \end{itemize}
\end{myproof}

正確に多様体の射影と部分解との関係を記述するものとして、閉包定理がある:「
$f_1,\dots,f_s \in k[x_1,\dots,x_n]$として、
\begin{enumerate}[label=(\alph*)]
  \item $\pi_l(\var(f_1,\dots,f_s))$を包む最小の多様体は$\var(\gen{f_1,\dots,f_s}_l)$である。
  \item $\var(f_1,\dots,f_s)\neq \emptyset$とする。
  $\pi_l(\var(f_1,\dots,f_s))$は、多様体$\var(\gen{f_1,\dots,f_s}_l)$から、これに真に包まれる多様体$W$を削ったものを包む:
  \begin{align}
    \Exists{W(:多様体,\, \subsetneq \var(\gen{f_1,\dots,f_s}_l))} \ub{\var(\gen{f_1,\dots,f_s}_l)-W}_{\neq \emptyset} \subset \pi_l(\var(f_1,\dots,f_s))
  \end{align}
  となる。
\end{enumerate}
」
\begin{myproof}
  (b)の$l=1$のときだけを証明する。$\var(f_1,\dots,f_s)$に関して、
  条件を満たす、この多様体に真に含まれる多様体を探す。
  \begin{algorithm}[H]
    \caption{削る多様体を探す}
    \begin{algorithmic}[1]
      \STATE{$list \Leftarrow [f_1,\dots,f_s]$}
      \STATE{$stop \Leftarrow false$}
      \WHILE{$stop = false$}
      \IF{$list \subset k[x_2,\dots,x_n]$}
        \STATE{$stop \Leftarrow true$}
        \STATE{$W \Leftarrow \emptyset$}
      \ELSE{}
        \STATE{$W \Leftarrow \var(\gen{list}_1)\cap \var(\LC'(list))$}
        \IF{$\var(\gen{list}_1)- W \neq \emptyset$}
          \STATE{$stop \Leftarrow true$}
        \ELSE{}
          \STATE{$list \Leftarrow [x\mapsto x-\LT'(x)](list) + \LC'(list)$}
        \ENDIF{}
      \ENDIF{}
      \ENDWHILE{}
    \end{algorithmic}
  \end{algorithm}
  ただし、$\LT'$は、$k[x_2,\dots,x_n][x_1]$とみなしたときの先頭項とする。
  \begin{itemize}
    \item アルゴリズムは停止する:
    L.3の停止条件から、L.3~L.15のループが1回実行されるごとに、必ず
    L.12が実行される。この行について、
    $[x\mapsto x-\LT'(x)](list)$は、$k[x_2,\dots,x_n][x_1]$での先頭項を消しており、
    $\LC'(list)\subset k[x_2,\dots,x_n]$なので、$list$の最高の($k[x_2,\dots,x_n][x_1]$での)次数は、$0$より大きければ真に減少する。

    このことから、$list$の次数はあるところで
    $0$に到達する。つまり、$list$の元がどれも$k[x_2,\dots,x_n]$に属することになる。
    すると、その次のループのなかで、L.4の条件が真となり、L.5で$stop = true$となるので、L.4の停止条件を満たすようになり、アルゴリズムは停止する。
    \item $\var(list)$は変わらない:
    $list$が変化するのはL.12でのみであり、このときには、
    \begin{enumerate}[label=(\alph*)]
      \item  $list_b \not\subset k[x_2,\dots,x_n]$
      \item $W_a = \var(\gen{list_b}_1)\cap \var(\LC'(list_b))$
      \item $\var(\gen{list_b}_1) - W_a = \emptyset$
      \item $list_a = [x\mapsto x-\LT'(x)](list_b) + \LC'(list_b)$
    \end{enumerate}
    となっている。

    まず、
    $\var(list_b) = \var(list_b + \LC'(list_b))$を示す。
    (b),(c)
    より、
    \begin{align}
      \var(\gen{list_b}_1) \subset W_a =
      \var(\gen{list_b}_1) \cap \var(\LC'(list_b))
      \subset
      \var(\LC'(list_b))
    \end{align}
    $\gen{list_b} \supset \gen{list_b}_1$なので、
    $\var(list_b) \subset \var(\gen{list_b}_1)$である。よって、
    \begin{align}
      \var(list_b) \subset \var(\LC'(list_b))
    \end{align}
    である。よって、
    \begin{align}
      \var(list_b) = \var(\LC'(list_b))\cap \var(list_b) =
      \var(\LC'(list_b) + list_b)
    \end{align}
    である。

    そして、
    \begin{align}
      \gen{\LC'(list_b) + list_b} =
      \gen{\LC'(list_b) + ([x\mapsto x-\LT'(x)](list_b)) }
      = \gen{list_a}
    \end{align}
    なので、
    \begin{align}
      \var(list_a) = \var(\LC'(list_b) + list_b)
      =
      \var(list_b)
    \end{align}
    である。

    \item $\pi_1(\var(list))$は変わらない:
    $\var(list)$が変わらないことから直ちに従う。
    \item $\var(\gen{list}_1)$は変わらない:
    $list$が変化する、すなわちL.12が実行されるときを考えればよく、
    「$\var(list)$は変わらない」の状況と同じとしてよい。
    閉包定理より、$\var(\gen{list_a}_1)$は$\pi_1(\var(list_a))$を包む最小の多様体である。また、$\var(list_b)$は$\pi_1(\var(list_b))$を包む最小の多様体である。しかし、先に示したことより、「$\pi_1(\var(list))$は変わらない」ので、
    $\pi_1(\var(list_b))=\pi_1(\var(list_a))$である。よって、
    $\var(\gen{list_a}_1)$も$\var(\gen{list_b}_1)$も同じ多様体
    $\pi_1(\var(list_b))=\pi_1(\var(list_a))$を包む最小の多様体なので、
    \begin{align}
      \var(\gen{list_a}_1) = \var(\gen{list_b}_1)
    \end{align}
    である。
    \item 停止時点で、$W\subsetneq \var(\gen{f_1,\dots,f_s}_1)$となり、$W$は多様体である。
    \begin{itemize}
      \item 停止直前に実行されたのがL.5である:
      このとき$W=\emptyset$なのであきらか。
      \item 停止直前に実行されたのがL.10である:
      このとき$W=\var(\gen{list}_1) \cap \var(\LC'(list))$なので、多様体ではある。

      さらに、$W$のこの式より、$W \subset \var(\gen{list}_1)$であることも保証される。

      最後に、$W\neq \var(\gen{list}_1)$であることを示せばよいが、そうだとするとL.9の条件が通過できず矛盾する。
    \end{itemize}
    \item 停止時点で、$W$は$\var(\gen{f_1,\dots,f_s}_1)-W \subset \pi_1(\var(f_1,\dots,f_s))$となる。
    \begin{itemize}
      \item 停止直前に実行されたのがL.5であるとき:
      $list\subset k[x_2,\dots,x_n]$となっている。
      よって、$\gen{list}_1 = \gen{list}$となる。
      よって、$\var(list)$は$x_1$を使わずに定義されていることわかり、
      どの部分解$\var(\gen{list}_1)$も、拡張できること、すなわち
      $\var(\gen{list}_1) = \pi_1(\gen{list})$がわかる。
      これまで示してきた不変より、
      \begin{align}
        \var(\gen{f_1,\dots,f_s}_1)-W
        &=
        \var(\gen{f_1,\dots,f_s}_1)-W\\
        &=
        \var(\gen{list}_1)-W\\
        &=
        \var(\gen{list}_1)\\
        &=
        \pi_1(\var(list))\\
        &=
        \pi_1(\var(f_1,\dots,f_s)).
      \end{align}
      また、$W$はあきらかに$\var(\gen{f_1,\dots,f_s}_1)$に含まれる多様体であり、満たされた。
      \item 停止直前がL.10のとき:
      $list$に関して、部分解の分解を考えると、$W= \var(\gen{list}_1) \cap \var(\LC'(list))$となるから、
      \begin{align}
        \var(\gen{list}_1)
        =
        \pi_1(\var(list)) \cup W
      \end{align}
      となる。よって、$\var(\gen{list}_1)-W\subset \pi_1(\var(list))$となる。
      これまで示してきたことより、
      \begin{align}
        \var(\gen{f_1,\dots,f_s}_1)-W
        &=
        \var(\gen{list}_1)-W\\
        &\subset
        \pi_1(\var(list))\\
        &=
        \pi_1(\var(f_1,\dots,f_s)).
      \end{align}

    \end{itemize}
  \end{itemize}
\end{myproof}
(a)は射影を多様体で上から抑え、(b)は多様体の差で下から抑えている。

この定理だと$\pi_1(\var(f_1,\dots,f_s))$が正確にどういう形をしているかは分からない。
実は
\begin{align}
  \pi_1(\var(f_1,\dots,f_s)) = \bigcup_{i=1}^t (A_i - B_i)
\end{align}
となる多様体$A_i,B_i$が存在する、つまり多様体の射影は多様体の差の和で書けることがわかり、このような(?)集合を構成可能という。あとでやる。

この節で$\pi$の記号を整備して、先の節での系を幾何学的に言い直すことができる。
すなわち:「$\var(g_1,\dots,g_s)$について、$\LC'(g_1),\dots,\LC'(g_s)$のうちで
定数(当然非0)があるならば、$\pi_1(\var(g_1,\dots,g_s)) = \var(\gen{g_1,\dots,g_s}_1)$となる。」$g_1,\dots,g_s$のうち0があるような場面は、その0を外しておけるので考える必要がない。このときは$\LC'$を考えることができなくなってしまう。また、そもそも非0の定数があったときには、多様体は空集合をあらわすようになる。このときは射影しても空でありやはり成立している。

\subsection{陰関数表示化}
\label{sub:陰関数表示化}
パラメタ表示された図形、つまり関数の像を陰関数表示することを考える。
ここで、陰関数表示とは、パラメタ表示された図形を包む最小のアフィン多様体を求めることである。図形を包む最小のアフィン多様体ということで、先の閉包定理を利用したいが、そのためにパラメタ表示された図形というのを何かのアフィン多様体の射影として表現できると便利である。そこで、グラフを考える。

$k^n$中で多項式で表示された図形を考える。この図形は、$F=(f_1,\dots,f_n)\colon k^m \to k^n$であらわされており、$f_\bullet \in k[t_1,\dots,t_m]$とする。
グラフへの埋め込み$i\colon k^m \to k^{m+n}$を、
\begin{align}
  i(t_1,\dots,t_m) = (t_1,\dots,t_m)\oplus F(t_1,\dots,t_m)
\end{align}
と定義する。この$i(k^m)$は$k[t_1,\dots,t_m,x_1,\dots,x_n]$のアフィン多様体であって、
\begin{align}
  i(k^n) = \var(f_1-x_1,\dots,f_n-x_n)
\end{align}
である。なぜならグラフは、$t_\bullet$のスロットには$t_\bullet$がそのまま入っていてほしいし、$x_\bullet$のスロットには$f(t_1,\dots,t_m)$が入っていてほしいからである。$x_\bullet$が$F$の値域側の文字であることには注意する。ただしここで、$x_\bullet$は
\begin{align}
  x_i(t_1,\dots,t_m,x_1,\dots,x_n) = x_i \quad (i=1,\dots,n)
\end{align}
という関数$x_\bullet \colon k^{m+n} \to k$であり、
$f_\bullet$は$f_\bullet \in k[t_1,\dots,t_m]$でもあるが、
\begin{align}
  f_i(t_1,\dots,t_m,x_1,\dots,x_n) = f_i(t_1,\dots,t_m)
\end{align}
とみなしている。
そして、グラフの射影はいままで通り、$\pi_m$を考える。すると、
\begin{align}
  \pi_m(t_1,\dots,t_m,x_1,\dots,x_n) = (x_1,\dots,x_n)
\end{align}
であるから、$F=\pi_m\circ i$となった。よって、$F(k^n) = \pi_m\circ i(k^n)=\pi_m(i(k^n))$となり、パラメタ付けされた図形$F(k^n)$は、
アフィン多様体$\var(x_1-f_1,\dots,x_n-f_n)=i(k^n)$の射影
$\pi_m(i(k^n))$としてあらわせた。これで閉包定理を使う準備ができた。

多項式でパラメタ表示された図形を陰関数表示する手法として、次がある:「
無限体$k$上の図形の多項式によるパラメタ表示$F=(f_1,\dots,f_n)\colon k^m \to k^n$について、$F(k^m)$を包む最小のアフィン多様体は$\var(\gen{\ub{x_1-f_1}_{k[t_1,\dots,t_m,x_1,\dots,x_n]},\dots,x_n-f_n}_m)$
である。」
\begin{myproof}
$k$が代数的閉体であるときには、$F(k^m) = \pi_m(i(k^m)) = \pi_m(\var{\gen{f_1-x_1,\dots,f_n-x_n}})$を包む最小の多様体を考えればよいが、
閉包定理よりこれは$\var(\gen{f_1-x_1,\dots,f_n-x_n}_m)$である。

$k$を包む代数的閉体$K$が存在するので、これをかんがえる(体論)。
以降、係数を$k$とするイデアルを$\gen{\bullet}_k$と書き、多様体を$\var_k(\bullet)$と書く。また、$K$についても同様とする。
$F(k^n)$を包む最小の多様体を$Z_k$とする。このとき、$\pi_m(i(k^n)) \subset Z_k$であることを示そう。
\end{myproof}

やりなおし。$I=\gen{x_1-f_1,\dots,x_n-f_n}$としてある。
\begin{myproof}
  $V=\var(I)$とする。

  まず、$\C$上で考える。
  パラメタで表示された図形$F(k^m)$は、先のグラフの利用により
  $\pi_m(i(k^m))=\pi_m(V)$と表示される。$V$はアフィン多様体で、
  $\pi_m(V)$はその射影なので、閉包定理によりこれを包む最小の多様体は
  $\var(I_m)$である。$\C$のときは証明おわり。

  $k\subset \C$上で考える。$k$は$\C$の1と$+$を持つので、
  無限体である。
  $\C$上の多様体と$k$上の多様体を区別するため、$\var_\C,\, \var_k$を考える。
  $V_k = \var_k(I),\, V_\C = \var_\C(I)$としてある。
  \begin{align}
    F(k^m) \desceq{グラフの射影} \pi_m(V_k) \descsubset{補題} \var_k(I_m).
  \end{align}
  これで$F(k^m)\subset \var_k(I_m)$は示された。あとは最小性を示せばよい。
  $F(k^m)$を包もうとすると一緒に$\var_k(I_m)$も包んでしまうことを示せばよい。
  $Z_k = \var_k(g_1,\dots,g_s) \subset k^n$を$F(k^m)$を包む多様体とする。
  各$i=1,\dots,s$について、$F(k^m)\subset Z_k$なので$g_i$は
  $F(k^m)$上消えてしまう。よって、$g_i\circ F$は$k^m$を消す。
  $g_i \in k[x_1,\dots,x_n]$であり、$F\in k[t_1,\dots,t_m]$なので、
  $g_i\circ F \in k[t_1,\dots,t_m]$であり、$g_i\circ F \in k[t_1,\dots,t_m]$
  である。
  $k$は無限体だと先に言ったので、$g_i\circ F$は多項式として0である。
  多項式として0なので、$(g_i\circ F)(\C^m) = 0$であり、
  $g_i$は$F(\C^m)$上で消える。よって、$F(\C^m) \subset Z_\C = \var_\C(g_1,\dots,g_s)$である。
  $\C$の場合の定理より、パラメタ表示$F(\C^m)$を包む最小の多様体は
  $\var_\C(I_m)$であるから、$\var_\C(I_m) \subset Z_\C$である。
  両方で$k^n$の結びをとって、
  \begin{align}
    \var_k(I_m)
    =
     \var_\C(I_m) \cap k^n
     \subset
     Z_\C \cap k^n
     =
     Z_k
  \end{align}
  である。これで、$F(k^m)$を含む多様体のうち最小のものは$\var_k(I_m)$であることが示された。

  一般の体については、その代数閉体を考えればよい。
\end{myproof}
登場人物:
\begin{itemize}
  \item $k$:無限体
  \item $F = (f_1,\dots,f_n)\colon k^m \to k^n$:図形のパラメタ。
  \item $I = \gen{x_1-f_1,\dots,x_n-f_n}$
  \item $V=\var(I) \subset k^{n+m}$、$F$のグラフ
  \item $V_k = \var_k(I)$:体$k$をとったときの$F$のグラフ($k$多様体)
  \item $V_\C = \var_\C(I)$:体$\C$をとったときの$F$のグラフ($\C$多様体)
  \item $Z_k$:$F(k^m)$を包む自由な多様体。
  \item $g_1,\dots,g_s \in k[t_1,\dots,t_m]$:$Z_k$を定義する自由な多項式。
  \item $Z_\C$:同じ$g_1,\dots,g_s$で定義される、$Z_k$に付随する多様体。$Z_k$より大きい。
\end{itemize}

次に有理パラメタ表示の陰関数表示を考える。
\begin{align}
  x_1 = \frac{f_1}{g_1},\dots,x_n=\frac{f_n}{g_n}
\end{align}
の陰関数表示を考える。
分母を処理するために、$g=g_1 \dots g_n$とし、パラメタ$y$を次のように導入し、
$k[y,t_1,\dots,t_m,x_1,\dots,x_n]$で考える。
「$g(x)$は0になる$\iff$ $g_1(x),\dots,g_n(x)$の1つ以上は0になる」
であり、「$g(x)$はnonzero$\iff$ $g_1(x),\dots,g_n(x)$はどれも0にならない」
となる。$W=\var(g)$とする。
\begin{align}
  J=\gen{g_1x_1-f_1,\dots,g_n x_n - f_n , 1-gy}
\end{align}
とする($1-gy$で$W$を避ける)。
\begin{align}
  j(t_1,\dots,t_m)=
  (\frac{1}{g(t_1,\dots,t_m)},
  t_1,\dots,t_m,
  \frac{f_1(t_1,\dots,t_m)}{g_1(t_1,\dots,t_m)},
  \dots,
  \frac{f_n(t_1,\dots,t_m)}{g_n(t_1,\dots,t_m)}).
\end{align}
こうしておくと、実は$j(k^m-W)=\var(J)$となる。
\begin{myproof}
  \begin{itemize}
    \item $\subset$:
    あきらか。実際、
    $(t_1,\dots,t_m) \in k^m-W$とする。
    \begin{align}
      j(t_1,\dots,t_m)=
      (\ub{\frac{1}{g(t_1,\dots,t_m)}}_{\Rightarrow y},t_1,\dots,t_m,
      \ub{\frac{f_1(t_1,\dots,t_m)}{g_1(t_1,\dots,t_m)}}_{\Rightarrow x_1},\dots, \ub{\frac{f_n(t_1,\dots,t_m)}{g_n(t_1,\dots,t_m)}}_{\Rightarrow x_n}).
    \end{align}
    これは$J$の生成元を考えれば、$\var(J)$に属する。
    \item $\supset$:
    $(y,t_1,\dots,t_m,x_1,\dots,x_n)\in \var(J)$とする。
    $yg(t_1,\dots,t_m)=1$となるので$g(t_1,\dots,t_m)$がnonzeroで、
    $g_1(t_1,\dots,t_m),\dots,g_n(t_1,\dots,t_m)$はどれもnonzeroになる。
    よって割り算が考えられて、
    $x_i = \frac{f_i(t_1,\dots,t_m)}{g_i(t_1,\dots,t_m)}$を考えられる。
    すると、$j(t_1,\dots,t_m)=(y,t_1,\dots,t_m,x_1,\dots,x_n)$となり、
    $(y,t_1,\dots,t_m,x_1,\dots,x_n)\in j(k^m-W)$となる。
  \end{itemize}
\end{myproof}
$F=\pi_{m+1}\circ j$なので、
\begin{align}
  F(k^m-W)=\pi_{m+1}(j(k^m-W))=\pi_{m+1}(\var(J))
\end{align}
となる。これで、パラメタ表示の図形をアフィン多様体の射影であらわせたので閉包定理が使える。有理陰関数表示化:「$k$を無限体とする。
$F\colon k^m - W \to k^n$を有理関数によるパラメタ付けとする。
$J$をイデアル
\begin{align}
    J=\gen{g_1x_1-f_1,\dots,g_nx_n-f_n,1-gy}\subset k[y,t_1,\dots,t_m,x_1,\dots,x_n]
\end{align}
とする。$g=g_1\dots g_n$とした。
$J_{m+1}=J\cap k[x_1,\dots,x_n]$を$(m+1)$次消去イデアルとする。このとき、
$\var(J_{m+1})$は$F(k^m-W)$を含む$k^n$の最小の多様体である。」
\begin{myproof}
  $k=\C$のときには$\pi_{m+1}(\var(J))=F(k^m-W)$と閉包定理よりあきらか。

  $k\subsetneq \C$とする。$k$は無限体である。
  $\C$上の多様体と$k$上の多様体を区別するため、$\var_\C,\var_k$を考える。
  $V_k = \var_k(J),\, V_\C = \var_\C(J)$とする。
  \begin{align}
    F(k^m-W) \desceq{さっきの} \pi_m(V_k)  \descsubset{射影が小さい補題} \var_k(J_{m+1}).
  \end{align}
  これで$F(k^m-W)\subset \var_k(J_{m+1})$は示された。
  あとは最小性を示せばよい。$Z_k = \var_k(h_1,\dots,h_s)\subset k^n$を
  $F(k^m-W)$を包む多様体とする。
  各$i=1,\dots,s$について、$F(k^m-W)\subset Z_k$なので、$h_i$は
  $F(k^m-W)$上消えてしまう。よって、$h_i\circ F$は$k^m-W$上消える。
  仮に$h_i\circ F$が0多項式でないとする。$g$は0多項式ではないので、
  $(h_i\circ F)\cdot g$は0多項式ではない。しかし、
  $h_i\circ F$は$k^m-W$上消え、$g$は$W$上消えるので、$(h_i\circ F)\cdot g$は
  $k^m$で消える。無限体上で0関数になる多項式は0多項式なので$(h_i \circ F)\cdot g$は
  0多項式である。これは矛盾であり、$h_i\circ F$は0多項式である。
  よって、$(h_i\circ F)(\C^m)=0$であり、$h_i$は$F(\C^m)$上で消える。
  よって、$F(\C^m)\subset Z_\C = \var_{\C}(I_m)\subset Z_\C$である。
  両方で$k^n$の結びをとって、
  \begin{align}
    \var_k(J_{m+1}) = \var_{\C}(J_{m+1})\cap k^n
    \subset
    \Z_\C \cap k^n = Z_k
  \end{align}
  である。これで、$F(k^m)$を含む多様体のうち最小のものは$\var_k(J_{m+1})$であることが示された。
\end{myproof}

\subsection{特異点と包絡線}
\label{sub:特異点と包絡線}
略

\subsection{因数分解の一意性と終結式}
\label{sub:因数分解の一意性と終結式}
定義1:
$k$を体とする。$f\in k[x_1,\dots,x_n]$が既約であるとは、
$f$が定数でない$k[x_1,\dots,x_n]$の2つの積で書けないことである
\footnote{$f=gh$と書いたとき$g=const.$か$h=const.$となること。}
。

命題2:すべての定数でない多項式$f\in k[x_1,\dots,x_n]$は、
$k$上で既約な多項式の積に分解できる。
\begin{myproof}
  \begin{enumerate}[label=(\arabic*)]
    \item Define $f$: $f\in k[x_1,\dots,x_n]$とする。既約ならば終わっているので、既約でないとする。
    \item Define $g,h$: 定数でない$g,h\in k [x_1,\dots,x_n]$で$f=gh$と分解する。
    \item $\deg g < \deg f$かつ$\deg h < \deg f$となっている。
    \item $g,h$が既約でないならさらに(2)のように分解する。これを繰替えすと次数が落ちていくので、どこかで停止し既約に分解される。
  \end{enumerate}
\end{myproof}

定理3:
$f\in k[x_1,\dots,x_n]$を$k$上既約な多項式とし、
$f$は積$gh$を割り切ると仮定する。ここで、$g,h\in k[x_1,\dots,x_n]$である。
このとき、$f$は$g$か$h$かのどちらかを割り切る。
\begin{myproof}
  \begin{enumerate}[label=(\arabic*)]
    \item $\implies$: 帰納法にする。$n=1$であるとする。
    \item $f$は$gh$を割り切るとする(仮定)。
    \item Define $p$: $p=\GCD(f,g)$とする。
    \item $\implies$: $p$が定数でないとする。
    \item (3)で$p$は$\GCD$なので、$p|f$であり、仮定より
    $f$は既約なので$p$は定数か$f$かだが、(4)より$f$は$p$の定数倍($f\simeq p$)である。
    \item (3)で$p$はGCDなので$p|g$であり、(5)より$f|g$である。
    \item (4)おわり。
    \item $\implies$: $p$は定数とする。$p=1$としてよい
    \footnote{GCDは定数倍はどうでもいい。}。
    \item Get $A,B$: (3)より、$Af+Bg = 1$となる$A,B\in k[x_1]$が存在する。
    \item (9)に$h$をかける。
    \begin{align}
      h = h(Af+Bg) = Ahf + Bgh.
    \end{align}
    \item (2)より$f|gh|Bgh$で、$f|Ahf$なので、(9)より$f|h$となる。
    \item (8)おわり。
    \item (1)おわり。$n=1$で示された。
    \item $\implies$: $n-1$で成立すると仮定する。
    \item Def $u$: $u\in k[x_2,\dots,x_n]$は既約で、$u|gh$とする。
    ($u$という特殊な$f$について結論「$u|g$または$u|h$を示す。」)
    \item $a_\bullet,b_\bullet$: を $g=\sum_{i=0}^l a_i x_1^i$とし、
    $h=\sum_{i=0}^m x_1^i$とする。
    \item
  \end{enumerate}
\end{myproof}

\subsection{終結式と拡張定理}
\label{sub:終結式と拡張定理}
終結式の定義:
$f,g\in k[x_1,\dots,x_n]$として、
\begin{align}
  f&=a_0x_1^l + \dots + a_l,\quad a_0\neq 0\\
  g&= b_0 x_1^m + \dots + b_m ,\quad b_0\neq 0
\end{align}
とする。これについて、終結式を
\easypicture{1431709536336.png}
とする。

命題1:
$f,g\in k[x_1,\dots,x_n]$の$x_1$に関する次数が正であると仮定する
\footnote{$\deg(f;x_1)>0$?}
。
\begin{enumerate}[label=(\roman*)]
  \item $\Res(f,g,x_1)$は$x_1$を消去した1次の消去イデアル
  $\gen{f,g}\cap k[x_2,\dots,x_n]$に含まれる。
  \item $\Res(f,g,x_1)=0$であることと、
  $f$と$g$が$k[x_1,\dots,x_n]$において$x_1$に関する
  次数が正の共通因子を持つことは同値である。
\end{enumerate}
\begin{myproof}
  (i)を示す。
  \begin{enumerate}[label=(\arabic*)]
    \item Get $l,m,a_\bullet,b_\bullet$:
    \begin{align}
      f &= a_0 x^l + \dots + a_l \\
      g &= b_0 x^m + \dots + b_m
    \end{align}
    ただし、$a_\bullet,b_\bullet \in k[x_2,\dots,x_n]$
    と書く。
    \item
    終結式$\Res(f,g,x_1)$は定義より$a_\bullet,b_\bullet$の積と和なので、
    $\Res(f,g,x_1)\in k[x_2,\dots,x_n]$である。
    \item
    Get $A,B$: $Af + Bg = \Res(f,g,x_1)$となる$A,B\in (k[x_2,\dots,x_n])[x_1]$が存在する
    \footnote{$\Res(f,g,x_1)=0$のときは$A=B=0$でよい。
    $\Res(f,g,x_1)\neq 0$のときには、$f,g\in k(x_2,\dots,x_n)[x_1]$とする。
    $A=c_0 x^{l-1} + \dots + c_{l-1},\,
    B=d_0 x^{m-1} + \dots + d_{m-1}$と、1つ低い次数で$c_\bullet,d_\bullet \in k(x_2,\dots,x_n)$の変数で表しておく。
    $Af+Bg=1$という方程式を考え、係数比較して
    $\Syl(f,g,x_1) (c_0,\dots,c_{l-1},d_0,\dots,d_{m-1})^T = (0,\dots,1)^T$
    という線型方程式を得る。$\Res(f,g,x_1)\neq 0$なのでこれは一意に解けて、
    クラメールの公式より
    各$c_\bullet,\,d_\bullet$は$\det(\Syl(f,g,x_1)のある行を(0,\dots,1)^T に交換)/\Res(f,g,x_1)$である。よって、$A,B$の係数はすべて
    $(\Syl(f,g,x_1)の積と和)/\Res(f,g,x_1)$である。よって、$A=\tilde A/\Res(f,g,x_1),\, B=\tilde B/\Res(f,g,x_1)$で、
    $\tilde A,\tilde B$は$\Syl(f,g,x_1)$の積と和となるものがある。
    よって、$Af+Bg=1$にこれを入れて$\tilde A f + \tilde B g = \Res(f,g,x_1)$となる。
    }
    。
    \item
    (3)より、$\Res(f,g,x_1)=Af+Bg \in \gen{f,g}$である。
    \item (2)(4)より、$\Res(f,g,x_1) \in \gen{f,g}\cap k[x_2,\dots,x_n]$となる。
  \end{enumerate}
  (ii)を示す。
  \begin{enumerate}[label=(\arabic*)]
    \item Section5-Prop8
    \footnote{1変数多項式について、共通因子と終結式が消えることの同値}
    を$f,g \in k[x_1,\dots,x_n] \subset k(x_2,\dots,x_n)[x_1]$に適用し、「
    $\Res(f,g,x_1)=0 \iff $ $f$と$g$は$x_1$に関して正の次数を持つ$k(x_2,\dots,x_n)[x_1]$の多項式を共通因子として持つ」となる。
    \item
    Section5-Cor4を適用して、「$f,g$は$k(x_2,\dots,x_n)[x_1]$で共通因子を持つ$\iff$ $f,g$は$k[x_1,\dots,x_n]$で共通因子を持つ」となる。
    \item (1),(2)より、
    「$\Res(f,g,x_1)=0 \iff$ $f,g$は$k[x_1,\dots,x_n]$で共通因子を持つ」となる。
  \end{enumerate}
\end{myproof}

系2:$f,g\in \C[x]$とする。このとき、$\Res(f,g,x)=0$であることと、
$f$と$g$が$\C$において共通根を持つことは同値である。」
\begin{myproof}
  $\C[x]$で2つの共通因子を持つことと、共通根を持つことは同値である。
\end{myproof}

命題3:$f,g\in \C[x_1,\dots,x_n]$に対して、$a_0,b_0\in k[x_2,\dots,x_n]$を
\begin{align}
  f &= a_0 x^l + \dots + a_l,\quad a_0\neq 0\\
  g &= b_0 x^m + \dots + b_m,\quad b_0\neq 0.
\end{align}
ととる。もし$\Res(f,g,x_1) \in \C[x_2,\dots,x_n]$が
$(c_2,\dots,c_n)\in \C^{n-1}$において消えるとすると次が成立する。
\begin{enumerate}[label=(\roman*)]
  \item $a_0$または$b_0$が$(c_2,\dots,c_n)\in \C^{n-1}$で消える。
  \item $c_1 \in \C$が存在して、$f$と$g$は
  $(c_1,\dots,c_n)\in \C^n$で消える。
\end{enumerate}
\begin{myproof}
  \begin{enumerate}[label=(\arabic*)]
    \item Def $\bbold c$: $\bbold c = (c_2,\dots,c_n)$とする。
    \item $\implies$: $a_0(\bbold c)\neq 0$かつ$b_0(\bbold c)\neq 0$だとする。(結論の片方を持ってくる。)
    \item
    (2)の仮定より、
    \begin{align}
      f(x_1,\bbold c)&=
      a_0(\bbold c)x_1^l + \dots + a_l(\bbold c),\quad a_0(\bbold c)\neq 0\\
      g(x_1,\bbold c)&=
      b_0(\bbold c)x_1^m + \dos + b_m(\bbold c),\quad b_0(\bbold c)\neq 0
    \end{align}
    となっている。
    \item
    Def $h$: $h=\Res(f,g,x_1)$とする。
    \item
    仮定より$h = \Res(f,g,x_1) = 0$となる。
    \item
    \easypicture{1431792619123.png}
    \item
    $f,g$の$a_\bullet,b_\bullet$の表現より、
    $f(x_1,\bbold c)$と$g(x_1,\bbold c)$の終結式は、上の
    行列式である。
    \item 上2つより、
    \begin{align}
      0 = h(\bbold c) = \Res(f(x_1,\bbold c),g(x_1,\bbold c),x_1).
    \end{align}
    \item Get $c_1$:
    系2より、$f(x_1,\bbold c)$と$g(x_1,\bbold c)$は共通根を持つ。
    この$x_1$を$c_1 = x_1$とおく。
    \item
    (9)より、
    $f(c_1,\bbold c) = g(c_1,\bbold c) = 0$となる。$c_1$が求めるものだった。
    \item (2)おわり。前件を後ろに否定して「または」で結論を得る。
  \end{enumerate}
\end{myproof}

定理4(2つの多項式に対する拡張定理):
$I=\gen{f,g}\subset \C[x_1,\dots,x_n]$とし、
$I_1$を$I$の1次の消去イデアルとする。
また、$a_0,b_0 \in \C[x_2,\dots,x_n]$を
\begin{align}
  f &= a_0 x^l + \dots a_l,\quad a_0 \neq 0 \\
  g &= b_0 x^m + \dots b_m, \quad b_0\neq 0
\end{align}
のものとする。
部分解$(c_2,\dots,c_n)\in \var(I_1)$があるとする。
もし、$(c_2,\dots,c_n) \notin \var(a_0,b_0)$ならば、
$c_1\in \C$が存在して$(c_1,\dots,c_n) \in \var(I)$となる
\footnote{拡張できた。}
。
\begin{myproof}
\begin{enumerate}[label=(\arabic*)]
  \item Def $\bbold c$: $\bbold c = (c_2,\dots,c_n)$とする。
  \item 命題1より、$\Res(f,g,x_1) \in I_1$となる。
  \item 上より、
  $\bbold c = (c_2,\dots,c_n) \in \var(I_1)$なので、
  終結式$\Res(f,g,x_1)$は$\bbold c$で消える。
  \item
  $\implies$: $a_0(\bbold c) \neq 0$かつ$b_0(\bbold c)\neq 0$とする。
  \item Get $c_1$:
  命題3より拡張した$c_1$がある。$(c_1,\bbold c) \in \var(f,g)$となる。
  \item (3)おわり。
  \item $\implies$: $a_0(\bbold c),b_0(\bbold c)$の一方が0でもう一方が0でないとする。$a_0(\bbold c) \neq 0$かつ$b_0(\bbold c)=0$として一般性をうしなわない。
  \item Def $N$: $N$を十分大きいとする。
  $x_1^N f$の$x_1$についての次数は$g$の$x_1$についての次数より大きい。
  \item $\gen{f,g} = \gen{f,g+x_1^N}$となる。
  \item $g+x_1^N f$の$x_1$についてのLTは$a_0$になっている。この係数は
  (7)より$a_0(\bbold c)\neq 0$である。
  \item Get $c_1$: 上より(4)-(6)が$f,g+x_1^N f$に適用でき、
  $(c_1,\bbold c) \in \var(f,g+x_1^N f)$となる。
  \item 上と(9)より、
  $(c_1,\bbold c) \in \var(f,g)$となる。
  \item (7)おわり。
  \item (4)-(6),(7)-(13)が示されたものだった。
\end{enumerate}
\end{myproof}

一般終結式を定義する。
3変数以上に対する終結式を定義したかった。
$f_1,\dots,f_s$の一般終結式を、変数$x_1,\dots,x_n$に$u_2,\dots,u_s$を追加して、
\begin{align}
  \Res(f_1, u_2f_2+\dots + u_s f_s, x_1)
  =
  \sum_\alpha h_\alpha(x_2,\dots,x_n)u^\alpha
\end{align}
と書いたときの$h_\alpha$たちと定義する
\footnote{$f_1,f_2$なら$u_2$だけ追加する。$\alpha$は1個になって、
$\Res(f_1,u_2f_2,x_1)=h_{e_2}(x_2,\dots,x_n)u^{\alpha_2}$となる。}
。

定理5:
拡張定理。
イデアル$I=\gen{f_1,\dots,f_s}\subset k[x_1,\dots,x_n]$をとり、
$I_1$を$I$の1次の消去イデアルとする。
各$1\le i \le s$にたいして、$f_i$を次の形に書く
\footnote{$g_i \in k[x_2,\dots,x_n]$は$f$の$x_1$に関する最高次}。
\begin{align}
  f_i = g_i(x_2,\dots,x_n)x_1^{N_i} + (x_1 の次数が <N_i である頃).
\end{align}
ここで、$N_i \ge 0$であり、$g_i \in \C[x_2,\dots,x_n]$は0でない。
部分解$(c_2,\dots,c_n) \in \var(I_1)$であると仮定する。
もし$(c_2,\dots,c_n)\notin \var(g_1,\dots,g_s)$であるならば、
$c_1\in \C$が存在して、$(c_1,\dots,c_n)\in \var(I)$となる。

\begin{myproof}
  \begin{enumerate}[label=(\arabic*)]
    \item Def $\bbold c$: $\bbold c = (c_2,\dots,c_n)$とする。
    \item ($f_1(x_1,\bbold c),\dots,f_s(x_1,\bbold c)$の共通根を求めたい。)
    \item $\implies$: $s\ge 3$とする。
    \item 仮定の「多様体に属さない」の仮定より、$\bbold c\notin \var(g_1,\dots,g_s)$とする。一般性を失なわず、
    $g_1(\bbold c)\neq 0$と仮定してよい\footnote{$g_1(\bbold c)=\dots=g_s(\bbold c)=0$だとするとおかしいので少なくとも1個はnonzeroでないといけず、それを1番にした。}。
    \item Get $h_\alpha$: $f_1,\dots,f_s$の一般終結式を名付ける。
    \begin{align}
      \Res(f_1,u_2f_2+\dots+u_sf_s,x_1)
      =
      \sum_\alpha h_\alpha u^\alpha
    \end{align}
    とする。
    \item Get $A,B$:
    命題1の終結式の性質より、$A,B\in \C[u_2,\dots,u_s,x_1,\dots,x_n]$で
    \begin{align}
      Af_1 + B(u_2f_2 + \dots + u_s f_s) =
      \Res(f_1,u_2f_2 + \dots + u_s f_s, x_1)
    \end{align}
    となるものが存在する。
    \item
    Get $A_\bullet, B_\bullet$:
    $A=\sum_{\alpha}A_\alpha u^\alpha,\quad
    B=\sum_{\beta}B_\beta u^\beta$とおく。
    ここで、$A_\alpha,B_\beta \in \C[x_1,\dots,x_n]$である。
    \item
    ($h_\alpha \in \gen{f_1,\dots,f_s} = I$を示す。 )
    \item
    Def $e_\bullet$:
    \begin{align}
      e_2 = (1,0,\dots,0),\dots, e_s = (0,\dots,0,1)
    \end{align}
    とする。ここで長さは$n-1$。
    \item
    (9)の定義を使って、
    \begin{align}
      u_2 f_2 + \dots + u_s f_s = \sum_{i\ge 2}u^{e_i}f_i.
    \end{align}
    \item
    \begin{align}
      \sum_\alpha h_\alpha u^\alpha
      &\desceq{(5),(6),(10)}
      (\sum_\alpha A_\alpha u^\alpha)f_i
      +
      (\sum_\beta B_\beta u^\beta)(\sum_{i\ge 2}u^{e_i}f_i)\\
      &=
      \sum_\alpha (A_\alpha f_1)u^\alpha
      +
      \sum_{i\ge 2,\bea}B_\beta f_i u^{\beta + e_i}\\
      &=
      \sum_\alpha(A_\alpha f_1)u^\alpha
      +
      \sum_\alpha (\sum_{\substack{i\ge 2,\beta \\ \beta+e_i = \alpha}}B_\beta f_i)u^\alpha\\
      &=
      \sum_\alpha(A_\alpha f_1 + \sum_{\substack{i\ge 2,\beta \\ \beta+e_i = \alpha}}B_\beta f_i)u^\alpha.
    \end{align}
    \item Fix $\alpha$.
    \item
    上の両辺の$u^\alpha$係数を比較し\footnote{decode}、
    \begin{align}
      h_\alpha = A_\alpha f_1 + \sum_{\substack{i\ge 2,\beta \\ \beta+e_i = \alpha}} B_\beta f_i.
    \end{align}
    \item
    $I=\gen{f_1,\dots,f_s}$だったので、上の式は$h_\alpha$をこれらの結合であらわしていることから$h_\alpha \in I$である。
    \item
    Free (12), $\alpha$。任意の$\alpha \in \Zge^{n-1}$について、
    $h_\alpha \in I$となる。
    \item
    (5)より、$h_\alpha \in \C[x_2,\dots,x_n]$だったので、
    上とあわせて$h_\alpha \in I_1$となる。
    \item
    仮定(部分解)より、$\bbold c\in \var(I_1)$なので、
    上より$h_\alpha(\bbold c)  = 0$である。
    \item
    Def $h$: $h=\Res(f_1,u_2f_2+\dots + u_s f_s, x_1)$とする。
    \item
    \begin{align}
      h(\bbold c)
      \desceq{(18)}
      \Res(f_1,u_2f_2+\dots+u_sf_s,x_1)(\bbold c)
      \desceq{(5)}
      \sum_\alpha (h_\alpha u^\alpha)(\bbold c)
      \desceq{(17)}
      0.
    \end{align}
    すなわち、終結式が$\bbold c$で消える。
    \item
    Def $h(\bbold c,u_2,\dots,u_n)$:$\C[x_1,u_2,\dots,u_s]$の多項式
    $h(\bbold c,u_2,\dots,u_n)$を、
    終結式$h$に$(x_2,\dots,x_n) \leftarrow \bbold c$を代入して
    得られた多項式と定義する。(実際は$x_1$もないよね?)
    \item
    (19),(20)より、$h(\bbold c,u_2,\dots,u_n) = 0$となる。
    \item
    $\implies$:
    $g_2(\bbold c)\neq 0$かつ$f_2$は変数$x_1$に関する次数が$f_3,\dots,f_s$のどれよりも大きいとする。
    \item
    \begin{align}
      h(\bbold c,u_2,\dots,u_n) =
      \Res(f_1(x_1,\bbold c), u_2f_2(x_1,\bbold c)+\dots+u_sf_s(x_1,\bbold c),x_1).
    \end{align}
    (「終結式を計算してから代入」と「代入してから終結式」が一致する)
    これは、仮定より$f_1$の最高次が$\bbold c$で消えないこと、
    (22)の仮定より$u_2f_2+\dots+u_sf_s$の最高次が$u_2g_2$であり、
    再び(22)の仮定より$g_2(\bbold c\neq 0$であることから従う。
    \item (19)と(23)より、
    \begin{align}
      \Res(f_1(x_1,\bbold c),u_2f_2(x_1,\bbold c)+\dots+u_sf_s(x_1,\bbold c),x_1)=0.
    \end{align}
    \item Get $F$:
    $f_1(x_1,\bbold c)$も$u_2f_2(x_1,\bbold c)+\dots+u_sf_s(x_1,\bbold c)$もどちらも$k[x_1,u_2,\dots,u_s]$
    の元なので、命題1が使えて(24)より、
    この2式に$x_1$について正の次数を持った共通因子$F$が得られる。
    \item $F$の定義より、$F$は$f_1(x_1,\bbold c)$を割り切る。
    よって、$F\in \C[x_1]$である。
    \item $F$は$u_2f_2(x_1,\bbold c)+\dots+u_sf_s(x_1,\bbold c)$を割り切るが、$f_\bullet(x_1,\bbold c)\in \C[x_1]$であることと、
    $u_\bullet \in \C[u_2,\dots,u_n]$であることから(正確には係数比較して)、
    $F$は$f_2(x_1,\bbold c),\dots,f_s(x_1,\bbold c)$をすべて割り切る。
    \item (26),(27)より、$F$は$f_1(x_1,\bbold c),\dots,f_s(x_1,\bbold c)$すべての、$x_1$について正の次数を持つ共通因子である。
    \item
    Def $c_1$: (25)より$F$は$x_1$について正の次数を持つので、$F$の根$c_1 \in \C$が存在する(代数閉体)。
    \item
    (27),(28)より、$c_1$は$f_i(x_1,\bbold c)$すべての共通根であり、
    部分解$\bbold c$が拡張されて$(c_1,\bbold c)$となった。
    \item (22)おわり。
    \item $\implies$: (22)がみたされないとき:
    \item Get $N$: 十分大きい
    \item (33)より、$f_2+x_1^N f_1$の$x_1$に関する最高次の係数は$g_1$である。
    \item (33)より、$f_2+x_1^N f_1$の$x_1$に関する次数は$f_3,\dots,f_s$のどれよりも大きい。
    \item Get $c_1$: 上2つより、(22)-(31)の議論が$f_1,f_2+x_1^N f_1,f_3,\dots,f_s$に適用でき、これに関する$\bbold c$の拡張$c_1$を得る。
    \item 上より、
    \begin{align}
      f_1(x_1,\bbold c) &= 0\\
      (f_2+x^Nf_1)(x_1,\bbold c) &= 0\\
      &\vdots\\
      f_s(x_1,\bbold c) &= 0
    \end{align}
    となっている。引き算して、$f_2(x_1,\bbold c)=0$も得られ、
    $c_1$は$\bbold c$の拡張になっている。
    \item (32)おわり。
    \item (31),(38)より、示された。
  \end{enumerate}
\end{myproof}
