\label{sec:射影代数幾何}


\subsection{射影平面}
\label{sub:射影平面}
\subsection{射影空間と射影多様体}
\label{sub:射影空間と射影多様体}
\subsection{射影化された代数-幾何対応}
\label{sub:射影化された代数-幾何対応}
\subsubsection{定理2:斉次イデアルの特徴付け}
\label{subs:定理2:斉次イデアルの特徴付け}
イデアルについて、以下同値。
\begin{enumerate}[label=(\roman*)]
  \item $I$は$k[x_0,\dots,x_n]$の斉次イデアル
  \item 斉次多項式$f_\bullet$を用いて$I=\gen{f_1,\dots,f_s}$
  \item 任意の多項式順序について、$I$の簡約グレブナ基底は斉次多項式からなる。
\end{enumerate}

\pf 斉次多項式で生成される$\implies$斉次イデアル:
イデアルから$f$をとって、
\begin{align}
  \sum_{d} \ub{g_d}_{d次}
  =
  f
  =
  \sum_i c_i f_i
\end{align}
と2通りに分解して、右側の$c_i$も分解して比べる。

\pf 斉次イデアル$\implies$斉次多項式で生成される:
イデアル$I$をヒルベルトの基底定理で
$I=\ge{F_1,\dots,f_t}$と書いて、
各$F_i$を分解すると、それらで生成されるイデアルが$I$と一致する。

\pf 斉次多項式で生成される$\implies$ 簡約グレブナ基底は斉次多項式からなる:
斉次多項式で生成されるので、
これにブッフベルガーをかますと、斉次多項式からなる
グレブナ基底は得られる。
割り算アルゴリズムを考えると、簡約化で余りを取るステップでも
斉次であることが保たれる。

\subsubsection{命題3:イデアルで生成される射影多様体}
\label{subs:命題3:イデアルで生成される射影多様体}
$\var(I) = \var(f_1,\dots,f_s)$

\pf
\begin{align}
  \var(\gen{f_1,\dots,f_s}) =
  \set{どの\gen{f_1,\dots,f_s}でも消える点}
  =
  \set{どのf_1,\dots,f_sでも消える点}
  =
  \var(f_1,\dots,f_s).
\end{align}

\subsubsection{命題4: 射影多様体からイデアル}
\label{subs:命題4: 射影多様体からイデアル}
\begin{align}
  \ideal(V) = \set{Vを消す関数}
\end{align}
とすると、$k$を無限体とすると$\ideal(V)$は斉次イデアルとなり、
$\ideal \colon \set{射影多様体} \to \set{斉次イデアル}$
が定義される。

\pf:
斉次を示す。$\ideal(V)$から元をとって$f$としておく。
\begin{align}
  f(\lambda a_0,\dots,\lambda a_n)
  =
  \sum_i \lambda^i f_i(a_0,\dots,a_n)
\end{align}
を$\lambda$についての方程式と見ると、無限体なので
各$f_i(a_0,\dots,a_n)=0$になって、各斉次成分$f_i$も$\ideal(V)$に属す。


\subsubsection{定理5: イデアルをとることの単射性}
\label{subs:定理5: イデアルをとることの単射性}
$k$を無限体とすると、$\ideal,\var$は逆転する。
また、$\var(\ideal(V))=V$となる。

\begin{itemize}
  \item $\ideal$の逆転:
  たくさんの点を消すには条件が少ないほうがいいので
  イデアルは小さいほうがいい。
  \item $\var$の逆転:
  少しの関数だと沢山の点が消えるので逆転する。
  \item
  $\var(\ideal(V))\subset V$:
  \begin{align}
    \var(\ideal(V))
    =
    \var(\ideal(\var(I)))
     \subset
    \var(I)
    =
    V.
  \end{align}
  \item
  $\var(\ideal(V))\supset V$:
  $I$が消す点は、「$I$が消す点」を消す関数が消す点?
\end{itemize}

\subsubsection{命題7: 斉次イデアルのラジカルは斉次イデアル}
\label{subs:命題7: 斉次イデアルのラジカルは斉次イデアル}
$\sqrt{I}$から元$f$を選んで、
$f=f_{max} + f'$と書いたら、$f^n$の先頭項も$f_{max}^n$
となる。あとはこれを引いて同じ操作を繰替えす。

\subsubsection{定理8: 射影幾何における弱系の零点定理}
\label{subs:定理8: 射影幾何における弱系の零点定理}
$k$を代数的兵隊として$I$を斉次イデアルとする。

\pf
$C_V=\var_a(I)$を、$I$で定義されるアフィン多様体とする。
(射影多様体のユークリッド空間での実現)

すべての$x_i$について、$x_i^{何か}$が$I$のグレブナ基底に入っている
$\implies$ $\var(I)$は空:
\pf
グレブナ基底の条件を定理6-3-6に使えば、
$C_V$が有限集合になる(lexなら簡単。)。$\var(I)$に何か点があったとすると、
その定数倍すべてが$\var(I)$に属することになり、有限性に反する。

すべての$x_i$について、$x_i^{何か}$が$I$に入っている$\implies$
すべての$x_i$について、$x_i^{何か}$が$I$のグレブナ基底に入っている:
\pf $x_i^m \in I$とする。
\begin{align}
  x_i^m = \LT(x_i^m) \in \LT(I)\subset
  \gen{LT(I)} = \gen{\LT(G)}.
\end{align}

ある$r\ge 1$について$\gen{x_0,\dots,x_n}^r \subset I$ $\implies$
すべての$x_i^{何か} \in I$:
\pf
あきらか。

$\var(I)$が空 $\implies$ ある$r\ge 1$について$\gen{x_0,\dots,x_n}^r \subset I$:
\pf
\begin{align}
  \gen{x_0,\dots,x_n}
  =
  \ideal_a(0)
  \descsubset{空だから}
  \ideal_a(C_V)
  \desceq{零点定理}
  \sqrt{I}.
\end{align}

\subsubsection{定理9: 射影幾何における強系の零点定理}
\label{subs:定理9: 射影幾何における強系の零点定理}
\begin{align}
  \sqrt{I}
  \desceq{零点定理}
  \ideal_a(\var_a(I))
  \desceq{簡単}
  \ideal(V)
  =
  \ideal(\var(I)).
\end{align}

\subsubsection{定理10: 射影多様体の$\ideal $と$\var$}
\label{subs:定理10: 射影多様体の$\ideal $と$\var$}

\subsection{アフィン多様体の射影完備化}
\label{sub:アフィン多様体の射影完備化}
\subsubsection{定義1: イデアルの斉次化}
\label{subs:定義1: イデアルの斉次化}
$I$を$x_1$から$x_n$に対して、この斉次化を
$I^h = \gen{f^h; f\in I}$とする(こっちは$x_0$があるかも)。

\subsubsection{命題2: 斉次化は斉次イデアル}
\label{subs:命題2: 斉次化は斉次イデアル}
$g\in \gen{f^h; f\in I}$とする。
\begin{align}
  g
  =
  \sum_{i=1}^N F_i f_i^h
  =
  \sum_{i=1}^N \sum_j F_{ij} \ub{f_{ij}^h}_{i次のやつのj番目}
  =
  \sum_{i=1}^N \sum_j \ub{(\sum_l F_{ijl})}_{F_{ij} の斉次分解} f_{ij}^h
  =
  \sum_d \sum_{i+l=d}\sum_j \ub{F_{ijl}}_{i次} \ub{f_{ij}^h}_{l次}
\end{align}

\subsubsection{定理4: グレブナ基底の斉次化}
\label{subs:定理4: グレブナ基底の斉次化}
$I$をイデアルとして、$G$を次数つき順序についてのグレブナ基底とすると、
$G^h$は$I^h$の基底。

\pf
\begin{itemize}
  \item $x_1,\dots,x_n$を所与の順序を優先し、次に$x_0$で評価する
  単項式順序$>_h$を考える。この順序で$G^h$がグレブナ基底である
  ことを言えば十分。
  \item 条件として、$G^h \subset I^h$と$\gen{\LT(I^h)} \subset \gen{\LT(G^h)}$
  があるが、後者を示せばよい。
  \item
  $F \in I^h$として、
  \begin{align}
    \LM_{>_h}(F) =
    \LM_{>_h}(\ub{x_0^e}_{Fに吐かせたe全部} \ub{f^h}_{非斉次化の斉次化})
    =
    x_0^e \LM_{>_h}(f^h)
    =
    x_0^e \LM_>(\ub{f}_{\in I})
    \divided
    \LM_>(\ub{g_i}_{\in G})
    =
    \LM_{>_h}(g_i^h).
  \end{align}
  よりこれは従う。
\end{itemize}

\subsubsection{定義6}
\label{subs:定義6}
アフィン多様体$W$について、$W$の射影完備化とは、
$\ol W = \var(\ideal_a (W)^h)$のこと。

\subsubsection{命題7}
\label{subs:命題7}
$W$をアフィン多様体とし、$\ol W$を射影完備化とする。

(i)$\ol W \cap U_0 = \ol W \cap k^n =  W$。
\pf $G$を$\ideal_a(W)$のグレブナ基底とする。
\begin{align}
  \ol W \cap U_0
  &=
  \var(\ideal_a(W)^h) \cap U_0\\
  &=
  \var(g^h; g\in G)\cap U_0\\
  &=
  \var(g^h; g\in G)\cap \var_a(x_0=1)\\
  &=
  \var_a(g^h(1,x_1,\dots,x_n); g\in G)\\
  &\desceq{斉次化して非斉次化している}
  \var_a(g; g\in G)\\
  &\desceq{定義}
  \var_a(\ideal_a(W))\\
  &=
  W.
\end{align}

(ii)$\ol W$は$W$を含む最小の射影多様体。
\pf
$W$を含む射影多様体広いやつ$V$を考える。
広いやつ$V$を構成する関数たちは$V$を消すので、もとのやつ$W$も消し、
もとのやつ$W$はアフィンだったので広いやつ$V$を構成する関数たちの
非斉次化ももとのやつ$W$を消す。よって、
広いやつ$V$を構成する関数たちの非斉次化の斉次化は$\ol W$を消す。
関数はその非斉次化の斉次化の倍数なので、
広いやつ$V$を構成する関数たちは$\ol W$を消す。
よって、広いやつ$V$は$\ol W$を常に包むので、
$\ol W$が$W$を含むやつのうち最小。

\subsection{射影的消去理論}
\label{sub:射影的消去理論}

\subsection{2次超曲面の幾何}
\label{sub:2次超曲面の幾何}
\subsubsection{命題1: 射影同値}
\label{subs:命題1: 射影同値}
~は簡単に示せる: $f_i$が$d$次だとしてなんとかする。

平方完成の応用:本当に平方完成でよさそう
